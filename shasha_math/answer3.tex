\documentclass[a4paper]{ctexart}
\usepackage{geometry}
\usepackage{multicol}
\usepackage{pstricks}
\usepackage{pst-plot}
\usepackage{color}
\usepackage{amsmath}
\geometry{left=2cm,right=2cm,top=2.5cm,bottom=2.5cm}

\setCJKmainfont[BoldFont={SimHei},ItalicFont={KaiTi}]
  {FangSong}

\author{Chunwei Yan}
\title{全书解答3}
\begin{document}
    \maketitle
%---content here----
\begin{multicols}{2}
%------ content --------------
\section{多元函数极限连续偏导}
\par
\textbf{底下你需要明白在$xoy$二维平面上,一点到另外一点可以有不同路径的由来就可以了。}
通常说的二元函数的图形是一张曲面
$$
z = f(x,y)
$$
\begin{center}
\begin{pspicture}(-4,-2)(4,4)
\psaxes[labels=none,ticks=none]{->}(0,0)(-3.5,-2.7)(3.5,4)
%--坐标轴
%\psgrid[subgriddiv=1,griddots=10,gridlabels=7pt](-3,-3)(3,3)
\psline[linewidth=1pt,linearc=0]{->}(0, 0)(-2.5, -2.5)
\psellipse[linestyle=dashed,dash=3pt 2pt](0,0)(2, 1)
\psellipse[linestyle=dashed,dash=3pt 2pt](0,2.5)(2, 1)
%- 坐标轴标定
\rput[bl](0, 0){O}
\rput[bl](3, .2){\zihao{4}$y$}
\rput[br](-2.5, -2.5){\zihao{4}$x$}
\rput[bl](.2, 4){\zihao{4}$z$}
% 两个圈子的连线
\psline[linestyle=dashed,dash=3pt 2pt, linewidth=1pt,linearc=0]{-}(-2, 0)(-2, 2.5)
\psline[linestyle=dashed,dash=3pt 2pt, linewidth=1pt,linearc=0]{-}(2, 0)(2, 2.5)
\psline[linewidth=1pt,linearc=1.5](-2, 2.5)(0, 4)(2, 2.5)
% 图像标注
\rput[bl](1.5, 2){\zihao{3}$\alpha$}
\rput[bl](1.5, .2){\zihao{3}$\beta$}
\end{pspicture}
\end{center}

\subsection{多元函数极限}
\par 
\begin{enumerate}
    \item 回顾一元函数的极限,只有一个自变量$x$:$x\rightarrow x_0$,有$f(x) \rightarrow A$, A为一个常数
    \item 二元函数的二重极限,有两个自变量
            $x,y$:$\{x\rightarrow x_0, y\rightarrow y_0;\} 有f(x,y) \rightarrow P_0(x_0, y_0)$ $P_0$为一个定点
\end{enumerate}
\par
但是需要注意,在二元函数里面,由于$x,y$存在于二维空间$xoy$中,在二维空间上$(x,y)$趋近于$P_0(x_0, y_0)$的路径多种多样----$x,y$同时变化,(在一元函数上,如$y=f(x)$,$x$趋近于某一个点$x=a$的路径只能为一条直线).
%-- 图样
\begin{center}
\begin{pspicture}(-4,-3)(4,4)
%\psgrid[subgriddiv=1,griddots=10,gridlabels=7pt](-3,-3)(3,3)
\psaxes[labels=none,ticks=none]{->}(0,0)(-3.5,-2.7)(3.5,4)
%--坐标轴
\psgrid[subgriddiv=1,griddots=10,gridlabels=7pt](-3,-3)(3,3)
\psline[linewidth=1pt,linearc=0]{->}(0, 0)(-2.5, -2.5)
\psellipse[linestyle=dashed,dash=3pt 2pt](0,0)(2, 1)
\psellipse[linestyle=dashed,dash=3pt 2pt](0,2.5)(2, 1)
%- 坐标轴标定
\rput[bl](0, 0){O}
\rput[bl](3, .2){\zihao{4}$y$}
\rput[br](-2.5, -2.5){\zihao{4}$x$}
\rput[bl](.2, 4){\zihao{4}$z$}
% 两个圈子的连线
\psline[linestyle=dashed,dash=3pt 2pt, linewidth=1pt,linearc=0]{-}(-2, 0)(-2, 2.5)
\psline[linestyle=dashed,dash=3pt 2pt, linewidth=1pt,linearc=0]{-}(2, 0)(2, 2.5)
\psline[linewidth=1pt,linearc=1.5](-2, 2.5)(0, 4)(2, 2.5)
% 图像标注
\rput[bl](1.5, 2){\zihao{3}$\alpha$}
\rput[bl](1.5, .2){\zihao{3}$\beta$}
% 加入点
\rput[bl](.5, .5){\zihao{3}$P_0'$}
\rput[bl](.5, 3){\zihao{3}$P_0$}
\psline[linestyle=dashed,dash=3pt 2pt, linewidth=1pt,linearc=0]{->}(.5, .5)(.5, 3)
% 路径趋向
% 曲面
\psline[linewidth=1pt,linearc=0,linecolor=red]{->}(0, 2.5)(.5, 3)
\psline[linewidth=1pt,linearc=.5,linecolor=red]{->}(-.5, 3.5)(0, 4)(.5, 3)
\psline[linewidth=1pt,linearc=.5,linecolor=red]{->}(1, 2.5)(1, 3.2)(.5, 3)
\psdots(.5, 3)
% x y面
\psline[linestyle=dashed,dash=3pt 2pt, linewidth=1pt,linearc=0,linecolor=red]{->}(0, .5)(.5, .5)
\psline[linewidth=1pt,linearc=0,linecolor=red]{->}(0, -.5)(.5, .5)
\psline[linewidth=1pt,linearc=.5,linecolor=red]{->}(-.5, .5)(0, 1)(.5, .5)
\psline[linewidth=1pt,linearc=.5,linecolor=red]{->}(1, -.5)(1, .2)(.5, .5)
\psdots(.5, .5)
% 路径标注
\rput[bl](0.2, -.7){\zihao{4}$l_1$}
\rput[bl](-.5, .5){\zihao{4}$l_2$}
\rput[bl](1.2, -.5){\zihao{4}$l_3$}
\end{pspicture}
\end{center}
\par
如图中,曲线中,点$P_0(x_0,y_0)$周围的点$P(x,y)$的趋向于$P_0$的路径多样,如图中红色轨迹。
\par
看上图的俯视图,只看$xoy$平面上,$P_0(X_0, y_0)$附近点的集合\{$(x,y)$\}趋近于它的轨迹:
\begin{center}
\begin{pspicture}(-4,-3)(4,4)
%\psgrid[subgriddiv=1,griddots=10,gridlabels=7pt](-3,-3)(3,3)
\psaxes[labels=none,ticks=none]{->}(0,0)(-3.5,-2.7)(3.5,4)
%--坐标轴
\psgrid[subgriddiv=1,griddots=10,gridlabels=7pt](-3,-3)(3,3)
\pscircle(0,0){2.5}
%- 坐标轴标定
\rput[bl](0, 0){O}
\rput[bl](3, .2){\zihao{4}$x$}
\rput[bl](.2, 3){\zihao{4}$y$}
\psline[linestyle=dashed,dash=3pt 2pt,linewidth=1pt,linearc=1,linecolor=red]{<-}(-.5,.5)(.5,.2)(1,1.5)(1.5,2)
\psline[linestyle=dashed,dash=3pt 2pt,linewidth=1pt,linearc=1,linecolor=red]{->}(-2,1)(-1,.7)(-.5,.5)
\psline[linestyle=dashed,dash=3pt 2pt,linewidth=1pt,linearc=1,linecolor=red]{<-}(-.5,.5)(1,-1)(1.5,-1.7)
\psdots(-.5, .5)
\rput[bl](-.5, .5){$P_0'$}


\end{pspicture}
\end{center}
\par
现在你知道在二元函数的空间里面(可以扩展到多元函数),趋近于一点的路径多种多样,如图中,设置可以360度,随便一个轨迹,只要最终趋近到$P_0$就可以了。 
\par
而要判断二元函数$y=f(x,y)$在某一点$P_0(x_0, y_0)$的极限存在,需要保证在$P_0$周围一个很小范围内的任何一个点,以任何一种轨迹,最终都能在曲面$z=f(x,y)$上(回到三维空间上,$(x,y,z)$)逼近到点$P'(x,y,z)$
\par
由此,一般如果要判定一个二元函数在某一点处无极限,只需要判断$(x,y)$在以某两个不同的轨迹趋近于$(x_0, y_0)$时得到的极限值不同
$$
\lim_{P\rightarrow P_0, (x,y) \in  l_1}{z = f(x,y)} \neq 
\lim_{P\rightarrow P_0, (x,y) \in  l_2}{z = f(x,y)}
$$
\paragraph{就像一元函数$y=f(x)$在某一点左右极限存在但不相等,则此一元函数在此点处极限不存在}
\par
{\textbf{\textcolor{red}{二元函数(或者多元函数),要证明其在某一点处极限不存在,只要证明其以不同路径趋近于此点时的极限不存在,为了便于计算,一般采用直线路径逼近:$y=kx$,k为变化的常数,如果结果中极限带$k$,说明按照不同$k$产生的直线轨迹$y=kx$会有不同的极限,那么二元函数的极限肯定不存在。反之,如果结果中不包含$k$,至少证明按照\textbf{直线轨迹}趋近时的极限存在相等,但是不能判断其连续,因为还可能有其他的曲线轨迹不行。}}}
\par
现实中,的确会出现一些不同轨迹极限不相同的情况:
这个你不需要深究,只要记住\textbf{如果二元函数在某一点的极限存在,那么此二元函数在此点周围的很小范围内,通过任何一条路径(所有路径)趋近于此点时,其极限存在且相同}
\par
\textcolor{red}{记忆的时候类比一元函数的情况就好了,二元函数(多元函数)只是比一元函数增加了一些自变量(维度)而已}
如判断:
\begin{enumerate}
    \item 一元函数$y=f(x)$在$x=0$处极限是否存在
    \item 二元函数$z=f(x,y)$在$x=0,y=0$处极限是否存在
\end{enumerate}
\begin{enumerate}
    \item 极限差别:
        \begin{enumerate}
            \item 一元函数:需要考虑左右极限是否存在且相等,即,$\lim_{x\rightarrow 0+}{f(x)}?=\lim_{x\rightarrow 0-}{f(x)}$,因为$x$趋近于$x_0=0$只有两个路径(方向):$x\rightarrow -0$,$x\rightarrow +0$
            \item 二元函数,因为$(x,y)$在一个二维平面,可以有不同的路径趋近于点$P_0(x_0,y_0)$,因此需要考虑所有的路径情况,如果有任何两个路径上极限不存在或者不相同,则此点处的二元函数极限不存在。
        \end{enumerate}
\end{enumerate}

\subsection{练习}
[P131例1]
证明下列重极限不存在
$$
\lim_{x\rightarrow 0, y\rightarrow 0}
{
    \frac{xy}{x^2 + y^2}
}
$$
\par
证明重极限不存在,只需要证明任意两个路径趋近时的极限不相等。 
\par
一般只需要任意取两个路径便可,但是更简便的方式就是用$y=kx$直线族($k$不同会出现不同的直线路径,直接将$k$作为一个变量,如果计算出来的极限里面包含$k$,则不同直线路径达到的极限会随着$k$不同而变化,则重极限不存在)
\par
解答里面的直线趋近,你应该就懂了




\subsection{多元函数的连续性}
\par 连续性方面会使用到前面多元函数极限的知识。 
\par 类比于一元函数的连续:$\lim_{x\rightarrow x_0}{f(x)}=f(x_0)$ 
\par 二元函数连续的定义是:$\lim_{x\rightarrow x_0, y\rightarrow y_0}{f(x,y)}=f(x_0, y_0)$
此处需要考虑$点P(x,y) \rightarrow P_0(x_0, y_0)$的路径.

而对于连续性的判断,两者的定义基本一致,只是判断的区域的差别:
\begin{enumerate}
    \item 一元函数是一维
    \item 二元函数是二维
    \item 多元函数是多维
\par 其他没有什么区别
\end{enumerate}
在定义上,如果
$$\lim_{x\rightarrow x_0\\ y\rightarrow y_0}{f(x,y) = f(x_0, y_0)}$$
则认为其在点$(x_0, y_0)$处连续


\subsection{多元函数偏导数}
\par
偏导数可以认为是只对某一维(变量)求导的特殊导数
\par
\textbf{\color{red}{底下你只需要明白偏导数的定义,和偏导数的求法就可以了}}
\paragraph{求法}
如果对$x$求偏导,则只关心$x$的变化。首先固定其他无关变量的值(当作为常数便可),原来的多元函数就变成了一个一元函数(非$x$的变量都当做了常数),然后对$x$求一下导数就可以了。
\paragraph{定义}
类似于一元函数的情况:
$$
f'(x_0) = \lim_{\Delta x \rightarrow 0}{\frac{f(x_0 + \Delta x) - f(x_0)}
{
    \Delta x
}}
$$
二元函数$z = f(x, y)$,当把$y$固定到$y_0$时,也就是一元函数的情况了
$$
f_{x}'(x_0,y_0) =\lim_{\Delta x\rightarrow 0}{ \frac{f(x_0 + \Delta x,y_0) - f(x_0, y_0)}
{
    \Delta x
}} 
$$
$$
= \lim_{x\rightarrow x_0}{
    \frac{
        f(x,y_0) - f(x_0, y_0)
    }
    {x - x_0}
}
$$

\par
如二元函数$z=f(x,y)$,我只想考虑此函数在点$(x_0, y_0)$处的$x$的变化,自然
要固定其他的变量$y=y_0$,
\par
于是原先的二元函数: $z=f(x,y)$降维为$z=f(x,y_0)$(一个一元函数),如此直接对$x$进行求导,就是$z=f(x,y)$关于$x$在$(x_9, y_0)$的偏导数。
\par
书上的例题:P14例1: 求$z=x^2 + 3xy + y^2$在点(1,2)处的偏导数。
需要分别对$x,y$求偏导。
\begin{enumerate}
\item 对$x$求导,则可以先将其他的变量固定,把$y=2$代入原式。有
    $ z = x^2 + 6x + 4 $
    就降为一个一元函数了,再对$x$求导,得到
    $$
    \frac{\partial z}
    {\partial x}
    = 2x + 6 = 2*1+6 = 8
    $$
\item 对$y$求偏导,首先固定无关变量x,降维,得到$z=1+3y+y^2$,然后求偏导
    $$
    \frac{\partial z}
    {\partial y}
    = 3 + 2y = 3 + 2*2 = 7
    $$
\end{enumerate}

%------ end content --------------
\end{multicols}
\end{document}
