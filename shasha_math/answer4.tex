\section{题目解答}
\subsection{P131,例1}
\par 
前面已经讲了,如果要重极限存在,如$\lim_{x\rightarrow x_0,\\ y\rightarrow y_0)}{f(x,y)\=z}$,因为其中需要$x,y$有一个逼近$x_0, y_0$的过程,反映到点上,设点$P(x,y)$,则$P$可以有一定的轨迹逼近到$P_0(x_0,y_0)$
\par 
如图:
\begin{center}
\begin{pspicture}(-4,-3)(4,4)
%\psgrid[subgriddiv=1,griddots=10,gridlabels=7pt](-3,-3)(3,3)
\psaxes[labels=none,ticks=none]{->}(0,0)(-3.5,-2.7)(3.5,4)
%--坐标轴
\psgrid[subgriddiv=1,griddots=10,gridlabels=7pt](-3,-3)(3,3)
\psline[linewidth=1pt,linearc=0]{->}(0, 0)(-2.5, -2.5)
\psellipse[linestyle=dashed,dash=3pt 2pt](0,0)(2, 1)
\psellipse[linestyle=dashed,dash=3pt 2pt](0,2.5)(2, 1)
%- 坐标轴标定
\rput[bl](0, 0){O}
\rput[bl](3, .2){\zihao{4}$y$}
\rput[br](-2.5, -2.5){\zihao{4}$x$}
\rput[bl](.2, 4){\zihao{4}$z$}
% 两个圈子的连线
\psline[linestyle=dashed,dash=3pt 2pt, linewidth=1pt,linearc=0]{-}(-2, 0)(-2, 2.5)
\psline[linestyle=dashed,dash=3pt 2pt, linewidth=1pt,linearc=0]{-}(2, 0)(2, 2.5)
\psline[linewidth=1pt,linearc=1.5](-2, 2.5)(0, 4)(2, 2.5)
% 图像标注
\rput[bl](1.5, 2){\zihao{3}$\alpha$}
\rput[bl](1.5, .2){\zihao{3}$\beta$}
% 加入点
\rput[bl](.5, .5){\zihao{3}$P_0'$}
\rput[bl](.5, 3){\zihao{3}$P_0$}
\psline[linestyle=dashed,dash=3pt 2pt, linewidth=1pt,linearc=0]{->}(.5, .5)(.5, 3)
% 路径趋向
% 曲面
\psline[linewidth=1pt,linearc=0,linecolor=red]{->}(0, 2.5)(.5, 3)
\psline[linewidth=1pt,linearc=.5,linecolor=red]{->}(-.5, 3.5)(0, 4)(.5, 3)
\psline[linewidth=1pt,linearc=.5,linecolor=red]{->}(1, 2.5)(1, 3.2)(.5, 3)
\psdots(.5, 3)
% x y面
\psline[linestyle=dashed,dash=3pt 2pt, linewidth=1pt,linearc=0,linecolor=red]{->}(0, .5)(.5, .5)
\psline[linewidth=1pt,linearc=0,linecolor=red]{->}(0, -.5)(.5, .5)
\psline[linewidth=1pt,linearc=.5,linecolor=red]{->}(-.5, .5)(0, 1)(.5, .5)
\psline[linewidth=1pt,linearc=.5,linecolor=red]{->}(1, -.5)(1, .2)(.5, .5)
\psdots(.5, .5)
% 路径标注
\rput[bl](0.2, -.7){\zihao{4}$l_1$}
\rput[bl](-.5, .5){\zihao{4}$l_2$}
\rput[bl](1.2, -.5){\zihao{4}$l_3$}
\end{pspicture}
\end{center}
\par
图中,假设$\alpha$表示二元函数$y=f(x,y)$表示的图形,像一个曲面圆锥的上表面,其投影到$xoy$平面,就是一个平面圆形$\beta$.
\par
$P_0$是曲面上一个定点,投影到$xoy$平面上就是$P_0'$点。
由于我们这里只讨论$x,y$的值变化,因此直接考虑二维平面$xoy$平面上的情况便可,也就是$\beta$面,
由于$z$由$x,y$决定--$x,y$为自变量,$z$为因变量。 


\subsection{全微分}
\par 书上有一个定义:
如果 $\Delta z$可以表示为:
\begin{equation}
\Delta z = f(x + \Delta x, y + \Delta y) - f(x, y) = A \Delta x + B \Delta y + o(\rho)
\end{equation}
其中 
$$
\rho = \sqrt{(\Delta x)^2 + (\Delta y)^2}
$$
\par 其中$A,B$与$x,y$无关
\par
参照P130定理1,如果函数$z=f(x,y)$在点$(x,y)$处可微,则该函数在点$(x,y)$处的偏导数
$\frac{\partial z}{\partial x}, \frac{\partial z}{\partial y}$必定存在,且
$dz =\frac{\partial z}{\partial x}dx + \frac{\partial z}{\partial y}dy$ 
也就是
\begin{equation}

\end{equation}





当要利用定义证明一个多元函数可微











