\documentclass[a4paper]{ctexart}
\usepackage{geometry}
\usepackage{multicol}
\usepackage{pstricks}
\usepackage{pst-plot}
\usepackage{color}
\usepackage{amsmath}
\geometry{left=2cm,right=2cm,top=2.5cm,bottom=2.5cm}


\setCJKmainfont[BoldFont={SimHei},ItalicFont={KaiTi}]
  {FangSong}

\author{Chunwei Yan}
\title{全书解答4}
\begin{document}
    \maketitle
%---content here----
\begin{multicols}{2}
%---content here----
\section{前言}
\par
现在开始做真题,同时,可以对着考纲,把每个知识点落实一下。 好好积累,多记忆,抓重点题型.
\section{充分条件与必要条件的区别}
这里说一下
如$a \rightarrow b$
则,我们说
\begin{enumerate}
    \item a是b的充分条件(a可以推出b) 
    \begin{enumerate}
        \item a成立,则b必然成立
    \end{enumerate}
    \item b是a的必要条件(a可以推出b)
    \begin{enumerate}
        \item b是a的前提条件
        \item b成立,a不一定成立;但b不成立,则a肯定不成立
    \end{enumerate}
\end{enumerate}
\subsection{理解}
\begin{enumerate}
    \item 可以把充分条件理解为,足够充分的条件,如a是b的充分条件,表明a是b成立足够充分的条件(a是b成立所需要的所有条件),因此,a成立,则b必然成立
    \item 可以把必要条件理解为,必须要满足的条件,如b是a的必要条件,表明如果a成立,b必须要成立,则b是a的前提条件(但不一定是足够充分的条件,可能还需要c成立),但是如果b不成立,那么a肯定不成立
\end{enumerate}


\section{可微}
\par 书上有一个定义:
如果 $\Delta z$可以表示为:
\begin{equation}
\Delta z = f(x + \Delta x, y + \Delta y) - f(x, y) = A \Delta x + B \Delta y + o(\rho)
\end{equation}
其中 
$$
\rho = \sqrt{(\Delta x)^2 + (\Delta y)^2}
$$
\par 其中$A,B$与$x,y$无关,则认为$z=f(x,y)$可微
\par
证明可微方面,参照\textbf{\textcolor{red}{P134\quad 三、讨论二元函数的可微性}}

\subsection{用定义证明可微}
\par 
当无法用偏导数来证明可微时,可以用定义来证明。
\par
由上面的定义来看,可微的充分条件是
$$
\lim_{x\rightarrow x_0, y\rightarrow y_0}
$$
\begin{equation}
{
    \frac{
        f(x, y) - f(x_0, y_0) - f'_x(x_0,y_0)\Delta x - f'_y(x_0, y_0)\Delta y
    }
    {
        \rho
    }
}
\end{equation}
$$
=0 = \frac{o(\rho)}{\rho}
$$
\par
其中$\rho = \sqrt{x^2+y^2}$,$\Delta x= x - x_0$,$\Delta y = y-y_0$,$o(\rho)$为$\rho$的无穷小
\par
只要满足这个条件,就认为可微
\par 
下面会有用到这个结论的题目,你看看考纲,是不是有用定义证明可微的要求,如果有就看,一切按照考纲上的要求.

\subsection{可微的充分条件}
\par
$z=f(x,y)$在点$(x,y)$可微的必要条件是其在点$(x,y)$处的偏导数$\frac{\partial z}{\partial x}$和$\frac{\partial z}{\partial x}$存在且连续.
\par
因此,一般证明可微,可以用偏导数存在且连续的方法证明.
\textbf{注意,偏导存在且连续是可微的充分条件而非必要条件,也就是说,可微并不能够推得偏导连续},前面有得到偏导存在时可微的必要条件
\par
总结一下,就是可微可以得到偏导存在,但并不能推得偏导连续.


\subsection{可微的必要条件}
参照P130定理1,如果函数$z=f(x,y)$在点$(x,y)$处可微,则该函数在点$(x,y)$处的偏导数
$\frac{\partial z}{\partial x}, \frac{\partial z}{\partial y}$必定存在,且
$dz =\frac{\partial z}{\partial x}dx + \frac{\partial z}{\partial y}dy$ 



\section{P131,例1}
%--- 插入图

%--- 开始解释一般的做法

\par
\textbf{下面,提到的二元函数的情况同时也可以推广到其他多元函数的情况}
\par
在上一个文档里面讲过,二元函数$z = f(x,y)$,从自变量角度,$x,y$在一个二维平面的一定区域(定义域)上,从一个点到另外一个点有多种路径。
%-- 图样
\subsection{只考虑自变量}
\par
从自变量角度,直接看定义域内,点$P(x,y)$趋近于点$O$的路径
\begin{center}
\begin{pspicture}(-4,-3)(4,4)
%\psgrid[subgriddiv=1,griddots=10,gridlabels=7pt](-3,-3)(3,3)
\psaxes[labels=none,ticks=none]{->}(0,0)(-3.5,-2.7)(3.5,4)
%--坐标轴
%\psgrid[subgriddiv=1,griddots=10,gridlabels=7pt](-3,-3)(3,3)
\pscircle(0,0){2.5}
%- 坐标轴标定
\rput[bl](0, 0){O}
\rput[bl](3, .2){\zihao{4}$x$}
\rput[bl](.2, 3){\zihao{4}$y$}
\psline[linestyle=dashed,dash=3pt 2pt,linewidth=1pt,linearc=1,linecolor=red]{<-}(0,0)(.5,.2)(1,1.5)(1.5,2)
\psline[linestyle=dashed,dash=3pt 2pt,linewidth=1pt,linearc=1,linecolor=red]{<-}(0,0)(.5,1)(1,1.5)(1.5,2)
\psline[linewidth=1pt,linearc=1,linecolor=blue]{<-}(0,0)(1.5,2)
\psline[linewidth=1pt,linearc=1,linecolor=blue]{<-}(0,0)(-1.3,1)
\psdots(-.5, .5)
\rput[bl](-.5, .5){$P_0'$}
\end{pspicture}
\end{center}
\par
在前面说过,如果要判断二元函数的极限存在,需要从各个方向的各种路径到固定点的极限都存在且相等,要证明极限不存在,只需要证明按照任意两个路径的极限不存在便可。一般最简单也最常用的是采用两个不同的直线轨迹趋近,就是类似图中的实线路径。
\par
采用$y=kx$的直线作为趋近轨迹,代入其中
$$
\lim_{x\rightarrow 0, y\rightarrow 0}{\frac{xy}{x^2+y^2}}
=
\lim_{x\rightarrow 0, y=kx\rightarrow 0}{\frac{kx^2}{x^2+k^2x^2}}
=\frac{x}{1+k^2}
$$
\par
因为结果里面包含了$k$,极限会因为$k$取不同的值而不同,也就是按照不同的轨迹会有不同的极限,说明极限不存在.
\par
比如,$k$分别取1,2. 如此,趋近的轨迹分别是$y=x$和$y=2x$,会得到
$$
\lim_{y=kx,x\rightarrow 0}{\frac{xy}{x^2+y^2}}
=
\begin{cases}
    \begin{array}{ll}
    \frac{1}{1+1}   &   k=1\\
    \frac{2}{1+4}   &   k=2
    \end{array}
\end{cases}
$$
\par
\textbf{这个是证明二元函数极限不存在的最常用的方法,好好掌握}


\section{P132,例2}
主要还是采用的之前等价无穷小那部分的知识点,那部分很重要

\subsection{知识点}
\par
还是参照\textbf{全书P11的3.几个重要极限与几个重要的等价无穷小}
\par
\textcolor{red}{做真题的时候,注意参照数学的考纲,把每个知识点都过一下,若觉得有不懂的点,好好看看,把规定要考的每个点都熟悉的话,就很好了}
$$
x\rightarrow 0, \sin{x} \sim x
$$
同样的
$$
x\rightarrow \infty, \sin{\frac{1}{x}} \sim \frac{1}{x}
$$
\par
\textbf{把等价无穷小部分再看看,那部分很重要,大部分题目都会碰到}

\subsection{解答}
参照\textbf{全书P11 \quad 3.几个重要的极限与几个重要的等价无穷小}
$$
\lim_{x\rightarrow 0}{(1+x)^\frac{1}{x}} = e
$$
一般用的时候,这边的$x$可以换成一个函数$\varphi(t)$,只要满足
$$
\lim_{\varphi(t)\rightarrow 0}{(1+\varphi(t))^\frac{1}{\varphi(t)}} = e
$$
\par
\textbf{这个式子是考纲里面专门列出来的两个式子之一,所以肯定会考的,多把那些基础的熟悉熟悉,可以把那些变形的形式也整理一下,多看看就可以了}
\par
最后结果的来源:
$$
\lim_{x\rightarrow \infty,y\rightarrow 1}
{
    (1+ \frac{1}{xy})^
    {\frac{y^2}{\sin{\frac{2}{x}}}}
}
=
(1+\frac{1}{x}) ^ {\frac{1}{\frac{2}{x}}}
$$
$$
=e^{\frac{1}{2}}
$$

\subsection{几种变形形式}
公式 $ \lim_{x\rightarrow 0}{(1+x)^\frac{1}{x}} = e $
有几种变形形式:
\begin{enumerate}
\item
$$
\lim_{x\rightarrow 0}{(1-x)^\frac{1}{x}} = 
\lim_{-x\rightarrow 0}{(1+(-x))^\frac{1}{x}}  
$$
$$
=\lim_{-x\rightarrow 0}
    {
    \frac{1} {
        (1+(-x))^
        {
            \frac{1}{x}}}  
    }
= \frac{1}{e} 
$$

\item 自然, $x\rightarrow 0$和$x\rightarrow \infty,\frac{1}{x}\rightarrow 0$等价,
$$
\lim_{x\rightarrow 0}{(1+x)^\frac{1}{x}} = 
\lim_{x\rightarrow 、infty}{(1+\frac{1}{x})^\frac{1}{\frac{1}{x}}} = 
e
$$
\end{enumerate}

\subsection{总结}
如此,等价去穷小的式子里面,真正考的时候,都是用函数替换的方式,自己每次做到这部分题目都可以把前面的公式过一下,有必要可以把这个题目的形式摘录下来,每天看看就熟悉,下次就第一时间能够反映过来了。

\section{P134,例6}
\subsection{知识点}
\par 
主要是微分与偏导及连续的关系
\par
根据\textbf{P130\quad6.多元函数连续、可导、可微的关系}
可以得到,对于二元函数
\begin{enumerate}
    \item 可微$\rightarrow$连续、可导
        \begin{enumerate}
            \item 连续是可微的必要条件(前提条件)
            \item 可导是可微的必要条件(前提条件)
        \end{enumerate}
    \item \textbf{连续与可导并没有直接关系}(这个与一元函数不同,一元函数可导可以推出连续)
\end{enumerate}
\par
再回顾一下,可微的的充分条件有两种,也就是证明可微的方法:

\begin{enumerate}
    \item 定义法,参照上面\textbf{可微}部分的内容
    \item 用偏导的方法进行证明,每个偏导均存在及连续$\Rightarrow$可微
\end{enumerate}


\section{P134,例7}
\par
题目主要考察可微与偏导的关系。 
\par
最后两行就是用定义判断函数在$(0,0)$是否可微。
$$
\frac{o(\rho)}{\rho}
=
$$
$$
\lim_{\Delta x\rightarrow 0, \Delta y\rightarrow 0}
{
    \frac{
        [f(\Delta x,\Delta y) - f(0,0)] - (f'_x(0,0)\Delta x + f'_y(0,0)\Delta y] 
    }
    {\rho}
}
$$
其中
$$
\rho = \sqrt{
    {\Delta x}^2
    +
    {\Delta y}^2
}
$$
在证明
$$
\lim_{\Delta x\rightarrow 0, \Delta y\rightarrow 0}
{
    \frac{\Delta y(\Delta x)^2}
    {
        [(\Delta x)^2 + (\Delta y)^2]^{\frac{3}{2}}
    }
}
$$
的极限值不存在时,同样采用了二元函数的极限存在性的判断。
上面讲到了,如何判断二元函数的极限是否存在,可以先用一个直线线段轨迹$y=kx$试一下,如果极限里面带$k$,说明,根据不同的直线轨迹,极限不同,因此此二元函数的极限不存在。
%------ end content --------------
\end{multicols}
\end{document}


