\documentclass[a4paper]{ctexart}
\usepackage{geometry}
\usepackage{multicol}
\usepackage{pstricks}
\usepackage{graphics,graphicx}
\usepackage{pst-plot}
\usepackage{color}
\usepackage{amsmath}
\geometry{left=2cm,right=2cm,top=2.5cm,bottom=2.5cm}

\setCJKmainfont[BoldFont={SimHei},ItalicFont={KaiTi}]
  {FangSong}
\title{全书解答8}
\author{Chunwei Yan}
\date{\today}
\begin{document}
\maketitle
%---content here----
\begin{multicols}{2}
\section{P87例8,方法2}
\par 首先,参照例6的结论,这个结论比较重要:

\subsection{例6关于$f(x)$及其原函数奇偶性和周期性的关系}
\par 首先明确一点: 
\begin{itemize}
    \item $f(x)$的原函数是$\int_0^x{f(x)} + C$,其中$C$是常数项
    \item $\int_0^x{f(x)}$只是$f(x)$的一个原函数(根据$C$的不同,还有很多其他的原函数)
\end{itemize}
\par 那么会有三种情况

\subsubsection{ 已知$F(x)$的情况,看$f(x)$的情况}
\begin{enumerate}
\item 如果$F(x)$是奇函数,则$f(x)$是偶函数
\item 如果$F(x)$是偶函数,则$f(x)$是奇函数
\item 如果$F(x)$是T周期函数,那么$f(x)$也是T周期函数
\end{enumerate}

\subsubsection{ 已知$f(x)$的情况,看$F(x)$的情况}
\par 反之,由于$f(x)$的反函数$F(x) = \int_0^x{f(x)} + C$受常数$C$的影响,情况会复杂一点
\begin{enumerate}
\item $f(x)$为偶函数,则其只有一个特殊的原函数$\int_0^x{f(x)}$为奇函数,而其他原函数$\int_0^x{f(x)}+C$则会由于常数项$C$的变化而变化
    \begin{itemize}
    \item 主要因为$F(x)+f(-x) = \{\int_0^x{f(x)} + C \} + \{\int_0^{-x}{f(x)} + C\} = 2C$
    \item 这个值只在$C=0$时,才有$F(x)+F(-x)=0$
    \end{itemize}
\item $f(x)$为奇函数,则其所有原函数$\int_0^x{f(x)}+C$全为偶函数
    \begin{itemize}
    \item $F(x) - F(-x) = \{ \int_{0}^{x}{f(x)} + C \} - \{ \int_{0}^{-x}{f(x)} + C \} = 0$
    \end{itemize}
\item 若f(x)为T周期函数,那么必须有$\int_0^T{f(x)dx}=0$,才能够有$F(x)$为周期函数
\end{enumerate}

\subsection{间断点}
\textcolor{red}{\textbf{如果不熟悉第一类间断点、跳跃间断点、第二类间断点,回顾一下P31$\xi 3$}}

\subsection{解答}
\par 采用了论证法,无非就是按照题目中的条件,推导出来结论
\par 跳跃间断点就是左右极限存在但不相等
\par 如果在$x=a_0$处存在跳跃间断点,也就是
\begin{equation}
\lim_{x\rightarrow {a_0}^+}{f(x)} = A
\end{equation}
\begin{equation}
\lim_{x\rightarrow {a_0}^-}{f(x)} = B
\end{equation}
\begin{equation}
A \neq B
\end{equation}
\par 调到方法2第3个公式
\begin{equation}
\varphi(x) = 
\begin{cases}
    \begin{array}{ll}
    f(x) - A    &   x>0;\\
    0           &   x=0;\\
    f(x) + A    &   x<0;
    \end{array}
\end{cases}
\end{equation}
\par 接下来,当$x>0$时
\begin{equation}
\int_0^x{\varphi(x)dx} = \int_0^x{(f(t) + A)dt} = \int_0^x {f(t)dt} + Ax
\end{equation}
\par 当$x<0$时类似
\par 如此,继续向下看,到
\begin{equation}
\int_0^x{f(t)dt} - A \left| x \right|
\end{equation}
判断其为偶函数,其中$\int_0^x{f(t)dt}$为偶函数就是根据上面提到的例6的结论

\section{P92例2最后一行解答}
\par 这里用到复杂分式的分解
\par 当出现类似
$$ \int{
    \frac{dx}{
        x(x-1)^2
    }
}
$$
\par 就可以将其中的分式分解为多个简单分式的和,然后拆分开来积分

\subsection{复杂分式的分解}
\par 首先进行化简,如
\begin{equation}
\frac{
    ax+b
}
{(x-1)^2x}
=
\frac{a}{(x-1)^2}
+
\frac{b}{(x-1)^2x}
\end{equation}
\subsubsection{分子为常数项的简单分式的分解}
\par 下面讨论类似于$\frac{1}{x(x-1)^3}$的式子的拆分
\par 先将分母拆分为所有可能的小的部分
\par 这里,分母可以拆分为

\begin{equation}
\begin{array}{l}
x\\
(x+1)\\
(x+1)^2\\
(x+1)^3\\
\end{array}
\end{equation}
然后设上参数
\begin{equation}
\frac{A}{x}+
\frac{B}{(x+1)}+
\frac{C}{(x+1)^2}+
\frac{D}{(x+1)^3}
=
\frac{1}
{x(x+1)^3}
\end{equation}
\par 求得
\begin{equation}
\begin{cases}
\begin{array}{l}
A = 1\\
D = 1 
\end{array}
\end{cases}
\end{equation}
\par 于是,就可以得出
\begin{equation}
\frac{1}
{x(x+1)^3}
=
\frac{1}{x} \times
\frac{1}{(x+1)^3}
\end{equation}
\subsubsection{分子为多项式的分式的分解}
\par 比如需要分解
\begin{equation}
\frac{x^2+x+2}
{x(x+1)^3}
\end{equation}
\par 首先化简
\begin{equation}
\frac{x^2+x+2}
{x(x+1)^3}
=
\frac{x}
{(x+1)^3}
+
\frac{3}
{x(x+1)^3}
\end{equation}
\par 其中$\frac{x}{(x+1)^3}$的分解
\par 对分母的分解方法不变,但是分子$x$的分解,需要将分子中的多项式还原为n项的参数多项式
\par 比如分子为n次多项式$g(n,x)$需要转化为$ = Ax^n + Bx^{(n-1)} + Cx^{(n-2)} + \cdots + D$
\par 这里分子比较简单,就是一次多项式$x$,需要转化为1次参数多项式$Ax + B$
\par 如果这里分子比较复杂,为$x^2+2$,二次多项式,那么也是从二次一直到0次(常数),为$Ax^2+Bx+C$,其中的参数A,B,C将会被计算出来
\par 回来原来的题目
\par 拆分分子分母,并且设方程
\begin{equation}
\frac{x}
{(x+1)^3}
=
\frac{Ax+C}{(x+1)}
+
\frac{Ax+C}{(x+1)^2}
+
\frac{Ax+C}{(x+1)^3}
\end{equation}













\section{P94例7倒数第三行}

\section{P96例11解答1、2最后一行}

\section{P98例15(1)评注}

\section{P100例18P101上数第3行}

\section{P105例5解答倒数7、8行}

\section{P110例1(2)}

\section{P112第1行}

\section{P120第1行}

\section{P120例7方法1}

\section{P121例8方法1}

\section{P123例10}
\end{multicols}
\end{document}
