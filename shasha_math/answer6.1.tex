\documentclass[a4paper]{ctexart}
\usepackage{geometry}
\usepackage{multicol}
\usepackage{pstricks}
\usepackage{graphics,graphicx}
\usepackage{pst-plot}
\usepackage{color}
\usepackage{amsmath}
\geometry{left=2cm,right=2cm,top=2.5cm,bottom=2.5cm}

\setCJKmainfont[BoldFont={SimHei},ItalicFont={KaiTi}]
  {FangSong}

\author{Chunwei Yan}
\title{全书解答6.1}
\begin{document}
    \maketitle
%---content here----
\begin{multicols}{2}
\section{P163例49}
\subsection{知识点}
\par 此题也是基于对称性的二元函数积分。
判断其关于$y=x$对称,需要两个条件:
\begin{enumerate}
    \item 积分域关于$y=x$对称
    \item 积分函数中,x和y对换,形式不变
\end{enumerate}

\subsection{解答}
\subsubsection{极坐标}
\par 我之前有写过一个极坐标的文档,可以对照那个文档看看。
\par 这里就简单回顾,如图,是一个极坐标,$A(\theta,\rho)$
\par 同时可以看到,如果在直角坐标下,$x=\rho \cos{\theta}$,$y=\rho \sin{\theta}$,如此,A的直角坐标为$A(\rho \cos{\theta}, \rho \sin{\theta})$

\begin{center}
\begin{pspicture}(-1,-1)(4,4)
%\psgrid[subgriddiv=1,griddots=10,gridlabels=7pt](-3,-3)(3,3)
%--坐标轴
\psline[linewidth=1pt,linearc=0]{->}(0, 0)(3.5, 0)
\psline[linestyle=dashed,dash=3pt 2pt,linewidth=1pt,linearc=0]{->}(0, 0)(0, 3.5)

%-- 弧形
\pswedge[linecolor=gray,linewidth=1pt,fillstyle=solid]{.5}{0}{45}

\psline[linewidth=1pt,linearc=0]{->}(0, 0)(2, 2)
\psline[linestyle=dashed,dash=3pt 2pt, linewidth=1pt,linearc=0]{-}(2, 0)(2, 2)
%--标注
\rput[br](0, 0){O}
\rput[bl](.5, .2){$\theta$}
\rput[bl](2, 2){A}
\rput[tl](1, -.2){$\rho \cos{\theta}$}
\rput[bl](2.2, 1){$\rho \sin{\theta}$}
\end{pspicture}
\end{center}

\subsubsection{方法一、用极坐标解}
\par 极坐标下的二元函数积分的具体内容参照P1160页
\par 固定公式需要记住,分为极点O在区域D内部、边界和外部3种情况。
\par 这里O是区域的圆心,利用内部的那种情况
\begin{equation}
\iint_{D}{f(x,y)d\sigma} = \int_{\alpha}^{\beta}\int_{\rho_1(\theta)}^{\rho_2(\theta)}{f(\rho\cos{\theta},\rho\sin{\theta})\rho d\rho }
\end{equation}
\par 其中,将 $x=\rho \cos{\theta}$,$y=\rho \sin{\theta}$代入,可以直接得到
\begin{equation}
\int_{0}^{2\pi}\int_{0}^R{
    (
        \frac{\cos^2{\theta}}{a^2}
        +
        \frac{\sin^2{\theta}}{b^2}
    )
    \rho^3d\rho
}
\end{equation}
\par 后面对$\rho$求积分,前面$\theta$不需要动
\begin{equation}
= \frac{R^4}{4}
\int_0^{2\pi}
{
    \frac{\cos^2{\theta}}{a^2}
    +
    \frac{\sin^2{\theta}}{b^2}
    d\theta
}
\end{equation}
\par 再后面把$\cos^2{\theta}$和$\sin^2{theta}$分开积分,这里只列出$\cos^2{\theta}$的积分
\begin{equation}
\int_0^{2\pi}
{
    \frac{cos^2{\theta}}{a^2} d \theta
}
=
\frac{1}{a^2}
\int_0^{2\pi}
{
    \frac{\cos{2\theta} + 1}{2} d\theta
}
\end{equation}
剩下的就是普通的积分了,算出来就是最后的结果了。
\textbf{\textcolor{red}{需要多看看极坐标求积分适用的范围}}

\subsubsection{方法二、用对称的方法求解}
\par 直接目测,利用上面介绍的方法也能够判定其对称
\par 于是, $x,y$可以互换
\par 底下就按照解答

\section{P164例52}
\par 这个是先x后y的积分
\par 具体
\begin{equation}
\int_{0}^{1}{dy} \int_{y^2}^{y}
{
    \frac{\sin{y}}{y}dx
}
\end{equation}
\par 代表的是,从前往后看,y从0到1,x从$y^2$到$y$
\par 为什么是x从$y^2$到$y$,而不是从$y$到$y^2$
\par 可以再图像中央画一条横线,可以看到其与弧形阴影有两个交点,按照x轴的方向从左往右看,就是$x=y^2$到$x=y$
\par 在积分的时候,注意前后的顺序,先后面对$x$积分,所以即使里面有$y$,也没有影响,把$y$当做常量,可以提到前面去。
\begin{equation}
=
\int_{0}^{1}{\frac{\sin{y}}{y}dy} \int_{y^2}^{y}
{
    dx
}
=
\int_{0}^{1}{(y-t^2)\frac{\sin{y}}{y}dy}
\end{equation}
\begin{equation}
= \int_{0}^{1}{
        \{
            \sin{y} - y\sin{y}
        \}dy
    }
\end{equation}
\par 分开进行计算
\par $\int_{0}^{1}{\sin{y}dy}$很容易计算,而
$
\int_{0}^{1}{\sin{y}dy}
$
\par 是一个标准的分部积分的例子,\textbf{\textcolor{red}{需要回顾之前的分部积分的内容,全书P89,做几道典型题找找感觉}}
\par 我这里直接用不定积分求出其原函数,然后可以代入值计算
\begin{equation}
\int{y\sin{y}dy}
=
\int{yd\cos{y}}
= y\cos{y} - \int{\cos{y} dy}
\end{equation}
\par 已经一目了然了,然后代入值就能解出了
\par \textbf{\textcolor{red}{不定积分和定积分的求解方法是基础中的基础,一定要很熟练,如果有点生疏的话,得回头好好看看。 这个知识点会穿插到大部分题目里面,很关键}}

\section{P165例54}
\par 方法三采用的是移动圆形的方法,可以看到图中的圆心x轴方向偏离$\frac{1}{2}$,y轴方向偏离$\frac{1}{2}$。
\par 通过$x-\frac{1}{2}$和$y-\frac{1}{2}$将圆心重新调到零点处。当调到零点后,就可以采用\textbf{对称}拆分的方法,极大地减少计算量
\par 由于积分区域D此时关于x和y均对称,而积分函数$f(x,y) = x+y$,在x轴右边$x=a$为$f(a,y)=a+y$,在左边对称的地方$x=-a$为$f(-a,y) = -a + y$,左右加起来为0, 同样,y轴上下加起来也为0,如此结果为0

\subsection{总结}
\par 看看例55,可以看到它也可以用移动的方法求解,整体将其向下移动1个单位($y-1$),可以使其关于x轴对称,然后就可以用对称的方法了。
\par 好好品味这里的典型题,利用一些技巧,降低计算量
\par 而且,这些技巧是真题里面最提倡的,好好做做,好好总结

\section{P166例55}
\subsection{方法一第7行的式子来源}
\par 前面提到平移的方法,$y-1$就是把图形整体向下平移1个单位,然后那个圆弧正好以原点为圆心,利用极坐标,就得到那个式子了。
\subsection{方法二第一行最后一个式子}
\par 还是极坐标那边的知识,好好看看我附件里面的那个PPT

\begin{center}
\begin{pspicture}(-4,-2)(4,4)
\psaxes[labels=none,ticks=none]{->}(0,0)(-3.5,-2.7)(3.5,4)
%--坐标轴
\pswedge[linecolor=black,linewidth=2pt,fillstyle=solid](0,1.5){1.5}{90}{270}
\psline[linestyle=dashed,dash=3pt 2pt, linewidth=1pt,linearc=0]{-}(0, 0)(-1.5, 2)
\psline[linestyle=dashed,dash=3pt 2pt, linewidth=1pt,linearc=0]{-}(0, 0)(-.7, 2.7)

\pswedge[linecolor=red,linewidth=1pt,fillstyle=none](0,0){.5}{0}{100}
\pswedge[linecolor=blue,linewidth=1pt,fillstyle=none](0,0){.8}{0}{125}

\rput[bl](0, 0){O}
\rput[bl](0, 2){A}
\rput[bl](-2, 0){B}
\end{pspicture}
\end{center}
\par 如图,极坐标的角度只从原点开始,从x轴正方向开始算
\par 因此,角度范围为从y轴正轴到x轴的负轴,算成角度就是$\frac{\pi}{2}$到$\pi$(切线)






%------ end content --------------
\end{multicols}
\end{document}
