\documentclass[a4paper]{ctexart}
\usepackage{geometry}
\usepackage{multicol}
\usepackage{pstricks}
\usepackage{pst-plot}
\usepackage{color}
\usepackage{amsmath}
\geometry{left=2cm,right=2cm,top=2.5cm,bottom=2.5cm}


\setCJKmainfont[BoldFont={SimHei},ItalicFont={KaiTi}]
  {FangSong}

\author{Chunwei Yan}
\title{最后一阶段的建议}
\begin{document}
    \maketitle
%---content here----
\section{各个科目的安排建议}

\subsection{数学}
\begin{enumerate}
    \item 一天做真题,一天复习消化
    \item 在消化题目的时候,需要注意多衍生,尽量把每一道题里面的每一个知识点都落实一下。
    \item 如果有发现漏洞,回顾全书里面的知识点。 好好做几道典型题
    \item 参照考纲,把每一个知识点落实一下。
    \item 有不熟练的知识点或者题型,标记一下,每天都拿出来练,比如一些公式,一些基础的题型,练上十次八次的会很熟练的!这个比单纯多做几道题效果要好很多。
    \par
    \textbf{\textcolor{red}{正好是最后一个月,好好复习回顾一下,也不会忘,熟练了直接用到考场上了。}}
    \par 所以不要担心之前的,好好把握最后一阶段的复习就可以了。
\end{enumerate}

\subsection{英语}
\begin{enumerate}
    \item 长难句,每天练习,把所有真题的阅读的长难句都仔细看,仔细过。 这个就是英语复习最大的技巧,最终肯定会受益匪浅的。
    \item 生词,过一过,特别注意特殊形式。至少真题以及\textbf{问题及选项}里出现过的生词一定要很熟练,常考的词专门练习。
    \item 定期做阅读真题,至少每天可以做一道,把方法和感觉保留住。
    \item 作文,有一本书就可以了。抽出几个比较好的作文(文笔和结构都比较合适)作为自己文体的积累好好读读背背,其他的作文(特别是真题里的),可以看看,划出好的句子词组,作积累。
\end{enumerate}

\subsection{政治}
政治你应该挺好的,这个按照你自己的节奏吧。
\begin{enumerate}
    \item 提前熟悉一下内容,然后做选择题,对照答案,把正确选项划出来记忆
    \item 《风中劲草》有关的内容多读多背
    \item 时事政治应该市面上的书能够涉及到
    \item 最后看看市面上有没有上一年的那种小册子,看看
    \item 不需要弄太多书,精选,然后踏踏实实记就可以了,战线拉的太广没有意义
\end{enumerate}

\subsection{专业课}
这个完全按照你自己的节奏就好了,不要忽略或者落下了,专业课在复试阶段是比较重要的。


\section{注意点}
\par
\subsection{数学}
\par 这个阶段真题是最重要的。只要最终能够达到这样的目标-- -- 真题里面每一道题都理解透彻了,那么最后的考试肯定在基础分之上了。
\par 没有必要刻意去做太多题。 真题,加一定量的典型题就够了,做题的时候要带着一种测试和总结的感觉做。这个阶段做题就是为了对前一阶段的知识内容作一个总结和回顾。
\par 有不会做的题,没有必要紧张,只需要做两件事就算达到目标了
\begin{enumerate}
    \item 把这道题好好弄懂,理解
    \item 把里面涉及到的所有知识点都去回顾一下,确认知识点没有漏洞了。 如果有漏洞了,回顾全书,生疏的知识点标出来,每天练习到熟练为止。如果题型生疏的话,另外把全书上的典型题型回顾一下。
\end{enumerate}

\subsection{英语}
\par 英语,每天都得练习,一直把感觉和水平坚持到最后。
\par 方法要加强,当拿到题目,用自己的一套方法(套路)能够做出来的话,那就有谱了。

\section{生活及目标}
\par 好好的,别太累了。毕竟最后一个阶段了,把身体维持好了上考场。 不要都疲劳都积累到最后了。
\par 中午和晚上吃完饭了,或者有烦劳了,可以出去散散步,散散心。发泄完了再回去复习,劳逸结合了。路上尽量走路吧,就当锻炼了,把身体养好,根据天气和地点增减衣服,特别注意别着凉了。下雾的时候一定要把口罩戴上,北京的雾越来越厉害了。 
\par 最后阶段了,没有必要紧张,毕竟付出了那么多。 好好的按照自己的计划坚持到最后,好好珍惜中间的过程,不要太顾虑结果。
\par 有句话叫做付出终有回报,不管是什么样子的回报,自己踏实地付出过,最终的收获肯定是丰硕的。
\par 好好的,踏踏实实的。带点虔诚,很多事情本来就是天经地义的。
\par 紧张的时候,冷静下来。 你不是一个人在战斗,我和你的家人会陪着你,一起去坚持,一起去面对。
\par 充满希望,自信,以自己最好的状态去迎接风浪吧! 
\end{document}


