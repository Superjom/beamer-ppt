\documentclass[a4paper]{ctexart}
\usepackage{geometry}
\usepackage{multicol}
\usepackage{pstricks}
\usepackage{pst-plot}
\usepackage{amsmath}
\geometry{left=2cm,right=2cm,top=2.5cm,bottom=2.5cm}

\author{Chunwei Yan}
\title{全书解答2}
\begin{document}
    \maketitle
%---content here----
\begin{multicols}{2}

\section{P59-例7}

% ---------- 椭圆图形 -----------------------------------------
\begin{pspicture}(-4,-3)(4,3)
%\psgrid[subgriddiv=1,griddots=10,gridlabels=7pt](-3,-3)(3,3)
\psaxes{->}(0,0)(-3.5,-2.7)(3.5,2.7)
\psellipse(0,0)(2.5,1.5)
%切线  
\psline[linewidth=1pt,linearc=0]{-}(4.3, -1)(-1, 3.3)
\psellipse[linestyle=dashed,dash=3pt 2pt](0,0)(.5, 2.5)
\psline[linestyle=dashed, dash=3pt 2pt,linewidth=1pt,linearc=0]{-}(0, -2.5)(3, 0)
%曲线模式
\psline[linestyle=dashed, dash=3pt 2pt,linewidth=1pt,linearc=.5]{-}(0, 2.7)(1, 2)(2, 2)(3, 0)

%字母
\rput[bl](0, 0){O}
\rput[b](3, 0){P}
\rput[bl](0, 3){A}
\rput[bl](0, -3){B}
\rput[bl](1, 0){a}
\rput[bl](0, 1){b}
\rput[bl](2, 2){$l_1$}
\end{pspicture}

\subsection{计算截距}
由图形得到,由于旋转的是一个三角形,所以需要求的体积是一个圆锥,可以直接用圆锥的计算公式计算。当然,如果旋转的是一个曲线$l_1$,则形成的就是一个曲面椎体,则必须用积分的方式求解体积。\\
首先求切线的公式:\\
由
$\frac{x^2}{a^2} + \frac{y^2}{b^2}=1$
得到:
$$
y^2 = b^2 - \frac{b^2}{a^2} x^2
$$
两边同时对x求导,得到:
$$
2y \frac{dy}{dx} = - \frac{b^2}{a^2}2x
$$
设切线斜率为$k$,即:
$$
k = y' = - \frac{b^2 x}{a^2 y}
$$
此处切点为$(\xi, \eta)$,所以,切线为:
$$
y - \eta = - \frac{b^2 \xi}{a^2 \eta}(x - \xi)
$$
可以得到截距:
$$
\begin{cases}
X = \frac{b^2\xi^2 + a^2\eta^2}{b^2\xi}
	= \frac{a^2}{\xi}\\
Y = \frac{b^2}{xi}
\end{cases}
$$
(化简时用到:$
\frac{\xi^2}{a^2} = \frac{\eta^2}{b^2} = 1
$)

\subsection{计算圆锥体积}
此处直接使用圆锥公式计算,当然,如果是其他曲线旋转一周的话,一定要像答案一样采用积分。
$$
V = \frac{1}{3} \pi r^2 h
$$
得到:
$$
V = \frac{\pi a^2 b^4} {3\eta^2 \xi}
$$

\subsection{求体积最小值}
由\textbf{P55 3}得到最大最小值求解的方法。
\paragraph 在条件
$ \frac{\xi^2}{a^2} = \frac{\eta^2}{b^2} = 1 $
下求解:
$$
V \propto \frac{1}{\eta^2\xi}
$$
所以,求$V$最小值,就是求$\eta^2\xi$最大值。
设$g(\xi, \eta) = \xi \eta^2$:
由 
$ \frac{\xi^2}{a^2} = \frac{\eta^2}{b^2} = 1 $
得到$\xi,\eta$互相间的关系,进行替代,得到:
$$
g(\xi, \eta) = \xi b^2 \frac{a^2 - \xi^2}{a^2}
$$
求解$g(\xi, \eta)$的最大值.\\
求微:
$$
\frac{dg}{d\xi} = \frac{b^2}{a^2} (a^2 - 3\xi^2)
$$
取的其极值点:
$
\xi = \frac{a}{\sqrt{3}}
$
\paragraph{判断有效性}
\begin{enumerate}
    \item 当 $0< \xi <\frac{a}{\sqrt{3}} $时,$\frac{dg}{d\xi}>0$
    \item 当 $\frac{a}{\sqrt{3}}$时, $\frac{dg}{d\xi}>0$
\end{enumerate}
得到 
$
\xi = \frac{a}{\sqrt{3}}时,V有唯一最小值
$

\subsection{公式积累}
\paragraph{圆锥体积}
$$
V = \frac{1}{3} \pi r^2 h
$$
\paragraph{球体积}
$$
V = \frac{4}{3} \pi r^3
$$


\section{P69 例12}
此为经典题型,要求$\xi \in (0, 1)$,使$(1+ \xi) f'(\xi) = f(\xi)$.
就是求解$(1+x)f'(x) = f(x)$,使得其在$x \in (0,1)$内有解。
$$
f'(x) = \frac{f(x)}(1+x)
$$
$$
\frac{df(x)}{dx} = \frac{f(x)}{1+x}
$$
这是一个很标准的微分函数\\
将两侧分子分母交换得到:
$$
\frac{df(x)}{f(x)} = \frac{dx}{1+x}
$$
通过两侧同时积分求解:
$$
\int{\frac{df(x)}{f(x)}} = \int{\frac{dx}{1+x}}
$$
得到$\ln{\left| f(x) \right|} = \ln{\left| 1+x \right|} + C$
改写为方程
$$
\frac{f(x)}{1+x} = C
$$
不妨设:
$$
g(x) = \frac{f(x)}{1+x} - C 
$$
此时可以看到,题目中$x = 1, 0$,代入:
有
$$
g(1) = \frac{f(1)}{2} - C = \frac{2f(0){2}} - C = f(0) 
$$
$$
g(0) = f(0) 
$$
所以:
$$
g(1) = g(0)
$$
利用\textbf{P61 罗尔定理}得到:
必存在$x \in (0,1)$,使得
$$
g'(x) = 0  = \frac{df(x)}{f(x)} = \frac{dx}{1+x}
$$
不妨将此解$x$表示为$\xi$\\
得证.

\section{P70 例13}
此题较难把握,题型难以掌握,建议跳过

\section{P71 例14(2)}
\paragraph{此部分,还是要把几个不等式好好熟悉一下}
\paragraph{首先,大题中,后面的小题肯定会用到前面小题中的结论,特别是前面小题比较简单的时候,就是提示}
首先由(1),结合\textbf{P61 拉格朗日中值定理}得到:
$$\exists \eta \in (0, \xi), f'(\eta)(\xi - 0) = \frac{a}{a+b}$$
$$\exists \zeta \in (\xi, 1), f'(\zeta) (1-\zeta) = \frac{b}{a + b}$$
将
$f'(\xi)$和$f'(\zeta)$代入
$$
\frac{a}{f'{\eta}} + \frac{b}{f'{\zeta}} = a+ b
$$
得证。

\section{P71 例15}
\paragraph{ P61 费马引理、罗尔定理、拉格朗日定理等等要非常熟悉,基本上这些都是重点,简单点的应用(小题)很频繁。 }
\textbf{具体的可以记住一些具体的图形例子。}
\begin{enumerate}
    \item A: 由罗尔定理可知,$f'(x)$的零点与$f(x)$并无直接关系,所以不要被迷惑
    \begin{center}
    反例
        \begin{pspicture}(-4,-3)(4,1.2)
        %\psgrid[subgriddiv=1,griddots=10,gridlabels=7pt](-3,-3)(3,3)
        \psaxes{->}(0,0)(-3.5,-2.7)(3.5,1)
        \psline[linewidth=1pt,linearc=.5,linecolor=red]{-}(-2, -2)(0, 0.2)(2, -2)
        \rput[bl](.2, .2){O}
        \end{pspicture}
    \end{center}
    
    \item B:\\
    \begin{center}
    反例
        \begin{pspicture}(-4,-3)(4,1.2)
        %\psgrid[subgriddiv=1,griddots=10,gridlabels=7pt](-3,-3)(3,3)
        \psaxes{->}(0,0)(-3.5,-2.7)(3.5,1)
        \psline[linewidth=1pt,linearc=.5,linecolor=red]{-}(-2, -2)(0, -.5)(2, -2)
        \rput[bl](.2, .2){O}
        \end{pspicture}
    \end{center}

    \item C:\\
    \begin{center}
    反例
        \begin{pspicture}(-4,-3)(4,1.2)
        %\psgrid[subgriddiv=1,griddots=10,gridlabels=7pt](-3,-3)(3,3)
        \psaxes{->}(0,0)(-3.5,-2.7)(3.5,1)
        \psline[linewidth=1pt,linearc=.5,linecolor=red]{-}(-2, -2)(0, -.5)(2, -2)(3, -.3)(4, -2)
        \rput[bl](.2, .2){O}
        \end{pspicture}
    \end{center}
\end{enumerate}

\section{P72 例17}
此处采用的是类似书本上拉格朗日中值公式的证明方法。
拉格朗日中值法:
$$
f(b) - f(a) = f'(\xi)(b-a)
$$
或者
$$
f'(\xi) = \frac{f(b) - f(a)}{(b-a)}
$$
很类似题目中的形式,
参照证明中的方法:
\begin{pspicture}(-4,-3)(4,4)
%\psgrid[subgriddiv=1,griddots=10,gridlabels=7pt](-3,-3)(3,3)
\psaxes{->}(0,0)(-3.5,-2.7)(3.5,2.7)
\psline[linestyle=dashed, dash=3pt 2pt,linewidth=1pt,linearc=0]{-}(-2, .5)(3, 3)
\psline[linewidth=1pt,linearc=1]{-}(-3, -1)(-2, .5)(0, 2)(3, 3)(4, 2)
\psline[linestyle=dashed, dash=3pt 2pt,linewidth=1pt,linearc=0,linecolor=red]{-}(-1, 1.5)(1, 2.5)
\rput[B](-2, .2){a}
\rput[B](3, .2){b}
\rput[br](-2, .5){A}
\rput[bl](3, 3){B}
\rput[bl](-.4, 2.2){$l_1$}
\end{pspicture}

\subsection{AB直线公式}
斜率
$$
k = \frac{f(b) - f(a)}{b - a}
$$
$$
y - f(a) = k (x - a)
=  \frac{f(b) - f(a)}{b - a} (x-a)
$$
\subsection{考察直线AB和$f(x)$的距离}
$$
\varphi(x) = f(x) - y
= f(x) - [f(a) +\frac{f(b) - f(a)}{b - a} (x-a) ]
$$
\paragraph{分析}
$\varphi(x)$表示曲线AB上任意一点到直线AB的距离。当$\varphi'(x)=0$时,就是此点上切线与直线AB平行的情况。
$$
\varphi'(x) =
f'(x) - \frac{f(b) - f(a)}{b - a}
$$
因为
$$
\varphi(a) = \varphi(b) = 0
$$
因此,在AB两点间,如果$\varphi(x)==0$,则$f(x)$曲线与直线AB吻合,所以$f(x)$就是一次式。与题目矛盾。
所以,至少存在一个点$x=\xi$,使得$\varphi(\xi)>0 or \varphi(\xi)<0$
这两种情况均会得到:
$$
\left| f'(\xi) \right| > 
\left| \frac{f(b) - f(a)}{b-a} \right|
$$
得证

\section{P73 例18}
\subsection{连续与可导的关系}
自己也注意积累,我只列出了几个最基础的
\begin{enumerate}
    \item 连续是可导的必要条件(前提),所以,知道可导,则说明必连续
    \item 可导,不代表导数连续。 所以,任意一点的导数及导数的极限并没有很直接的关系。 当考虑到导数的极限时,确定其导数是否有充分条件连续
    \item 一阶可导=>连续,二阶可导=>一阶导数连续,反之不成立
\end{enumerate}
所以,相应得到:
\begin{enumerate}
    \item A: 就是连续性的判断,如果成立,就说明$f'(x)$在$x=x_0$处连续
    \item B: 同样是连续性的判断条件
    \item D: 
    \begin{center}
    反例
        \begin{pspicture}(-4,-3)(4,1.2)
        %\psgrid[subgriddiv=1,griddots=10,gridlabels=7pt](-3,-3)(3,3)
        \psaxes{->}(0,0)(-3.5,-2.7)(3.5,1)
        \psline[linewidth=1pt,linecolor=red]{-}(-2, -2)(0, 0)(2, -2)
        \rput[bl](.2, .2){O}
        \end{pspicture}
    \end{center}
    在O点处导数不存在(两侧导数值不一致)
\end{enumerate}
    %\item C:可以积累一下正确结论:
    %$$ \lim_{x \rightarrow x_0}{\frac{f(x) - f(x_0)}{x - x_0}} = \lim_{x \rightarrow x_0}{f'(x)} $$
    \subsection{衍生和积累}
    自己还是要注意,由点到面的复习方法。 此处就可以把连续性和可导性那部分好好看看。会比单独一道题要收获很多。
    下次的话,同类型的题目,题型方法OK,知识点也懂的话,就没有问题了。

    \section{P82 例1}
    \subsection{列出分段函数}
    需要确立好$x$的前提范围
    \subsection{进行积分}
    \subsection{参数统一}
    因为最终的分段函数的范围需要连接起来
    得到了:

    $$
	\int { \left| 1-\left| x \right|  \right|  } dx
    =
	\begin{cases}
	\begin{array}{ll}
	- \frac{x^2}{2} - x - 1 + C_1,	&	x \le -1\\
	\frac{x^2}{2} + x + C_2,			&	-1 < x \le 0\\
	x - \frac{x^2}{2} + C_3,			&	0< x \le 1\\
	\frac{x^2}{2} - x + 1 + C_4,		&	x > 1
	\end{array}
	\end{cases}
    $$

    此处$C_1, C_2, C_3, C_4$的来源:分段函数的各段都是由单独的积分而来,常数C并不统一,因此每段用$C_1,C_2,C_3,C_4$单独替代。
    如此,每一段函数间是有联系的。
    \paragraph{参照 P80 定理 3.1.1 定积分存在定理}
    不定积分的原函数都是能够可导的,如:
    $$
    F(x) = \int{f(x)dx}
    $$
    则
    $$
    F'(x) = f(x)
    $$

    既然可导,说明,F(x)连续。\\
    因此,可以积累结论:不定积分必连续,且可导。
    此时,可以得到:分段函数应该是可导连续的。因此,函数$f(x)$需要在$x = -1, 0, 1$处均连续。
    因此:

    $$
    x \rightarrow -1:
	- \frac{x^2}{2} - x - 1 + C_1
    =
	\frac{x^2}{2} + x + C_2
    $$
    $$
    \cdots
    $$

    \section{P83 例2}
    \begin{center}
    $$
    t \in [0, 1]
    $$
    \begin{pspicture}(-.5,-.5)(4,4)
    \psset{unit=2cm}
    \psgrid[subgriddiv=1,griddots=10,gridlabels=7pt](-.2,-.2)(1.5, 1.5)
    \psaxes{->}(0,0)(-.2,-.2)(1.5,1.5)
    \psframe(0, 0)(1, 1) 
    \psline[linestyle=dashed, dash=3pt 2pt,linewidth=1pt,linearc=0]{-}(-1, 1.5)(1.5, -1)
    \pspolygon [fillstyle=vlines] (0,0)(.5, 0)(0, .5)
    \end{pspicture}
    \end{center}

    \begin{center}
    $$
    t \in [1, 2]
    $$
    \begin{pspicture}(-.5,-.5)(4,4)
    \psset{unit=2cm}
    %\psgrid[subgriddiv=1,griddots=10,gridlabels=7pt](-.2,-.2)(1.5, 1.5)
    \psaxes{->}(0,0)(-.2,-.2)(1.5,1.5)
    \psframe(0, 0)(1, 1) 
    \psline[linestyle=dashed, dash=3pt 2pt,linewidth=1pt,linearc=0]{-}(0, 1.5)(1.5, 0)
    \pspolygon[fillstyle=vlines](0, 0)(0, 1)(.5, 1)(1, .5)(1, 0)
    %\pspolygon [fillstyle=vlines](1, 0)(1, .5)(.5 1)(0, 1)
    %总长度
    \psaxes{|-|}(0, -.1)(1.5, -.1)
    \rput[t](0.7, -0.2){t}
    %
    \psaxes{|-|}(1, -.3)(1.5, -.3)
    \rput[t](1.2, -0.4){t-1}
    % 
    \psaxes{|-|}(1.1, .5)(1.1, 1)
    \rput[lt](1.2, 0.75){$2 - t$}
    
    \end{pspicture}
    \end{center}
    可以直接考虑正方形总面积$S$去除空白处的面积$S'(t)$\\
    $$
    S'(t) = \frac{1}{2} (2-t)^2
    $$
    $S - S'(t)$便得
    \begin{center}
    $$
    t>2
    $$
    \begin{pspicture}(-.5,-.5)(4,4)
    \psset{unit=2cm}
    \psgrid[subgriddiv=1,griddots=10,gridlabels=7pt](-.2,-.2)(2, 2)
    \psaxes{->}(0,0)(-.2,-.2)(1.5,1.5)
    \psframe [fillstyle=vlines] (0, 0)(1, 1) 
    \psline[linestyle=dashed, dash=3pt 2pt,linewidth=1pt,linearc=0]{-}(2, 0)(0, 2)
    \end{pspicture}
    \end{center}

    \section{P84 例4}
    多联系,这类题只是步骤稍微多了点,按照步骤都可以做出来。
    自己懂了之后,就会发现,这个题目只是两个题目的组合而已。
    \subsection{分段函数}
    将分段函数$f(x)$按照不同的范围进行积分\\
    \begin{enumerate}
        \item 当$x<0$\\
        $$
        \begin{array}{ll}
        F(x) & = \int_{1}^{x}{f(t)dt}\\
            &   = -\int_{x}^{1}{f(t)dt}\\
            &   = -\int_{x}^{0}{f(t)dt} -\int_{0}^{1}{f(t)dt}   \\
            &   = -\int_{x}^{0}{e^tdt} - \int_{0}^{1}{xdt}\\
            &   = e^x - \frac{3}{2}
        \end{array}
        $$
        \item 当$x \ge 0$时\\
        $$
        \begin{array}{ll}
        F(x)    & =- \int_{x}^{1}{tdt} \\
                & =\frac{x^2}{2} - \frac{1}{2}
        \end{array}
        $$
    \end{enumerate}
    所以
    $$
    F(X) = 
    \begin{cases}
        \begin{array}{ll}
        e^x - \frac{3}{2},          &    x<0\\
        \frac{x^2}{2} - \frac{1}{2},&    x\ge 0
        \end{array}
    \end{cases}
    $$
    如此,只是判断一个普通的分段函数在某个点处的性质
    \subsection{判断性质}
    $\cdots$

    \section{P85 例5}
    \paragraph{参照 P80 定积分存在定理,原函数存在定理}
    于是判断两个函数$f(x),g(x)$的连续性
    $f(x)$在0处不连续
    $$
    \lim_{x \rightarrow 0}{g(x)}
    = 
    \lim_{x \rightarrow 0}{x\sin{\frac{1}{x}}}
    =
    \lim_{x \rightarrow 0}{x\frac{1}{x}} = 0 = g(0)
    $$
    因此连续,如此判断\textcircled{4}正确\\
    $f(x)$只存在几个可数的间断点,因此\textcircled{2}正确

    \section{P85 例6}
    \paragraph{做到这种选择题,如果证明太过复杂,注意排除法,另外,复习的时候,注意积累小结论}
    这道题里面核心点:
    $$
    F(x) = \int{f(x)dx} + C
    $$
    其中参数$C$才是关键。\\
    A,B,C中$=>$不会受到$C$的影响。\\
    但是回头,则需要注意$C$
    \begin{enumerate}
        \item A \quad 判断 =>
        $F(x)$为奇,
        $$
        So \quad F(x) = -F(-x)
        $$
        $$
        Because \quad f(x) = F'(x)
        $$
        $$
        So \quad f(x) = F'(x) = -(F(-x))' = f(-x)
        $$
        所以 =>成立\\

        \item A \quad 判断 <=
        $f(x)$为偶函数
        $$
        \begin{array}{l}
        So \quad f(x) = f(-x)\\
        Because \quad F(x) = \int_{0}^{x}{f(t)dt} + C\\
        So \quad F(-x)  = \int_{0}^{-x}{f(t)dt} + C\\
        Set \quad u = -t\\
        \end{array}
        $$
        因为t的范围为$0 \rightarrow -x$\\
        因此$u=-x$的范围为$0 \rightarrow x$\\
        $$
        \begin{array}{ll}
        So \quad F(-x) & = \int_{0}^{x}{f(-u)d-u} + C\\
                        & = -\int_{0}^{x}{f(u)du} + C
        \end{array}
        $$
        $$
        So \quad F(x) + F(-x) = 2C
        $$
        只有$C=0$时,<=才成立.

        \item B \quad 判断 => \\
        同上\\
        \item B \quad 判断 <=\\
        $f(x)$为奇数
        $$
        \begin{array}{l}
        f(x) = -f(-x)\\
        Because \quad F(x) = \int_{0}^{x}{f(t)dt} + C\\
        F(-x) = \int_{0}^{-x}{f(t)dt} + C\\
        Set \quad u = -t\\
        Because \quad t: 0 \rightarrow -x\\
        So \quad u=-t: -0 \rightarrow -(-x)\\
            \begin{array}{ll}
            So F(-x) & = \int_{0}^{x}{f(-u)d-u} + C\\
                     & = \int_{0}^{x}{f(u)du} + C\\
                     & = F(x)
            \end{array}
        \end{array}
        $$
    \end{enumerate}
    其他两个情况也类似
    \subsection{关键点}
    定积分替换变量的方法
    $$
    \begin{array}{l}
    F(-x) = \int_{0}^{-x}{f(t)dt} + C\\
    Set \quad u = -t\\
    Because \quad t: 0 \rightarrow -x\\
    So \quad u=-t: -0 \rightarrow -(-x)\\
        \begin{array}{ll}
        So F(-x) & = \int_{0}^{x}{f(-u)d-u} + C\\
                 & = \int_{0}^{x}{f(u)du} + C\\
                 & = F(x)
        \end{array}
    \end{array}
    $$


    

\end{multicols}
\end{document}
