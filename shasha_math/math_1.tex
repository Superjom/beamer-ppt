\documentclass[a4paper]{article}
\usepackage{geometry}
\geometry{left=2.5cm,right=2.5cm,top=2.5cm,bottom=2.5cm}
\usepackage{amsmath}
\usepackage{titlesec}
\usepackage{color}
\author{Chunwei Yan}

\title{第一章\quad 函数与极限}
\usepackage{xeCJK}
\usepackage{graphicx}
%\setCJKmainfont{WenQuanYi Micro Hei Mono}
\setCJKmainfont{FangSong} 
\DeclareMathSizes{13}{13}{8.7}{7.5}

\begin{document}\large
%------------------- begin document ------------------------
    \maketitle

\section{前言}
这些都是摘录的书上觉得比较有代表性的一些基础题,比较简单,但是重要的是透出的后面的知识点. 
好好做,每一道题都应该掌握了.

这段时间你在复习后面的内容,可以回头做做前面的小题,最重要的是复习一下前面的题型和知识点.

\section{题目}

%========== section 1 
\subsection{第三节 \quad 函数的极限}
\paragraph{P34 \quad 例6}
函数
$$
f(x)=
\begin{cases}
\begin{array}{ll}
x-1 &   x<0\\
0   &   x=0\\
x+1 &   x>0\\
\end{array}
\end{cases}
$$
当$x\rightarrow 0$ 时,证明$f(x)$的极限不存在.

%========== section 2
\subsection{第四节 \quad 无穷小与无穷大}
\textbf{注意}无穷小指$x \rightarrow 0$,而无穷大指
            $x \rightarrow +\infty$ or $x \rightarrow -\infty$.

%========== section 3
\subsection{第五节 \quad 极限运算法则}

%.1-5 习题
\textcolor{red}{\textbf{P48 \quad 习题\quad 1-5}}
\paragraph{1\quad 计算下列极限}

%---- 1(8)
\subparagraph{(8)}
$$
\lim_{x \rightarrow \infty}
     {
         \frac
         {
             x^2 + x
         }
         {
             x^4 - 3x^2 + 1
         }
     }
$$

%---- 1(11)
\subparagraph{(11)}
$$
\lim_{n \rightarrow \infty}
    {
        1 + \frac{1}{2} + \frac{1}{4} + \cdots + \frac{1}{2^n}
    }
$$

%---- 2(1)
\paragraph{3\quad 计算下列极限}
\subparagraph{(1)}
$$
\lim_{x \rightarrow 2}
     {
         \frac{x^3 + 2x^2}
              {(x-2)^2}
     }
$$

\paragraph{3\quad 计算下了极限}
\subparagraph{(2)}
$$
\lim_{x \rightarrow \infty }
     {
         \frac{ \arctan{x} } {x}
     }
$$

%========== section 6
\subsection{第六节\quad 极限存在准则\quad 两个主要极限}
\textcolor{red}{\textbf{P55 \quad 习题\quad 1-6}}

\paragraph{1. 计算下列极限}
\subparagraph{(5)}
$$
\lim_{x \rightarrow 0}
     {
         \frac{1 - \cos{2x}} 
              {x \sin{x}}
     }
$$

\subparagraph{(6)(x为不等于0的常数)}
$$
\lim_{x \rightarrow \infty}
     {
         2^n \sin{\frac{x}{2^n} }
     }
$$


%========== section 7
\subsection{第七节\quad 无穷小的比较}
\textcolor{red}{\textbf{P59 \quad 习题\quad 1-7}}

\paragraph{4.利用等价无穷小的性质,求下列极限}
\subparagraph{(2)(n,m 为正整数)}
$$
\lim_{x \rightarrow 0}
     {
         \frac{\sin{x^n}}
              {(\sin{x})^m}
     }
$$ 

%========== section 8
\subsection{第八节 \quad  函数的连续性和间断性}
\textcolor{red}{\textbf{P64 \quad 习题\quad 1-8}}
\paragraph{}
讨论函数 $f(x) = \lim_{n \rightarrow \infty} { \frac{1 - x^{2n}} {1 + x^{2n}} }$ 的连续性,若有间断点,判断其类型.

%========== section 9
\subsection{第九节\quad 连续函数的运算与初等函数的连续性}
\paragraph{P68 \quad 例题}
\subparagraph{例6}
求$\lim_{x \rightarrow 0}
     {
         \frac{
             \log_{a}{(1+x)}
            }{x}
     }$

\subparagraph{例7}
求 $\lim_{x \rightarrow 0}
     {
         \frac{a^x - 1}{x}
     }$

\subparagraph{例8}
求 $\lim_{x \rightarrow 0}
     {
         (1 + 2x)^{\frac{3}{\sin{x}}}
     }$


\subsection{P73 总习题一}
\paragraph{8.求下列极限}
%--- 8(1)
\subparagraph{(1)}
$$
\lim_{x \rightarrow 1}
     {
         \frac{x^2 - x + 1}
              {(x-1)^2}
         }
$$

%--- 8(5)
\subparagraph{(5)}
$$
\lim_{x \rightarrow 0}
     {
         (
            \frac{a^x + b^x + c^x}
                 {3}
         )^ {\frac{1}{x}}
     }.
     (a>0,b>0,c>0)
$$


%------------------- end document ------------------------
\end{document}

