\documentclass[a4paper]{ctexart}
\usepackage{geometry}
\geometry{left=2cm,right=2cm,top=2.5cm,bottom=2.5cm}

\author{Chunwei Yan}
\title{<+ title +>}
\begin{document}
    \maketitle
%---content here----

\end{document}

\documentclass{beamer}
\usepackage{ctex} %注意这个宏包
\usetheme{umbc4}
\useinnertheme{umbcboxes}
\setbeamercolor{umbcboxes}{bg=violet!15,fg=black}  % redefine box color!

\title{A STATISTICAL APPROACH TO MACHINE TRANSLATION}
\author{Chunwei Yan}
\institute[PKUSZ]{
\texttt{superjom@sz.pku.edu.cn}
}
\date{\today}

\begin{document}


\begin{frame}
\frametitle{There Is No Largest Prime Number}
\framesubtitle{The proof uses \textit{reductio ad absurdum}.}
\begin{theorem}
There is no largest prime number.
\end{theorem}
\begin{proof}
\begin{enumerate}
\item<1-> Suppose $p$ were the largest prime number.
\item<2-> Let $q$ be the product of the first $p$ numbers.
\item<3-> Then $q + 1$ is not divisible by any of them.
\item<1-> But $q + 1$ is greater than $1$, thus divisible by some prime
number not in the first $p$ numbers.\qedhere
\end{enumerate}
\end{proof}
\uncover<4->{The proof used \textit{reductio ad absurdum}.}
\end{frame}

\end{document}
