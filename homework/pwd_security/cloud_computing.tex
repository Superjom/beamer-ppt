\documentclass[a4paper]{ctexart}
\usepackage{geometry}
\usepackage{pstricks}
\usepackage{multicol}
\usepackage{pst-plot}
\usepackage{color}
\geometry{left=2.5cm,right=2.5cm,top=2.5cm,bottom=2.5cm}

\author{严春伟}
\title{云计算安全}
\begin{document}
    \maketitle
    \section{前言}
    \par 云计算是一种以网络为载体,整合大规模可扩展的服务形式。云计算能够将包括计算、存储、数据、应用等可扩展的,分布式资源高效分配和协同工作,从而实现一种超级计算模式。 
    \par 通过整合相关的资源提供更加由弹性的服务,以及集中式的,更加可靠中央服务,云计算可以说是一种相当成功的商业运行模式,在当今信息时代大行其道。但最着云计算的快速发展,云计算领域也面临着各种潜在的风险和安全隐患。 
    \par 本文将会阐述云计算安全方面的一些技术。 

    \section{云计算安全的挑战}
    \subsection{云安全的一些事故}
    云计算发展的这几年,已经发生了一些事故,这里总结了2012年前的大的事故
    \begin{itemize}
        \item 2008年2月15日Amazon出现了网络服务宕机事件,使得几千个依赖亚马逊的EC2云计算和s3云存储的网站受到影响
        \item 2009年2月24日,Google Gmail邮箱爆发全球性故障,服务中断时间长达4h,起因其一个数据中心例行维护,使得另外一个数据中心过载
        \item 2009年3月7日,Google发生大批用户文件外泄事件
        \item 2009年3月15日,Microsoft的云计算平台Azure停止运行长达22h
        \item 2009年6月11日,Amazon的EC2中断了几个小时,起因是雷击损坏了公司数据中心的电力设施
        \item 2010年1月,几乎6万8千名的Salesforce.com用户经历了至少1个小时的宕机。
        \item 2010年3月,VMware的合作伙伴Terremark就发生了七小时的停机事件
        \item 2011年3月,gmail再次爆发大规模的用户数据泄漏事件,大约有15万Gmail用户在周日早上发现自己的所有邮件和聊天记录被删除
        \item 2011年4月22日,亚马逊云数据中心服务器大面积宕机,这一事件被认为是亚马逊史上最为严重的云计算安全事件
    \end{itemize}
    {\color{red}
    总结这些事故,可以得到一些结论:
    \par 与传统的网络安全多针对软件漏洞不同,云计算安全事故涉及到软件和硬件两个方面。硬件方面,如停电、停机等事故,或者负荷超载等造成的宕机;软件方面,由于一些常规的软件安全方面的漏洞,造成用户的信息外泄,比如Gmail或者Microsoft的用户信息外泄
    \par 云计算框架把更多的计算资源整合成一个整体,但其中一个部分出现问题也会使整体的服务发生很大的影响。比如Gmail全球性的服务中断,仅因为欧洲的一个数据中心超载。
\par 由于云计算厂商同时服务着众多的用户,云计算服务短暂的事故也影响非常广泛。2010年1月著名的云计算厂商Salesforce.com 1个小时的宕机,影响了几乎6万 8千名云计算的用户。}

    \subsubsection{云计算五大问题}
    \begin{enumerate}
        \item 虚拟化安全问题
        \item 数据集中后的安全问题
        \item 云平台可用性问题
        \item 云平台遭受攻击问题
        \item 法律问题
    \end{enumerate}
    \section{云计算安全的关键技术}
        \subsection{数据安全}
        \subsubsection{数据传输安全}
            \par 在使用公共云时,数据加密至关重要。一般云计算服务商会对存储的数据进行加密,在数据传输的时候,采用SSL,SSH等安全协议保证安全访问。但是,还是有一个隐患,那就是在内存中的数据,依旧是明文,这为使得利用操作系统漏洞攻击载入内存中的数据成为可能。
        \subsubsection{数据隔离}
            \par 采用安全独立的云区域提供虚拟机来实现数据资源的高度隔离,除此以外,云服务商与组织内部的通信采用加密的VPN专用通道。

        \subsubsection{数据残留}
            {\color{red}
            \par 由于公共云中,共享资源高度重用,并被用户共享。当一个用户的数据空间被废弃,但是由于内存或者硬盘的物理特性可以被恢复数据,这就客观上残留了用户敏感数据泄露的可能。 因此,云计算服务商需要保证一个用户的数据空间在回收后,必须进行彻底的擦除后再分配给其他用户使用。 在应用中,相关技术已经比较成熟。

        \subsection{应用安全}
            \par 云计算安全的实质是安全责任的转移,云计算时代之前,由单个用户自己负责服务器的安全及维护,而如今,用户将自己的服务架设在云计算框架上,将一部分安全责任转嫁到自己信任的云计算服务提供商身上。 
            \par 将分散的小应用部署到“云”上,可以降低单个用户的运营维护的压力。例如:最近沸沸扬扬的12306订票网站(12月16日和9日,此网站因为空调故障两度瘫痪,影响的用户应该是千万级),如果架构在SAE(sina云计算平台)上,那么应该就不必要有这么多宕机事件(如果出问题,那么大部分责任是SAE的,远不是单纯12306本身的问题了)。另外,如果当初开发的时候就考虑到要部署到“云”上,那么此前的沸沸扬扬的9亿元升级其实只需要在SAE那边扩容而已,而且还可以在重要时段进行暂时性的扩容,以应对春节或者国庆长假时候的压力。但是是否足够信任 “云”上的运行安全,是一个考察决策者智慧和勇气的,在作者看来,如今的12306,除非从底层进行重构,或者部署到“云”上,否则所谓的升级只能是一条不归路。
        \par 就像能量守恒定律一样,安全责任不可能消失,只能通过转移的方式得到最优的效果。}

        \subsection{终端用户的安全}
            \par 对于管理远程“云”上应用的用户,自己所用计算机就像钥匙,钥匙丢了,那就算有最坚固的防盗门也是形同虚设。用户的账户信息的安全也需要用户自己多加保密,同时, 用户所用终端机器上也需要有完善的安全软件的保护。 用户需要注意自己所用终端软件的安全,比如浏览器的安全漏洞问题等,定期做好打补丁以及杀毒软件的更新,从终端角度确保云安全。

        \subsection{应用运维的安全}
            \par 应用在云平台的安全运行,当然,需要云计算服务提供端的高度配合。 如PaaS云提供商能够为用户提供相对安全的应用运行环境,也就是云计算服务提供商承担了运行安全相关的责任,通过一定的措施如虚拟化或者沙箱,保证应用运行时不会受到云架构内部或者外部的侵犯。
            \par {\color{red} 但应用本身的安全责任也需要考虑。 如果应用是用户自行开发,那么用户自己需要承担应用本身的安全责任,如果应用当中用到了第三方的程序,那么第三方需要承担自己程序部分的安全责任。}
            \par 目前PaaS服务提供商为了安全,会提供自行维护安全的平台API,比如很多安全特性被封装成了平台相关的安全对象和Web服务,用户的应用需要注意调用这些“云”上高效集成和安全的接口,能够更加高效地保证其运行的安全。
            \par 而IaaS云提供商的服务利用虚拟机来分隔应用,每个虚拟机相当于一个相对独立和部分完整功能的操作系统。把虚拟机当做为一个沙箱,IaaS提供商的服务并不能够穿透沙箱,因此虚拟机中运行的应用对于云提供商是完全透明的,用户需要自行负责自己应用的大部分安全维护以及其他责任。
            \par 用户自己需要有一定的安全运维的实力,这一方面IaaS云提供商并不会有太大的协助。

        \subsection{虚拟化安全}
        \par 基于虚拟化技术的云计算的安全主要有两方面:一是虚拟化软件的安全; 另外一个是使用虚拟化技术的虚拟服务器的安全。

        \subsubsection{虚拟软件安全}
            \par 该软件层直接部署在裸机之上,虚拟的主要是服务器。IaaS云平台上,该软件层对于用户是透明的,管理方面完全是云提供商的义务。
            \par 由于虚拟化软件层运行在多租户的环境下,一台主机也许被多个用户多个虚拟机分享。 云计算提供商会采取一定的措施,严格限制未授权的用户访问虚拟化软件层,限制对于Hypervisor和其他形式的虚拟化层次的物理和逻辑访问控制。

        \subsubsection{虚拟服务器安全}
            \par 应该选择具有TPM安全模块的物理服务器,TPM安全模块可以再虚拟服务器启动时,监测用户密码,如果发现密码及用户名的Hash序列不对,就不会启动此虚拟服务器。 尽可能使用新的带有多核的处理器,并支持虚拟技术的CPU,这就能保证CPU之间的物理隔离,提高安全性。
            \par 另外,为了实现各虚拟服务器间物理隔离的目的,应该在创建虚拟服务器的时候,为每一台虚拟机分配一个独立的分区,以实现磁盘页表的分隔。 每台虚拟机系统还需要安全自己独立的完整的安全防护系统,如防火墙、杀毒软件等。 另外,对于虚拟机的备份也非常重要,如数据、配置需要定期备份,以提供完整的增量或差量备份方式。

        \subsection{物理及其他方面的安全}
        \subsubsection{电力及其他设施}
        \subsubsection{人员安全}
        \subsubsection{冗余度和扩展性}
\end{document}


