\documentclass[a4paper]{ctexart}
\usepackage{geometry}
\usepackage{listings}
\usepackage{hyperref}
\geometry{left=2cm,right=2cm,top=2.5cm,bottom=2.5cm}

\author{严春伟}
\title{隐式马尔科夫模型}
\begin{document}
    \maketitle

\section{前言}
\par 
隐马尔科夫模型(hidden Markov model, HMM)是一种统计学习模型,可以用于标注问题。隐马尔科夫模型在序列化的数据的处理中有比较成熟的应用,如:
\begin{itemize}
\item 语音识别
\item 自然语言处理
\item 生物信息
\item 模式识别
\end{itemize}
\par
本文将会论述隐马尔科夫的基本问题,如定义,解决问题,基本算法等,此外,本文还会论述HMM在著名的中文分词库ICTCLAS中的应用原理。

\section{基本概念}
\subsection{定义}
\paragraph{隐马尔科夫模型(HMM)} 描述一个隐藏的马尔科夫链生成不可观察的状态(state)随机序列,在由各个状态生成一个观测而产生观测随机序列的过程。
\par
HMM的确定因素:
\begin{enumerate}
\item 初始状态分布 $\pi$
\item 状态转移概率分布 $A$
\item 观测概率 $B$
\end{enumerate}

\paragraph{隐马尔科夫的要素}
\begin{enumerate}
\item N,表示模型中的状态数目
\item M,表示模型中每个状态不同的观察符号
\item A,状态转移概率分布。$A=\{a_{ij}\}$
    \begin{equation}
    a_{ij} = P(q_t = S_j | q_{t-1} = Sj) \quad 1\lei, j\le N     
    \end{equation}
\item B,观察字符在状态j时的概率分布,$B=\{b_j(k)\}$,其中
    \begin{equation}
    b_j(k) = P(v_k|q_t = S_j) \quad 1\le j \le N, 1 \le k \le M
    \end{equation}
\end{enumerate}

\section{HMM的概率计算}
\section{HMM的系数学习}
\section{HMM在ICTCLAS中的应用}




\end{document}
