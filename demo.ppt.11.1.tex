\documentclass{beamer}
\usepackage{ctex} %注意这个宏包
\usepackage{multicol}
\usepackage{color}
%\usetheme{umbc4}
\usetheme[height=12mm]{Rochester}
%\usetheme{Berkeley}
\useinnertheme{Marburg}
\useinnertheme{rounded}
\usepackage{pstricks}
\usepackage{pst-plot}
\CTEXoptions[today=old]
%\setbeamercolor{umbcboxes}{bg=violet!15,fg=black}  % redefine box color!

\title{A STATISTICAL APPROACH TO MACHINE TRANSLATION}
\author{Chunwei Yan}
\institute[PKUSZ]{
\texttt{superjom@sz.pku.edu.cn}
}
\date{\today}

\begin{document}

% ------------- title page ----------------------------
%--- the titlepage frame -------------------------%
\begin{frame}[plain]
  \titlepage
\end{frame}

%--- overview -------------------------%
\begin{frame}{Overview}
    \begin{enumerate}
    \item A Brief History
    %\pause
    \item Interest in MT
    %\pause
    \item Language Models
    \item Translation Models
    \item Demo of Word Alignment Calculation
    \end{enumerate}
\end{frame}


%--- history -------------------------%
\begin{frame}{A Brief History}
    \begin{itemize}
        \item One of the first application envisioned for computers
        \pause
        \item First idea  of SMT:Warren Weaver(沃伦·韦弗)
            \begin{itemize}
                \item 1949
                \item Claude Shannon's(香侬) information theory
            \end{itemize}
        \pause
        \item Re-introduced in 1991 by researchers at IBM's Thomas J. Watson Research Center
    \end{itemize}
\end{frame}

%--- interests -------------------------%
\begin{frame}{Interest in MT}
    \begin{itemize}
        \item Translation is a universal need
        \item MT is popular on the web
            \begin{itemize}
                \item Google Translator
                \item Bing Translator
                \item 百度在线翻译
            \end{itemize}
        \item (Semi-)automated translation could lead to huge savings
    \end{itemize}
\end{frame}

%--- introduction -------------------------%
\begin{frame}{Introduction to Translation}
    \begin{definition}
    Job of a translator: render in one language the meaning expressed by a passage of text in another language
    \end{definition}
\end{frame}

\begin{frame}{Introduction to Translation}
    \begin{block}{Assume}
    Every sentence in one language is a possible translation of any sentence in the other.
    \begin{center}
        \begin{tabular}{ l  l }
            \hline
            C1  &   E1\\
            \hline
            C2  &   E2\\
            \hline
            C3  &   E3\\
            \hline
        \end{tabular}
    \end{center}
    \end{block}

    \pause
    \begin{block}{Program 1}
        \begin{center}
        \begin{pspicture}(-1.2,-1.2)(4,1.2)
            \psline[linewidth=1pt,linearc=0]{->}(-1,1)(0,1)
            \psline[linewidth=1pt,linearc=0]{->}(-1,-1)(0,-1)
            \psline[linewidth=1pt,linearc=0]{->}(2.3,0)(3.1,0)
            \psframe[fillstyle=solid,fillcolor=lightgray](0.5,-1.2)(2,1.2)
            %--- charactors--
            \rput[br]{*0}(-1,1){\emph{e}}
            \rput[br]{*0}(-1,-1){\emph{c}}
            \rput[br]{*0}(3.5,0){\emph{$P(e|c)$}}
        \end{pspicture}
        \end{center}
    \end{block}


\end{frame}

%--- introduction -------------------------%
\begin{frame}{Introduction}
    \begin{block}{Assume}
    Given a Chinese sentence c, seek the English sentence e that maximizes $P(e|c)$
    \end{block}
    \pause
    \begin{block}{ Program 2}
        \begin{center}
        \begin{pspicture}(-1.2,-1.2)(5.5,1.2)
            \psline[linewidth=1pt,linearc=0]{->}(2.3,1)(3,1)
            \psline[linewidth=1pt,linearc=0]{->}(2.3,-1)(3,-1)
            \psline[linewidth=1pt,linearc=0]{->}(-1,0)(0,0)
            \psframe[fillstyle=solid,fillcolor=lightgray](0.5,-1.2)(2,1.2)
            %--- charactors--
            \rput[br]{*0}(4.7,0.8){\emph{e1,$P(e1|c)$}}
            \rput[br]{*0}(3.5,0){\emph{$\cdots$}}
            \rput[br]{*0}(4.7,-0.8){\emph{en,$P(e2|c)$}}
            \rput[br]{*0}(-1,0){\emph{c}}
        \end{pspicture}
        \end{center}
    \end{block}
    \pause
    \begin{displaybox}{5cm}     % width of the box is 5cm 
    \[ arg\max_{e}{P(e|c)} \] 
     \end{displaybox} 
\end{frame}

% ------ models -------------------------------
\begin{frame}{Introduction to Models}
    Using Bayes Rule(贝叶斯法则):
    $$
    P(e|c) = \frac{P(ec)}{P(c)} = \frac{P(c|e)P(e)}{P(c)}
    $$
    \begin{center}
        So: $P(e|c) \sim P(c|e)P(e)$
    \end{center}
    \pause
    \begin{displaybox}{9cm}     % width of the box is 5cm 
    \[ 
        arg\max_{e}{P(e|c)} = arg\max_{e}{P(e)*P(c|e)}
     \] 
     \end{displaybox} 
\end{frame}

% ------ models jieshi --------------------------
\begin{frame}{Noisy Model}
    \begin{displaybox}{5cm}     % width of the box is 5cm 
    \[ arg\max_{e}{P(e|f)P(e)} \] 
     \end{displaybox} 
    \pause
    
    \begin{block}{Bayesian Reasioning}
        Observe f and try to come up with the most likely translation e, every e gets the score $P(e)*P(f|e)$
        \begin{itemize}
            \item Medical symptom
                \pause
                \begin{itemize}
                    \item The probability that symptoms f will arise from disease e \textcolor{blue}{\textbf{P(f|e)}}
                \pause
                    \item Is disease e a common disease? \textcolor{blue}{\textbf{P(e)}}
                \end{itemize}
        \end{itemize}
    \end{block}
\end{frame}

\begin{frame}{Noisy Model}
    \begin{displaybox}{5cm}     % width of the box is 5cm 
    \[ arg\max_{e}{P(e|f)P(e)} \] 
     \end{displaybox} 
    \pause
    \begin{enumerate}
    \item Imagine that someone has e in his head \textcolor{blue}{\textbf{P(e)}} 
    \pause
    \item By the time it gets on to the printed page it is corrupted by "noise" and becomes f \textcolor{blue}{\textbf{P(f|e)}}
    \end{enumerate}
\end{frame}

%------ translation model ----------------------
\begin{frame}{SMT Model}
    \begin{block}{$P(e)$ Language Model}
        Is the disease symptom common?\\
        Or is the English sentence common?
        \begin{itemize}
            \item good enough
            \item grammatically right
        \end{itemize}
        \pause
        $P(e)$ worries about English word order and asign a high score to e if it is grammatical
    \end{block}
    \pause
    \begin{block}{$P(e|f)$ Translation Model}
        Will symptoms f arise from disease e?\\
        Or can e sentence be transformed to f?\\
        \pause
        $P(f|e)$ will ensure that a good e will have words that generally translate to words in f.
    \end{block}
\end{frame}

%------ translation model ----------------------
\begin{frame}{SMT Model Demo}
    \begin{block}{$P(e)$ Language Model}
        \begin{enumerate}
            \item a bites dog him
            \item bites hime a dog
            \item a dog bites him
        \end{enumerate}
    \end{block}
    \pause
    \begin{block}{$P(e|f)$ Translation Model}
        \textcolor{blue}{\textbf{Word bag}} \\
        Turn a bag of French words into a bag of English words\\
        Translate "一只狗咬了他" to 
        \begin{enumerate}
            \item a bites dog him
            \item bites hime a dog
            \item a dog bites him
        \end{enumerate}
    \end{block}
\end{frame}

%-------- Language Model --------------------
\begin{frame}{Language Model}
        Language Model worries about English word order
    \pause
    \begin{block}{N-grams}
        \begin{tabular}{ l  l  l}
            n=1 & unigram & P(x)\\
            n=2 & bigram  & P(y|x)\\
            n=3 & trigram & P(yz|x)
        \end{tabular}
    \end{block}

\end{frame}

\begin{frame}{Language Model}
Given a word string, $s_1 s_2 \cdots s_n$, we can write
    \pause
    \begin{displaybox}{10cm}     % width of the box is 5cm 
    \[ 
    P(s_1 s_2 \cdots s_n) 
        = P(s_1)P(s_2|s_1) \cdots P(s_n|s_1 s_2 \cdots s_{n-1})
     \] 
     \end{displaybox} 
    \pause
Using Bigram Model:
    \pause
    \begin{displaybox}{10cm}     % width of the box is 5cm 
    \[ 
    P(s_1 s_2 \cdots s_n) 
        \sim P(s_1|NULL)P(s_2|s_1) \cdots P(s_n|s_{n-1})
     \] 
     \end{displaybox} 

    \pause
    \begin{block}{Bigram Model Demo}
        $$
        b(y|x) = \frac{number-of-occurrences(xy)}
                {number-of-occurrences(x)}
        $$
        P(a dog bites him) $\sim$\\
        b(a|NULL) b(dog|a)  b(bites|dog)  b(him|bites)
    \end{block}

\end{frame}

%------- Translation Model--------------------
\begin{frame}{Translation Model}

    \begin{center}
    \begin{pspicture}(-3,-.2)(3,1.3)
        %------ chinese -------
        \rput[br]{*0}(-3,1){\emph{他}}
        \rput[br]{*0}(-1,1){\emph{在}}
        \rput[br]{*0}(1,1){\emph{花园}}
        \rput[br]{*0}(2,1){\emph{散步}}
       %--- charactors--
        \rput[br]{*0}(-3,0){\emph{He}}
        \rput[br]{*0}(-1,0){\emph{walks}}
        \rput[br]{*0}(0,0){\emph{in}}
        \rput[br]{*0}(1.3,0){\emph{the}}
        \rput[br]{*0}(3,0){\emph{garden}}
        %------ links ---------
        \psline[linewidth=1pt,linearc=0]{->}(-3.2,1)(-3.3,.3)
        \psline[linewidth=1pt,linearc=0]{->}(-1.2,1)(-.3,.3)
        \psline[linewidth=1pt,linearc=0]{->}(.7,1)(1,.3)
        \psline[linewidth=1pt,linearc=0]{->}(.7,1)(2.7,.3)
        \psline[linewidth=1pt,linearc=0]{->}(1.5,1)(-1.3,.3)
    \end{pspicture}
    \end{center}

    \pause
    \begin{block}{Fertility Parameters(生殖力系数)}
        The number of the English words that a Chinese word produces in a given alignment
        n(1|他)
        n(2|花园)
    \end{block}

    \pause
    \begin{block}{Distortion Parameters(畸变系数)}
        d(3|2)=1 : frequency of 在->in \\
        d(3|2,4,5)=1: add length of the Chinese and English sentences 
    \end{block}
\end{frame}

\begin{frame}{Translation Model}

    \begin{block}{Translation Parameters}
        The frequency of a Chinese word that connects to a certain English word. \\
        t(walk|散步)=1\\
        t(garden|花园)=1
    \end{block}
    \pause
    \begin{displaybox}{10cm}     % width of the box is 5cm 
    \[ 
        P(a,f|e) = \prod_{i=1}^{l}{
                n(ph_i | e_i)
            }
            *
            \prod_{j=1}^{m}{
                t(f_j | ea_j)
            }
            *
            \prod_{j=1}^{m}{
                d(j | a_j, l, m)
            }
     \] 
     \end{displaybox} 
\end{frame}


\begin{frame}{Translation Model}
If we had a bunch of English strings and a bunch of step-by-step rewritings into French, then life would be easy.
    \pause
    \begin{block}{Translation Sequence}
        他 在 花园 散步 => He walks in the garden\\

        \begin{center}
        \begin{pspicture}(-3,-.2)(3,1.3)
            %------ chinese -------
            \rput[br]{*0}(-3,1){\emph{他}}
            \rput[br]{*0}(-1,1){\emph{在}}
            \rput[br]{*0}(1,1){\emph{花园}}
            \rput[br]{*0}(2,1){\emph{散步}}
           %--- charactors--
            \rput[br]{*0}(-3,0){\emph{He}}
            \rput[br]{*0}(-1,0){\emph{walks}}
            \rput[br]{*0}(0,0){\emph{in}}
            \rput[br]{*0}(1.3,0){\emph{the}}
            \rput[br]{*0}(3,0){\emph{garden}}
            %------ links ---------
            \psline[linewidth=1pt,linearc=0]{->}(-3.2,1)(-3.3,.3)
            \psline[linewidth=1pt,linearc=0]{->}(-1.2,1)(-.3,.3)
            \psline[linewidth=1pt,linearc=0]{->}(.7,1)(1,.3)
            \psline[linewidth=1pt,linearc=0]{->}(.7,1)(2.7,.3)
            \psline[linewidth=1pt,linearc=0]{->}(1.5,1)(-1.3,.3)
        \end{pspicture}
        \end{center}

        \begin{itemize}
            \item input> 他 在 花园 散步 
            \pause
            \item 他在花园花园散步\textcolor{blue}{< Fertility Parameter}
            \pause
            \item He in the garden walks \textcolor{blue}{< Translation Model}
            \pause
            \item He walks in the garden\textcolor{blue}{< Language Model}
        \end{itemize}
    \end{block}
\end{frame}

%--- word-for-word alignment -------------------------%
\begin{frame}{Word-for-Word Alignments}
    \begin{center}
    \begin{pspicture}(-3,-.2)(3,1.3)
        %------ chinese -------
        \rput[br]{*0}(-3,1){\emph{他}}
        \rput[br]{*0}(-1,1){\emph{在}}
        \rput[br]{*0}(1,1){\emph{花园}}
        \rput[br]{*0}(2,1){\emph{散步}}
       %--- charactors--
        \rput[br]{*0}(-3,0){\emph{He}}
        \rput[br]{*0}(-1,0){\emph{walks}}
        \rput[br]{*0}(0,0){\emph{in}}
        \rput[br]{*0}(1.3,0){\emph{the}}
        \rput[br]{*0}(3,0){\emph{garden}}
        %------ links ---------
        \psline[linewidth=1pt,linearc=0]{->}(-3.2,1)(-3.3,.3)
        \psline[linewidth=1pt,linearc=0]{->}(-1.2,1)(-.3,.3)
        \psline[linewidth=1pt,linearc=0]{->}(.7,1)(1,.3)
        \psline[linewidth=1pt,linearc=0]{->}(.7,1)(2.7,.3)
        \psline[linewidth=1pt,linearc=0]{->}(1.5,1)(-1.3,.3)
    \end{pspicture}
    \end{center}
    
    \pause
    \begin{itemize}
        \item n(2|在):See how many times "在" connected to two English words
        \item t(garden|花园):Count up all the English words generated by all the occurrences of "花园", and see how many of those words are "garden"
    \end{itemize}

\end{frame}

%--- Estimating Parameter Values from Word-for-word Alignments
\begin{frame}{Estimationg Parameter Values}
    \begin{displaybox}{8cm}     % width of the box is 5cm 
    \[ 
        d(p_e|p_c,l_c,l_e) = 
        \frac{dc(p_e|p_c,l_c,l_e)}
        {
            \sum_{j=1}^{N}{(j|p_c,l_c,l_e)}
        }
     \] 
    \end{displaybox} 
    \pause

    \begin{displaybox}{8cm}     % width of the box is 5cm 
    \[ 
        n(i|c) = \frac{Count(c-connect-i-words)}
            {
                    Count(c-connections)
            }
     \] 
    \end{displaybox} 
    \pause

    \begin{displaybox}{8cm}     % width of the box is 5cm 
    \[ 
        t(e|c) = \frac{c-connect-e}
                    {Count(c-connections)}
     \] 
    \end{displaybox} 
\end{frame}

%---- Chicken and Egg-----------------------------------
\begin{frame}{EM (Estimation-Maximiztion) algorithm}
    \begin{center}
    \begin{overprint} 
        \onslide<1>
        \begin{block}{Compute parameter estimates}
        Given the alignment, we compute parameters
            \begin{center}
            \begin{pspicture}(-3,-.2)(3,1.3)
                %------ chinese -------
                \rput[br]{*0}(-3,1){\emph{他}}
                \rput[br]{*0}(-1,1){\emph{在}}
                \rput[br]{*0}(1,1){\emph{花园}}
                \rput[br]{*0}(2,1){\emph{散步}}
               %--- charactors--
                \rput[br]{*0}(-3,0){\emph{He}}
                \rput[br]{*0}(-1,0){\emph{walks}}
                \rput[br]{*0}(0,0){\emph{in}}
                \rput[br]{*0}(1.3,0){\emph{the}}
                \rput[br]{*0}(3,0){\emph{garden}}
                %------ links ---------
                \psline[linewidth=1pt,linearc=0]{->}(-3.2,1)(-3.3,.3)
                \psline[linewidth=1pt,linearc=0]{->}(-1.2,1)(-.3,.3)
                \psline[linewidth=1pt,linearc=0]{->}(.7,1)(1,.3)
                \psline[linewidth=1pt,linearc=0]{->}(.7,1)(2.7,.3)
                \psline[linewidth=1pt,linearc=0]{->}(1.5,1)(-1.3,.3)
            \end{pspicture}
            \end{center}
            n(2|花园) t(he|他) $\cdots$
        \end{block}

        \onslide<2,3,4>
        \begin{block}{Compute Alignment Probabilities}
            $$
            \textcolor{blue}{P(a,f|e)} = \prod_{i=1}^{l}{
                    \textcolor{red}{n}(ph_i | e_i)
                }
                *
                \prod_{j=1}^{m}{
                    \textcolor{red}{t}(f_j | e_j)
                }
                *
                \prod_{j=1}^{m}{
                    \textcolor{red}{d}(j | a_j, l, m)
                }
            $$
            $$
            \textcolor{red}{P(a|e,f)} = \frac{\textcolor{blue}{P(a,f|e)}}
                            {\sum_{a}{P(a,f|e)}}
            $$
        \end{block}
    \end{overprint}
    \end{center}

    \begin{center}
    \begin{overprint}
        \onslide<3>
        \begin{alertblock}{Chicken and Egg}
            Uh oh, what went wrong?
        \end{alertblock}

        \onslide<4>
        \begin{block}{Chicken and Egg}
            The EM algorithm can solve the problem.
        \end{block}
    \end{overprint}
    \end{center}
\end{frame}

%----- last demo ---------------------------------------------
\begin{frame}{SMT Word-Alignment Demo}
    (一只狗,a dog), (狗,dog)
    \begin{block}{Possible Alignments}
        \begin{center}
        \begin{tabular}{ l  l }
        \textcircled{1}
        %--- image 1 ----------------------
        \begin{pspicture}(-2,-.2)(2,1.3)
            \rput[br]{*0}(-1,1){\emph{一只}}
            \rput[br]{*0}(1,1){\emph{狗}}
            \rput[br]{*0}(-1.4,0){\emph{a}}
            \rput[br]{*0}(1,0){\emph{dog}}
            \psline[linewidth=1pt,linearc=0]{->}(-1.4,1)(-1.4,0.4)
            \psline[linewidth=1pt,linearc=0]{->}(1,1)(1,0.4)
        \end{pspicture} &
        %--- image 2 ---------------------
        \textcircled{2}
        \begin{pspicture}(-2,-.2)(2,1.3)
            \rput[br]{*0}(-1,1){\emph{一只}}
            \rput[br]{*0}(1,1){\emph{狗}}
            \rput[br]{*0}(-1.4,0){\emph{a}}
            \rput[br]{*0}(1,0){\emph{dog}}
            \psline[linewidth=1pt,linearc=0]{->}(-1.4,1)(1,0.4)
            \psline[linewidth=1pt,linearc=0]{->}(1,1)(-1.4,0.4)
        \end{pspicture} \\
        \hline
        %--- image 3 ---------------------
        \textcircled{3}
        \begin{pspicture}(-2,-.2)(2,1.3)
            \rput[br]{*0}(1,1){\emph{狗}}
            \rput[br]{*0}(1,0){\emph{dog}}
            \psline[linewidth=1pt,linearc=0]{->}(1,1)(1,0.4)
        \end{pspicture}&

        \end{tabular}
        \end{center}
    \end{block}
    
\end{frame}

\begin{frame}{EM Process}
    \begin{block}{Possible Alignments}
        \begin{center}
        \begin{tabular}{ l  l }
        \textcircled{1}
        %--- image 1 ----------------------
        \begin{pspicture}(-2,-.2)(2,1.3)
            \rput[br]{*0}(-1,1){\emph{一只}}
            \rput[br]{*0}(1,1){\emph{狗}}
            \rput[br]{*0}(-1.4,0){\emph{a}}
            \rput[br]{*0}(1,0){\emph{dog}}
            \psline[linewidth=1pt,linearc=0]{->}(-1.4,1)(-1.4,0.4)
            \psline[linewidth=1pt,linearc=0]{->}(1,1)(1,0.4)
        \end{pspicture} &
        %--- image 2 ---------------------
        \textcircled{2}
        \begin{pspicture}(-2,-.2)(2,1.3)
            \rput[br]{*0}(-1,1){\emph{一只}}
            \rput[br]{*0}(1,1){\emph{狗}}
            \rput[br]{*0}(-1.4,0){\emph{a}}
            \rput[br]{*0}(1,0){\emph{dog}}
            \psline[linewidth=1pt,linearc=0]{->}(-1.4,1)(1,0.4)
            \psline[linewidth=1pt,linearc=0]{->}(1,1)(-1.4,0.4)
        \end{pspicture} \\
        \hline
        %--- image 3 ---------------------
        \textcircled{3}
        \begin{pspicture}(-2,-.2)(2,1.3)
            \rput[br]{*0}(1,1){\emph{狗}}
            \rput[br]{*0}(1,0){\emph{dog}}
            \psline[linewidth=1pt,linearc=0]{->}(1,1)(1,0.4)
        \end{pspicture}&
        \end{tabular}
        \end{center}
    \end{block}
    \begin{block}{Step 1. Set parameter values uniformly}
        \begin{itemize}
            \item t(a|一只)=1/2
            \item t(dog|一只)=1/2
            \item t(a|狗)=1/2
            \item t(dog|狗)=1/2
        \end{itemize}
    \end{block}
\end{frame}

\begin{frame}{EM Process}
    \begin{block}{Possible Alignments}
        \begin{center}
        \begin{tabular}{ l  l }
        \textcircled{1}
        %--- image 1 ----------------------
        \begin{pspicture}(-2,-.2)(2,1.3)
            \rput[br]{*0}(-1,1){\emph{一只}}
            \rput[br]{*0}(1,1){\emph{狗}}
            \rput[br]{*0}(-1.4,0){\emph{a}}
            \rput[br]{*0}(1,0){\emph{dog}}
            \psline[linewidth=1pt,linearc=0]{->}(-1.4,1)(-1.4,0.4)
            \psline[linewidth=1pt,linearc=0]{->}(1,1)(1,0.4)
        \end{pspicture} &
        %--- image 2 ---------------------
        \textcircled{2}
        \begin{pspicture}(-2,-.2)(2,1.3)
            \rput[br]{*0}(-1,1){\emph{一只}}
            \rput[br]{*0}(1,1){\emph{狗}}
            \rput[br]{*0}(-1.4,0){\emph{a}}
            \rput[br]{*0}(1,0){\emph{dog}}
            \psline[linewidth=1pt,linearc=0]{->}(-1.4,1)(1,0.4)
            \psline[linewidth=1pt,linearc=0]{->}(1,1)(-1.4,0.4)
        \end{pspicture} \\
        \hline
        %--- image 3 ---------------------
        \textcircled{3}
        \begin{pspicture}(-2,-.2)(2,1.3)
            \rput[br]{*0}(1,1){\emph{狗}}
            \rput[br]{*0}(1,0){\emph{dog}}
            \psline[linewidth=1pt,linearc=0]{->}(1,1)(1,0.4)
        \end{pspicture}&

        \end{tabular}
        \end{center}
    \end{block}

    \begin{block}{Step 2. Compute $P(a,f|e)$ for all alignments}
        \begin{itemize}
            \item \textcircled{1} $P(a, f|e) = 1/2 * 1/2
                    = 1/4$
            \item \textcircled{2} $P(a, f|e) = 1/2 * 1/2
                    = 1/4$
            \item \textcircled{3} $P(a,f|e)=1/2$
        \end{itemize}
    \end{block}
\end{frame}


\begin{frame}{EM Process}
    \begin{block}{Possible Alignments}
        \begin{center}
        \begin{tabular}{ l  l }
        \textcircled{1}
        %--- image 1 ----------------------
        \begin{pspicture}(-2,-.2)(2,1.3)
            \rput[br]{*0}(-1,1){\emph{一只}}
            \rput[br]{*0}(1,1){\emph{狗}}
            \rput[br]{*0}(-1.4,0){\emph{a}}
            \rput[br]{*0}(1,0){\emph{dog}}
            \psline[linewidth=1pt,linearc=0]{->}(-1.4,1)(-1.4,0.4)
            \psline[linewidth=1pt,linearc=0]{->}(1,1)(1,0.4)
        \end{pspicture} &
        %--- image 2 ---------------------
        \textcircled{2}
        \begin{pspicture}(-2,-.2)(2,1.3)
            \rput[br]{*0}(-1,1){\emph{一只}}
            \rput[br]{*0}(1,1){\emph{狗}}
            \rput[br]{*0}(-1.4,0){\emph{a}}
            \rput[br]{*0}(1,0){\emph{dog}}
            \psline[linewidth=1pt,linearc=0]{->}(-1.4,1)(1,0.4)
            \psline[linewidth=1pt,linearc=0]{->}(1,1)(-1.4,0.4)
        \end{pspicture} \\
        \hline
        %--- image 3 ---------------------
        \textcircled{3}
        \begin{pspicture}(-2,-.2)(2,1.3)
            \rput[br]{*0}(1,1){\emph{狗}}
            \rput[br]{*0}(1,0){\emph{dog}}
            \psline[linewidth=1pt,linearc=0]{->}(1,1)(1,0.4)
        \end{pspicture}&

        \end{tabular}
        \end{center}
    \end{block}

    \begin{block}{Step 3. Normalize $P(a,f|e)$ values to yield $P(a|e,f)$ values}
        \begin{itemize}
            \item \textcircled{1} $P(a|e,f) =\frac{1/4}{2/4} = 1/2$
            \item \textcircled{2} $P(a|e,f) = \frac{1/4}{2/4}= 1/2$
            \item \textcircled{3} $P(a|e,f)=\frac{1/2}{1/2}= 1$
        \end{itemize}
    \end{block}
\end{frame}

\begin{frame}{EM Process}
    \begin{block}{Step 4. Collect fractional counts}
        \begin{itemize}
            \item tc(a|狗)=1/2
            \item tc(dog|狗)=1 + 1/2 = 3/2
            \item tc(a|一只)=1/2
            \item tc(dog|一只)=1/2
        \end{itemize}
    \end{block}
\end{frame}

\begin{frame}{EM Process}
    \begin{block}{Step 5. Normalize fractional counts to get revised parameter values.}
        \begin{itemize}
            \item t(a|狗)=$\frac{1/2}{4/2}$ = 1/4
            \item t(dog|狗)=$\frac{3/2}{4/2}$ = 3/4
            \item t(a|一只)=$\frac{1/2}{1}$ = 1/2
            \item t(dog|一只)=$\frac{1/2}{1}$= 1/2
        \end{itemize}
    \end{block}
\end{frame}

\begin{frame}{EM Process}
\begin{center}
    \begin{overprint}
    \onslide<1>
    \begin{block}{Repeating steps 2-5 many times}
        \begin{itemize}
            \item t(a|狗)=0.0001
            \item t(dog|狗)=0.9999
            \item t(a|一只)=0.9999
            \item t(dog|一只)=0.0001
        \end{itemize}
    \end{block}

    \onslide<2>
    \begin{block}{Final Word Alignment}
        \begin{itemize}
            \item 一只->a
            \item 狗->dog
        \end{itemize}
    
    \end{block}
    \end{overprint}
\end{center}
\end{frame}

\begin{frame}{Summarize}

    \begin{block}{Translation Sequence}
        他 在 花园 散步 => He walks in the garden\\

        \begin{center}
        \begin{pspicture}(-3,-.2)(3,1.3)
            %------ chinese -------
            \rput[br]{*0}(-3,1){\emph{他}}
            \rput[br]{*0}(-1,1){\emph{在}}
            \rput[br]{*0}(1,1){\emph{花园}}
            \rput[br]{*0}(2,1){\emph{散步}}
           %--- charactors--
            \rput[br]{*0}(-3,0){\emph{He}}
            \rput[br]{*0}(-1,0){\emph{walks}}
            \rput[br]{*0}(0,0){\emph{in}}
            \rput[br]{*0}(1.3,0){\emph{the}}
            \rput[br]{*0}(3,0){\emph{garden}}
            %------ links ---------
            \psline[linewidth=1pt,linearc=0]{->}(-3.2,1)(-3.3,.3)
            \psline[linewidth=1pt,linearc=0]{->}(-1.2,1)(-.3,.3)
            \psline[linewidth=1pt,linearc=0]{->}(.7,1)(1,.3)
            \psline[linewidth=1pt,linearc=0]{->}(.7,1)(2.7,.3)
            \psline[linewidth=1pt,linearc=0]{->}(1.5,1)(-1.3,.3)
        \end{pspicture}
        \end{center}

        \begin{itemize}
            \item input> 他 在 花园 散步 
            \pause
            \item 他在花园花园散步\textcolor{blue}{< Fertility Parameter}
            \pause
            \item He in the garden walks \textcolor{blue}{< Translation Model}
            \pause
            \item He walks in the garden\textcolor{blue}{< Language Model}
        \end{itemize}
    \end{block}
\end{frame}



\begin{frame}{References}
\begin{thebibliography}{10}
\bibitem{Computational Linguistics Volume 16, Number 2, June 1990}[Peter F.Brown, John Cocke, Stephen A. Della Pietra, $\cdots$]
\newblock A STATISTICAL APPROACH TO MACHINE TRANSLATION,June 1990
\bibitem{prepared in connection with the JHU summer workshop}[Kevin Knight]
\newblock A Statistical MT Tutorial Workbook, April 30, 1999
\end{thebibliography}

\end{frame}

\begin{frame}{Thank You!}
    Thank you for your listenning!
\end{frame}

\end{document}
