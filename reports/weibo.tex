\documentclass{beamer}
\usepackage{ctex} %注意这个宏包
\usepackage{multicol}
\usepackage{xcolor}
\usepackage{graphics,graphicx}
\usepackage{color}
%\usetheme{umbc4}
\usetheme{Montpellier} 
\usepackage{pst-plot}
\CTEXoptions[today=old]
\title{从KDDCUP2012看微博好友推荐}
\author{Chunwei Yan}
\institute[PKUSZ]{
\texttt{superjom@sz.pku.edu.cn}
}
\date{\today}

\begin{document}
% ------------- title page ----------------------------
%--- the titlepage frame -------------------------%
\begin{frame}[plain]
  \titlepage
\end{frame}


\begin{frame}{背景}
\begin{itemize}
    \item KDDCUP2012
    \item 腾讯微博
    \item 1st-0.4265 undergrads@数据和知识管理实验室@上海交大
    \item 2st-0.41874 盛大研究院
\end{itemize}
\end{frame}

\begin{frame}{Outline}
    \begin{itemize}
        \item 训练数据格式
        \item 超越矩阵分解模型
        \item 其他方法
        \item 实验分析
    \end{itemize}
\end{frame}

\section{训练数据格式}

\begin{frame}{KDDCUP 2012 Track1}
    \begin{itemize}
    \item 目标: 用户好友推荐, MAP@3
    \pause
    \item datasets: 
        \begin{enumerate}
        \item 训练集:(UserId)(ItemId)(Result)(Unix-timestamp)
        \item 其他信息:
            \begin{itemize}
                \item Profile: (UserId)(Year-of-birth)(Gender)(Number-of-tweet)(Tag-Ids)
                \item item: (ItemId)(Item-Category)(Item-Keyword)
                \item user-action: (UserId)(Action-Destination-UserId)(Number-of-at-action)(Number-of-retweet)(Number-of-comment)
            \end{itemize}
        \end{enumerate}
    \end{itemize}
\end{frame}

\section{常规方法}
\begin{frame}{矩阵分解模型(SVD/SVD++)}
    \begin{equation}
        \hat{r}_{ui} = 
        ( 
        \sum_{c\in C(u)} \textcolor{red}{\alpha_c^{(u)}}\bold{p}_c)^T
        (
        \sum_{c\in C(i)} \textcolor{blue}{\beta_c^{(i)}}\bold{q}_c)
        +
        \sum_{c\in C(u,i)} \textcolor{orange}{\gamma_c^{(u,i)}}g_c
    \end{equation}

    \pause
    \begin{itemize}
        \item $\theta = \{\bold{p}, \bold{q}, g\}$. 模型参数,随机梯度下降
        \item $\color{red}\alpha_c^{(u)}$ 用户特征(user features), tags, keywords, 社交网络
        \item $\color{blue}\beta_c^{(i)}$ item features, 分类,网络
        \item $\color{orange}\gamma_c^{(u,i)}$ 公共特征, user, item 间的交互
    \end{itemize}
\end{frame}

\section{超越矩阵分解}
\begin{frame}{Additive Forest}
    \begin{equation}
        \hat{r}_{ui} = 
        \sum_{s=1}^{S} f_{s,root(i,s)}(x_{u})
        \label{}
    \end{equation}
    \begin{itemize}
        \item $x_{u}$ 用户$u$的特性
        \item $f_{s,root(i,s)}$ 用回归树定义的函数
        \item 采用梯度提升法学习
    \end{itemize}
\end{frame}

\begin{frame}{Additive Forest}
    \begin{center}
        \includegraphics[height=180pt]{additive-tree-1.png}
    \end{center}
\end{frame}

\begin{frame}{Additive Forest 实例}
    \begin{center}
        \includegraphics[height=200pt]{additive-tree-2.png}
    \end{center}
\end{frame}

\begin{frame}{矩阵分解模型 vs Additive Forest}
    \begin{tabular}{ | l | l | l | p{5cm} |}
        \hline
           &矩阵分解 & Additive Forest \\
           \hline
        稀疏矩阵处理 & 非常好         &  一般\\ \hline
        不同信息整合 & 线性组合       & 非线性组合\\ \hline
        对连续值的处理 & 人为划分     & 自动产生划分 \\
        \hline
    \end{tabular}
    \pause
    \begin{itemize}
        \item 两种模型都各有各自的特点
        \item 结合他们的特点对提高精度至关重要
    \end{itemize}
\end{frame}

\begin{frame}{矩阵分解模型 vs Additive Forest}
        \pause
    \begin{columns}[t]
        \column{6cm}
        \begin{block}{矩阵分解模型}
        \begin{equation}
            \hat{r}_{ui} = 
            p_u^T q_i + W_{i, \textcolor{red}{ag(u)}}
            \label{}
        \end{equation}
        \begin{itemize}
        \item $\textcolor{red}{ag(u)}$ 年龄划分索引
        \item 需要实现人为划分
        \end{itemize}
        \begin{center}
            \includegraphics[height=55pt]{fm1.png}
        \end{center}
        \end{block}

        \pause
        \column{6cm}
        \begin{block}{Additive Forest}
            \begin{center}
                \includegraphics[height=145pt]{additive-tree-3.png}
            \end{center}
        \end{block}
    \end{columns}
\end{frame}

\section{其他模型}
\begin{frame}{社交网络}
\begin{equation}
    \hat{r}_{ui} = 
    \textcolor{red}{
    ( \frac{1}{\sqrt{\left|F(u)\right|}}
    \sum_{j\in F(u)}\bold{p}_j)^T}
    \bold{q}_i + b_i
    \label{}
\end{equation}

\pause
\begin{itemize}
    \item $F(u)$ user $u$ follow 的好友
    \item 模拟其好友对其影响
\end{itemize}
\end{frame}

\begin{frame}{关键词和Tag}
    \begin{itemize}
        \item 分类信息对用户的预测有影响 
        \item 影响无法无法计量

        \pause
        \item 作为潜在因素(SVD++)
    \end{itemize}
    \begin{equation}
        \bold{p}'_u = 
        \bold{p}_u + 
            \frac{1}{\left|\left|w_u\right|\right|_2}
            \sum_{j\in K(u)} w_{u,j}\bold{\textcolor{blue}{y}}_j
        \label{}
    \end{equation}
    \begin{itemize}
        \item $K(u)$ 用户$u$的 keywords or tags 
        \item $w_{u,j}$ 特征的权重
    \end{itemize}
\end{frame}

\begin{frame}{用户序列模式(User Sequential Patterns)}
    \begin{itemize}
        \item 推测用户点击趋势
    \end{itemize}

    \begin{center}
        \includegraphics[height=105pt]{qequential1.png}
    \end{center}

    \pause
    \begin{equation}
        \hat{r}'_{ui}{(t)} = 
        \hat{r}_{ui} + f(\triangle  t),
        f(\triangle  t) = \sum_{s=1}^{S} f_s(\triangle  t)
        \label{}
    \end{equation}
\end{frame}

\section{实验结果}
\begin{frame}{实验结果}
    \begin{center}
        \includegraphics[height=148pt]{res.png}
    \end{center}
\end{frame}

\begin{frame}{引用}

\begin{thebibliography}{10}
\bibitem{mapreduce} [Tianqi Chen, Linpeng Tan, QIn Liu and so on,2012]
    \newblock ACM
    \newblock \emph{Combining Factorization Model and Additive Forest for Collaborative Followee Recommendation}, 2004

\bibitem{d} [Yehuda Koren, 2008]
    \newblock \emph{Factorization Meets the Neighborhood: a Multifaceted Collborative Filtering Model}, 2008
\end{thebibliography}
    
\end{frame}

\begin{frame}
    \begin{center}
        {\Large Thank You, Questions? }
    \end{center}
\end{frame}

\end{document}
