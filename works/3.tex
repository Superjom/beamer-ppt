\documentclass[a4paper]{ctexart}
\usepackage{geometry}
\geometry{left=2cm,right=2cm,top=2.5cm,bottom=2.5cm}

\author{Chunwei Yan}
\title{<+ title +>}
\begin{document}
    \maketitle
%---content here----

\section{Text 3}
In the early 1960s Wilt Chamberlain was one of only three players in the National Basketball Association (NBA) listed at over seven feet. If he had played last season, however, he would have been one of 42. The bodies playing major professional sports have changed dramatically over the years, and managers have been more than willing to adjust team uniforms to fit the growing numbers of bigger, longer frames.
\par
The trend in sports, though, may be obscuring an unrecognized reality: Americans have generally stopped growing. Though typically about two inches taller now than 140 years ago, today’s people – especially those born to families who have lived in the U.S. for many generations – apparently reached their limit in the early 1960s. And they aren’t likely to get any taller. “In the general population today, at this genetic, environmental level, we’ve pretty much gone as far as we can go,” says anthropologist William Cameron Chumlea of Wright State University. In the case of NBA players, their increase in height appears to result from the increasingly common practice of recruiting players from all over the world.
\par
Growth, which rarely continues beyond the age of 20, demands calories and nutrients – notably, protein – to feed expanding tissues. At the start of the 20th century, under-nutrition and childhood infections got in the way. But as diet and health improved, children and adolescents have, on average, increased in height by about an inch and a half every 20 years, a pattern known as the secular trend in height. Yet according to the Centers for Disease Control and Prevention, average height – 5′9″ for men, 5′4″ for women – hasn’t really changed since 1960.
\par
Genetically speaking, there are advantages to avoiding substantial height. During childbirth, larger babies have more difficulty passing through the birth canal. Moreover, even though humans have been upright for millions of years, our feet and back continue to struggle with bipedal posture and cannot easily withstand repeated strain imposed by oversize limbs. “There are some real constraints that are set by the genetic architecture of the individual organism,” says anthropologist William Leonard of Northwestern University.
\par
Genetically speaking, there are advantages to avoiding substantial height. During childbirth, larger babies have more difficulty passing through the birth canal. Moreover, even though humans have been upright for millions of years, our feet and back continue to struggle with bipedal posture and cannot easily withstand repeated strain imposed by oversize limbs. “There are some real constraints that are set by the genetic architecture of the individual organism,” says anthropologist William Leonard of Northwestern University.
\par
Genetic maximums can change, but don’t expect this to happen soon. Claire C. Gordon, senior anthropologist at the Army Research Center in Natick, Mass., ensures that 90 percent of the uniforms and workstations fit recruits without alteration. She says that, unlike those for basketball, the length of military uniforms has not changed for some time. And if you need to predict human height in the near future to design a piece of equipment, Gordon says that by and large, “you could use today’s data and feel fairly confident.”
\\\\
31.	Wilt Chamberlain is cited as an example to\\
$[A]$ illustrate the change of height of NBA players.\\
$[B]$ show the popularity of NBA players in the U.S..\\
$[C]$ compare different generations of NBA players.\\
$[D]$ assess the achievements of famous NBA players.\\
\\
32.	Which of the following plays a key role in body growth according to the text?\\
$[A]$ Genetic modification.\\
$[B]$ Natural environment.\\
$[C]$ Living standards.\\
$[D]$ Daily exercise.\\
\\
33.	On which of the following statements would the author most probably agree?\\
$[A]$ Non-Americans add to the average height of the nation.\\
$[B]$ Human height is conditioned by the upright posture.\\
$[C]$ Americans are the tallest on average in the world.\\
$[D]$ Larger babies tend to become taller in adulthood.\\
\\
34.	We learn from the last paragraph that in the near future\\
$[A]$ the garment industry will reconsider the uniform size.\\
$[B]$ the design of military uniforms will remain unchanged.\\
$[C]$ genetic testing will be employed in selecting sportsmen.\\
$[D]$ the existing data of human height will still be applicable.\\
\\
35.	The text intends to tell us that\\
$[A]$ the change of human height follows a cyclic pattern.\\
$[B]$ human height is becoming even more predictable.\\
$[C]$ Americans have reached their genetic growth limit.\\
$[D]$ the genetic pattern of Americans has altered.\\

\section{Text 4}
In 1784, five years before he became president of the United States, George Washington, 52, was nearly toothless. So he hired a dentist to transplant nine teeth into his jaw – having extracted them from the mouths of his slaves.
\par
That’s a far different image from the cherry-tree-chopping George most people remember from their history books. But recently, many historians have begun to focus on the roles slavery played in the lives of the founding generation. They have been spurred in part by DNA evidence made available in 1998, which almost certainly proved Thomas Jefferson had fathered at least one child with his slave Sally Hemings. And only over the past 30 years have scholars examined history from the bottom up. Works of several historians reveal the moral compromises made by the nation’s early leaders and the fragile nature of the country’s infancy. More significantly, they argue that many of the Founding Fathers knew slavery was wrong – and yet most did little to fight it.
\par
More than anything, the historians say, the founders were hampered by the culture of their time. While Washington and Jefferson privately expressed distaste for slavery, they also understood that it was part of the political and economic bedrock of the country they helped to create.
\par
For one thing, the South could not afford to part with its slaves. Owning slaves was “like having a large bank account,” says Wiencek, author of An Imperfect God: George Washington, His Slaves, and the Creation of America. The southern states would not have signed the Constitution without protections for the “peculiar institution,” including a clause that counted a slave as three fifths of a man for purposes of congressional representation.
\par
And the statesmen’s political lives depended on slavery. The three-fifths formula handed Jefferson his narrow victory in the presidential election of 1800 by inflating the votes of the southern states in the Electoral College. Once in office, Jefferson extended slavery with the Louisiana Purchase in 1803; the new land was carved into 13 states, including three slave states.
\par
Still, Jefferson freed Hemings’s children – though not Hemings herself or his approximately 150 other slaves. Washington, who had begun to believe that all men were created equal after observing the bravery of the black soldiers during the Revolutionary War, overcame the strong opposition of his relatives to grant his slaves their freedom in his will. Only a decade earlier, such an act would have required legislative approval in Virginia.\\
\\
36.	George Washington’s dental surgery is mentioned to\\
$[A]$ show the primitive medical practice in the past.\\
$[B]$ demonstrate the cruelty of slavery in his days.\\
$[C]$ stress the role of slaves in the U.S. history.\\
$[D]$ reveal some unknown aspect of his life.\\
\\
37.	We may infer from the second paragraph that\\
$[A]$ DNA technology has been widely applied to history research.\\
$[B]$ in its early days the U.S. was confronted with delicate situations.\\
$[C]$ historians deliberately made up some stories of Jefferson’s life.\\
$[D]$ political compromises are easily found throughout the U.S. history.\\
\\
38.	What do we learn about Thomas Jefferson?\\
$[A]$ His political view changed his attitude towards slavery.\\
$[B]$ His status as a father made him free the child slaves.\\
$[C]$ His attitude towards slavery was complex.\\
$[D]$ His affair with a slave stained his prestige.\\
\\
39.	Which of the following is true according to the text?\\
$[A]$ Some Founding Fathers benefit politically from slavery.\\
$[B]$ Slaves in the old days did not have the right to vote.\\
$[C]$ Slave owners usually had large savings accounts.\\
$[D]$ Slavery was regarded as a peculiar institution.\\
\\
40.	Washington’s decision to free slaves originated from his\\
$[A]$ moral considerations.\\
$[B]$ military experience.\\
$[C]$ financial conditions.\\
$[D]$ political stand.\\
\end{document}


