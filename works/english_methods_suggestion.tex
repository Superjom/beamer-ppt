\documentclass[a4paper]{ctexart}
\usepackage{geometry}
\geometry{left=2cm,right=2cm,top=2.5cm,bottom=2.5cm}

\author{Chunwei Yan}
\title{英语阅读方法}
\begin{document}
    \maketitle
%---content here----

\section{复习建议}
最后一个阶段,是总结和加强的阶段。
\subsection{长难句}
长难句关系到总体的语感和阅读效率,一定要好好把握,把长难句摆在最重要的位置!
\subsection{单词}
最后阶段,没有必要再以量取胜了,重点把之前背过的复习一下。\\
如果时间比较紧的话,可以重点把往年阅读真题中出现过的单词和变形重点过一下,这个工作很重要。 一般准备的话,长难句和单词就OK了。
\subsection{阅读}
至少在离考试一个月的时候,能够把自己的方法固定下来,而且真题的阅读错误率能够降低并且稳定下来。\\
在离考试一个月里,做阅读题的话,主要是为了保持语感。\\ 
没有必要狂做,做每一道题都需要时间。 好好把握方法,多积累,充分利用做过的每一道题就够了。

\subsection{关于模拟题}
模拟题的质量比起真题差的不是一点半点,而且方法完全不通,很可能会误导自己对阅读的把握。\\
在现在这个阶段,不要再花宝贵的时间在这些垃圾题上了。 有时间,多把长难句看看,多读读。 多花点时间在真题的分析上,事半功倍。

\subsection{关于新题型}
不要把新题型作为自己的杀手锏去花太多的时间。 \\
不管是什么题型,最终考察的肯定是跟传统阅读题一样的方面:英语阅读的语感和单词的积累。只是换了一种呈现的方式而已。\\
不要有太多顾虑,对于这种题型,一般大家的发挥都比较平均。 \\
那些模拟题里面的所谓新题型根本不可信,只会浪费时间和误导自己的感觉。 把大部分时间回到传统的阅读真题的复习上,以不变应万变,在最后的两周看看往年的真题里面的新题型就够了。

\section{阅读题规律}
\subsection{做题时间一般在18min左右}
\subsection{内容定位}
在做题目的时候,需要快速定位,然后根据相关的内容进行分析,得到最终结果。

\paragraph{一般问题的顺序与文章内容的顺序一致}
在做题时,需要在文章中寻找依据,一般按照文章的顺序定位。 如第一道题的选项在第一段,那么第二道题的依据应该在第二段或者以后的地方。 利用这个规律进行宏观的定位。

\paragraph{可以按照特征进行定位}
\begin{enumerate}
    \item 如人名,数字,年份,引语,各种大写字母等 
    \item 特殊标点符号:冒号、引号、破折号等
\end{enumerate}

\paragraph{通过连接词快速判断前后内容的承接}
如:however, but, yet, in contrast, by comparison, whereas, because, therefore, so, hence, etc.

\paragraph{注意程度词、语气词}
语气太过绝对的选项更有可能是错误的,需要多与原文中的关键句比对一下,如must,totally等.

\subsection{内容把握}
英文,一般每段第一句概括这一段的内容;如果把每一段首句连接起来读一下,就是本篇文章的概括。这一点在最后一道选择里面会用到很多。

\section{题型及相应方法}
\subsection{细节题}
\paragraph{解题方法}
\begin{enumerate}
    \item 利用上面介绍的方法快速定位到其对应的关键句子,每个词进行分析
    \item 选项中含有绝对语气词的需要特别注意,一般不是答案。 如: must, always, never, only, completely
    \item 一定要尊重原文
\end{enumerate}

\paragraph{干扰项的特征}

\begin{enumerate}
    \item 与原文内容相似,但是太绝对化,或者意思完全不同
    \item 原文中根本没有提到,但是自己第一眼看上去觉得很有道理
    \item 与原文逻辑关系颠倒
    \item 范围或者缩小,或者夸张扩大
\end{enumerate}

\subsection{推理题}
\paragraph{提问方式}

\begin{enumerate}
    \item From the passage we can draw the conclusion that \underline{\hbox to 30mm{}} 
    \item It is implied/indicated/suggested that \underline{\hbox to 30mm{}} 
    \item What's implied but not stated by the author is that \underline{\hbox to 30mm{}} 
\end{enumerate}

\paragraph{解题方法}
\begin{enumerate}
    \item 搞清楚问的意思,准确定位关键句,进行推理判断
    \item 如果需要全局的总结推理,可以把首段末段,包括每一段的首句连接起来看看
    \item 太绝对的选项更有可能是错误的,正确的选项总是留有余地,如:may, probably, sometimes, often, etc.
    \item 推理过头,概括过度的选项是无关选项
    \item 选项中符合一般常识,富有哲理,属于普遍现象的往往是选项
    \item 文中未加修改的句子和文中直接陈述事实的一般不是选项(不具有概括性和推理性)
\end{enumerate}

\subsection{主旨句}
包括中心思想题和最好标题
\paragraph{主要形式}
\begin{enumerate}
    \item What is the main idea/subject of this passage?
    \item Which of the following is the best title for the passage?
    \item In this passage the author mainly argues that \underline{\hbox to 30mm{}} 
\end{enumerate}
\paragraph{解题思路}
题目的主旨分布规律
\begin{enumerate}
    \item 首段和尾段
    \item 特殊标点符号处,特别是首段的特殊标点处
    \item 语义转折处
    \item 因果句,如关系词:because, due to, owing to, since , for, as, therefore, consequently, result in, originate from etc.
\end{enumerate}

\paragraph{解题方法}
\begin{enumerate}
    \item 首尾段最重要, 每一段的首句就概括了这一段内容
    \item 关注出现频率高的单词
    \item 概括全文,内容全面的选项一般正确
    \item 干扰项的特点:只有局部信息,概括太宽,无关信息
\end{enumerate}

\section{语义题}
要求推测某个生词在特定语境里的含义
\paragraph{常见问题形式}
\begin{enumerate}
    \item According to the author, the word "X" means \underline{\hbox to 30mm{}} 
    \item The term "XX" in para... can be best replaced by \underline{\hbox to 30mm{}} 
\end{enumerate}

\paragraph{解题思路}
\begin{enumerate}
    \item 答案与文章主题关系相近,符合主题的,一般就是答案
    \item 通过连接词,判断其与并列的相关词的关系,如 and, or, but, i.e.
    \item 通过特殊标点符号的关系判断含义(类似于连接词)
    \item 选项与被考词在含以上肤浅相近的一般不是答案(缺少语境的支持)
\end{enumerate}

\subsection{态度题}
考察作者对文中某一事物的态度及观点
\paragraph{常见问题形式}
\begin{enumerate}
    \item What's the writer's attitude to ...?
    \item In the author's option, \underline{\hbox to 30mm{}} 
\end{enumerate}
\paragraph{解题思路}
\begin{enumerate}
    \item 分清文章文体,议论文的中心句往往暗示作者态度
    \item 理解中心句的基础上判断
    \item 区分文章中作者的观点和引文的观点
    \item 自己积累一定的表示感情色彩的词
    \item 如表达褒义的词:positive, supporting, praising, optimistic, admiring, interesting, humorous, enthusiastic, pleasant, concerned, sober
\end{enumerate}

\subsection{排序题及匹配题}
\paragraph{解题思路}
\begin{enumerate}
    \item 通读首段和其余各段的首句和尾句,划出关键词(冠词、指代词、名词、显示逻辑关系的词)
    \item 确定各段首尾句关系,做一题读一段,先慢后快
    \item 检查全文的连贯性和一致性
\end{enumerate}
\paragraph{解题三原则}
\begin{enumerate}
    \item 关注首段尾段,确定文章的主旨
    \item 关注段落间的内部逻辑关系
    \item 关注首段处的关键词、同义词和信号词
\end{enumerate}


\section{做题步骤}
\subsection{第一步,看题,找出关键点}
做题的时候,一种常见效率较低的情况是,先看文章内容,接着看题。 由于对文章的内容的了解比较模糊,看到每一个选项都会觉得有道理。自己在模棱两可的时候,越容易被问题蒙骗。
\paragraph{首先明确:}
\begin {enumerate}
    \item 阅读考察的是综合能力,从前往后,分别为:长难句的快速理解、方法、相关单词的熟练度
    \item 老师会在每个选项里面做手脚,往往看起来都好像是那么回事。要分辨出来,需要的是对问题选项对应部分的内容客观分析
\end {enumerate}

第一步直接看题,主要在不了解题目内容的情况下,最能够客观地对题目及选项的语气做评估。
\begin {enumerate}
    \item 读题目,搞清楚题目的问题是什么,划出题眼,如问选项中哪些说法是对/错的,高清楚是要选True还是False.
    \item 读选项,同样标出关键词,如语气、程度的词汇,注意主语和宾语。 一般会有张冠李戴、夸张或者其他程度错误等等.
    \item 不要选项说是什么就是什么,要带着怀疑的态度,对自己拿不准的词划出来,一般出题的老师就会在这些点上做手脚。 再从题目到文章中,就会清楚很多。
\end {enumerate}

\subsection{第二步,回归文章,找出每个选项对应的部分句子,客观分析}
一般,选项都能够与文章中的内容一一对应,找出对应的那部分内容,细心察看。 特别是对选项中被划出来的关键词,特别关注。
\end{document}


