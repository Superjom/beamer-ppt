\documentclass[a4paper]{ctexart}
\usepackage{geometry}
\geometry{left=2cm,right=2cm,top=2.5cm,bottom=2.5cm}

\author{Chunwei Yan}
\title{英语阅读演示}
\begin{document}
    \maketitle
%---content here----
\section{复习建议}
最后一个阶段,是总结和加强的阶段。
\subsection{长难句}
长难句关系到总体的语感和阅读效率,一定要好好把握,把长难句摆在最重要的位置!
\subsection{单词}
最后阶段,没有必要再以量取胜了,重点把之前背过的复习一下。\\
如果时间比较紧的话,可以重点把往年阅读真题中出现过的单词和变形重点过一下,这个工作很重要。 一般准备的话,长难句和单词就OK了。
\subsection{阅读}
至少在离考试一个月的时候,能够把自己的方法固定下来,而且真题的阅读错误率能够降低并且稳定下来。\\
在离考试一个月里,做阅读题的话,主要是为了保持语感。\\ 
没有必要狂做,做每一道题都需要时间。 好好把握方法,多积累,充分利用做过的每一道题就够了。

\subsection{关于模拟题}
模拟题的质量比起真题差的不是一点半点,而且方法完全不通,很可能会误导自己对阅读的把握。\\
在现在这个阶段,不要再花宝贵的时间在这些垃圾题上了。 有时间,多把长难句看看,多读读。 多花点时间在真题的分析上,事半功倍。

\subsection{关于新题型}
不要把新题型作为自己的杀手锏去花太多的时间。 \\
不管是什么题型,最终考察的肯定是跟传统阅读题一样的方面:英语阅读的语感和单词的积累。只是换了一种呈现的方式而已。\\
不要有太多顾虑,对于这种题型,一般大家的发挥都比较平均。 \\
那些模拟题里面的所谓新题型根本不可信,只会浪费时间和误导自己的感觉。 把大部分时间回到传统的阅读真题的复习上,以不变应万变,在最后的两周看看往年的真题里面的新题型就够了。

\section{阅读题规律}
\subsection{内容定位}
在做题目的时候,需要快速定位,然后根据相关的内容进行分析,得到最终结果。

\paragraph{一般问题的顺序与文章内容的顺序一致}
在做题时,需要在文章中寻找依据,一般按照文章的顺序定位。 如第一道题的选项在第一段,那么第二道题的依据应该在第二段或者以后的地方。 利用这个规律进行宏观的定位。

\paragraph{可以按照特殊符号进行定位}
如人名,或者数字,或者年份等等,都可以用来快速定位。

\subsection{内容把握}
英文,一般每段第一句概括这一段的内容;如果把每一段首句连接起来读一下,就是本篇文章的概括。这一点在最后一道选择里面会用到很多。

\section{做题步骤}

\subsection{第一步,看题,找出关键点}
做题的时候,一种常见效率较低的情况是,先看文章内容,接着看题。 由于对文章的内容的了解比较模糊,看到每一个选项都会觉得有道理。自己在模棱两可的时候,越容易被问题蒙骗。
\paragraph{首先明确:}
\begin {enumerate}
    \item 阅读考察的是综合能力,从前往后,分别为:长难句的快速理解、方法、相关单词的熟练度
    \item 老师会在每个选项里面做手脚,往往看起来都好像是那么回事。要分辨出来,需要的是对问题选项对应部分的内容客观分析
\end {enumerate}

第一步直接看题,主要在不了解题目内容的情况下,最能够客观地对题目及选项的语气做评估。
\begin {enumerate}
    \item 读题目,搞清楚题目的问题是什么,划出题眼,如问选项中哪些说法是对/错的,高清楚是要选True还是False.
    \item 读选项,同样标出关键词,如语气、程度的词汇,注意主语和宾语。 一般会有张冠李戴、夸张或者其他程度错误等等.
    \item 不要选项说是什么就是什么,要带着怀疑的态度,对自己拿不准的词划出来,一般出题的老师就会在这些点上做手脚。 再从题目到文章中,就会清楚很多。
\end {enumerate}

\subsection{第二步,回归文章,找出每个选项对应的部分句子,客观分析}
一般,选项都能够与文章中的内容一一对应,找出对应的那部分内容,细心察看。 特别是对选项中被划出来的关键词,特别关注。

\section{Text 1}

\subsection{原题}
While still catching-up to men in some spheres of modern life, women appear to be way ahead in at least one undesirable category. “Women are particularly susceptible to developing depression and anxiety disorders in response to stress compared to men,” according to Dr. Yehuda, chief psychiatrist at New York’s Veteran’s Administration Hospital.
Studies of both animals and humans have shown that sex hormones somehow affect the stress response, causing females under stress to produce more of the trigger chemicals than do males under the same conditions. In several of the studies, when stressed-out female rats had their ovaries (the female reproductive organs) removed, their chemical responses became equal to those of the males.
\par
Adding to a woman’s increased dose of stress chemicals, are her increased “opportunities” for stress. “It’s not necessarily that women don’t cope as well. It’s just that they have so much more to cope with,” says Dr. Yehuda. “Their capacity for tolerating stress may even be greater than men’s,” she observes, “it’s just that they’re dealing with so many more things that they become worn out from it more visibly and sooner.”
\par
Dr. Yehuda notes another difference between the sexes. “I think that the kinds of things that women are exposed to tend to be in more of a chronic or repeated nature. Men go to war and are exposed to combat stress. Men are exposed to more acts of random physical violence. The kinds of interpersonal violence that women are exposed to tend to be in domestic situations, by, unfortunately, parents or other family members, and they tend not to be one-shot deals. The wear-and-tear that comes from these longer relationships can be quite devastating.”
\par
Adeline Alvarez married at 18 and gave birth to a son, but was determined to finish college. “I struggled a lot to get the college degree. I was living in so much frustration that that was my escape, to go to school, and get ahead and do better.” Later, her marriage ended and she became a single mother. “It’s the hardest thing to take care of a teenager, have a job, pay the rent, pay the car payment, and pay the debt. I lived from paycheck to paycheck.”
\par
Not everyone experiences the kinds of severe chronic stresses Alvarez describes. But most women today are coping with a lot of obligations, with few breaks, and feeling the strain. Alvarez’s experience demonstrates the importance of finding ways to diffuse stress before it threatens your health and your ability to function.
\\
\\
21.	Which of the following is true according to the first two paragraphs?\\
$[A]$ Women are biologically more vulnerable to stress.\\
$[B]$ Women are still suffering much stress caused by men.\\
$[C]$ Women are more experienced than men in coping with stress.\\
$[D]$ Men and women show different inclinations when faced with stress.\\
\\
22.	Dr. Yehuda’s research suggests that women\\
$[A]$ need extra doses of chemicals to handle stress.\\
$[B]$ have limited capacity for tolerating stress.\\
$[C]$ are more capable of avoiding stress.\\
$[D]$ are exposed to more stress.\\
\\
23.	According to Paragraph 4, the stress women confront tends to be\\
$[A]$ domestic and temporary.\\
$[B]$ irregular and violent.\\
$[C]$ durable and frequent.\\
$[D]$ trivial and random.\\
\\
24.	The sentence “I lived from paycheck to paycheck.” (Line 6, Para. 5) shows that\\
$[A]$ Alvarez cared about nothing but making money.\\
$[B]$ Alvarez’s salary barely covered her household expenses.\\
$[C]$ Alvarez got paychecks from different jobs.\\
$[D]$ Alvarez paid practically everything by check.\\
\\
25.	Which of the following would be the best title for the text?\\
$[A]$ Strain of Stress: No Way Out?\\
$[B]$ Responses to Stress: Gender Difference\\
$[C]$ Stress Analysis: What Chemicals Say\\
$[D]$ Gender Inequality: Women Under Stress\\



\subsection{第一步,看问题}

21.	Which of the following is true according to the first two paragraphs?\\
$[A]$ Women are biologically more vulnerable to stress.\\
$[B]$ Women are still suffering much stress caused by men.\\
$[C]$ Women are more experienced than men in coping with stress.\\
$[D]$ Men and women show different inclinations when faced with stress.\\
\\
22.	Dr. Yehuda’s research suggests that women\\
$[A]$ need extra doses of chemicals to handle stress.\\
$[B]$ have limited capacity for tolerating stress.\\
$[C]$ are more capable of avoiding stress.\\
$[D]$ are exposed to more stress.\\
\\
23.	According to Paragraph 4, the stress women confront tends to be\\
$[A]$ domestic and temporary.\\
$[B]$ irregular and violent.\\
$[C]$ durable and frequent.\\
$[D]$ trivial and random.\\
\\
24.	The sentence “I lived from paycheck to paycheck.” (Line 6, Para. 5) shows that\\
$[A]$ Alvarez cared about nothing but making money.\\
$[B]$ Alvarez’s salary barely covered her household expenses.\\
$[C]$ Alvarez got paychecks from different jobs.\\
$[D]$ Alvarez paid practically everything by check.\\
\\
25.	Which of the following would be the best title for the text?\\
$[A]$ Strain of Stress: No Way Out?\\
$[B]$ Responses to Stress: Gender Difference\\
$[C]$ Stress Analysis: What Chemicals Say\\
$[D]$ Gender Inequality: Women Under Stress\\

\subsection{第二步,对应问题,从原文中查找依据的句子}

While still catching-up to men in some spheres of modern life, women appear to be way ahead in at least one undesirable category. “Women are particularly susceptible to developing depression and anxiety disorders in response to stress compared to men,” according to Dr. Yehuda, chief psychiatrist at New York’s Veteran’s Administration Hospital.
Studies of both animals and humans have shown that sex hormones somehow affect the stress response, causing females under stress to produce more of the trigger chemicals than do males under the same conditions. In several of the studies, when stressed-out female rats had their ovaries (the female reproductive organs) removed, their chemical responses became equal to those of the males.
\par
Adding to a woman’s increased dose of stress chemicals, are her increased “opportunities” for stress. “It’s not necessarily that women don’t cope as well. It’s just that they have so much more to cope with,” says Dr. Yehuda. “Their capacity for tolerating stress may even be greater than men’s,” she observes, “it’s just that they’re dealing with so many more things that they become worn out from it more visibly and sooner.”
\par
Dr. Yehuda notes another difference between the sexes. “I think that the kinds of things that women are exposed to tend to be in more of a chronic or repeated nature. Men go to war and are exposed to combat stress. Men are exposed to more acts of random physical violence. The kinds of interpersonal violence that women are exposed to tend to be in domestic situations, by, unfortunately, parents or other family members, and they tend not to be one-shot deals. The wear-and-tear that comes from these longer relationships can be quite devastating.”
\par
Adeline Alvarez married at 18 and gave birth to a son, but was determined to finish college. “I struggled a lot to get the college degree. I was living in so much frustration that that was my escape, to go to school, and get ahead and do better.” Later, her marriage ended and she became a single mother. “It’s the hardest thing to take care of a teenager, have a job, pay the rent, pay the car payment, and pay the debt. I lived from paycheck to paycheck.”
\par
Not everyone experiences the kinds of severe chronic stresses Alvarez describes. But most women today are coping with a lot of obligations, with few breaks, and feeling the strain. Alvarez’s experience demonstrates the importance of finding ways to diffuse stress before it threatens your health and your ability to function.
\\


\section{Text 2}
It used to be so straightforward. A team of researchers working together in the laboratory would submit the results of their research to a journal. A journal editor would then remove the authors’ names and affiliations from the paper and send it to their peers for review. Depending on the comments received, the editor would accept the paper for publication or decline it. Copyright rested with the journal publisher, and researchers seeking knowledge of the results would have to subscribe to the journal.
\par
No longer. The Internet – and pressure from funding agencies, who are questioning why commercial publishers are making money from government-funded research by restricting access to it – is making access to scientific results a reality. The Organization for Economic Co-operation and Development (OECD) has just issued a report describing the far-reaching consequences of this. The report, by John Houghton of Victoria University in Australia and Graham Vickery of the OECD, makes heavy reading for publishers who have, so far, made handsome profits. But it goes further than that. It signals a change in what has, until now, been a key element of scientific endeavor.
\par
The value of knowledge and the return on the public investment in research depends, in part, upon wide distribution and ready access. It is big business. In America, the core scientific publishing market is estimated at between \$7 billion and \$11 billion. The International Association of Scientific, Technical and Medical Publishers says that there are more than 2,000 publishers worldwide specializing in these subjects. They publish more than 1.2 million articles each year in some 16,000 journals.
\par
This is now changing. According to the OECD report, some 75\% of scholarly journals are now online. Entirely new business models are emerging; three main ones were identified by the report’s authors. There is the so-called big deal, where institutional subscribers pay for access to a collection of online journal titles through site-licensing agreements. There is open-access publishing, typically supported by asking the author (or his employer) to pay for the paper to be published. Finally, there are open-access archives, where organizations such as universities or international laboratories support institutional repositories. Other models exist that are hybrids of these three, such as delayed open-access, where journals allow only subscribers to read a paper for the first six months, before making it freely available to everyone who wishes to see it. All this could change the traditional form of the peer-review process, at least for the publication of papers.
\\
\\
26.	In the first paragraph, the author discusses\\
$[A]$ the background information of journal editing.\\
$[B]$ the publication routine of laboratory reports.\\
$[C]$ the relations of authors with journal publishers.\\
$[D]$ the traditional process of journal publication.\\
\\
27.	Which of the following is true of the OECD report?\\
$[A]$ It criticizes government-funded research.\\
$[B]$ It introduces an effective means of publication.\\
$[C]$ It upsets profit-making journal publishers.\\
$[D]$ It benefits scientific research considerably.\\
\\
28.	According to the text, online publication is significant in that\\
$[A]$ it provides an easier access to scientific results.\\
$[B]$ it brings huge profits to scientific researchers.\\
$[C]$ it emphasizes the crucial role of scientific knowledge.\\
$[D]$ it facilitates public investment in scientific research.\\
\\
29.	With the open-access publishing model, the author of a paper is required to\\
$[A]$ cover the cost of its publication.\\
$[B]$ subscribe to the journal publishing it.\\
$[C]$ allow other online journals to use it freely.\\
$[D]$ complete the peer-review before submission.\\
\\
30.	Which of the following best summarizes the text?\\
$[A]$ The Internet is posing a threat to publishers.\\
$[B]$ A new mode of publication is emerging.\\
$[C]$ Authors welcome the new channel for publication.\\
$[D]$ Publication is rendered easier by online service.\\

\section{Text 3}
In the early 1960s Wilt Chamberlain was one of only three players in the National Basketball Association (NBA) listed at over seven feet. If he had played last season, however, he would have been one of 42. The bodies playing major professional sports have changed dramatically over the years, and managers have been more than willing to adjust team uniforms to fit the growing numbers of bigger, longer frames.
\par
The trend in sports, though, may be obscuring an unrecognized reality: Americans have generally stopped growing. Though typically about two inches taller now than 140 years ago, today’s people – especially those born to families who have lived in the U.S. for many generations – apparently reached their limit in the early 1960s. And they aren’t likely to get any taller. “In the general population today, at this genetic, environmental level, we’ve pretty much gone as far as we can go,” says anthropologist William Cameron Chumlea of Wright State University. In the case of NBA players, their increase in height appears to result from the increasingly common practice of recruiting players from all over the world.
\par
Growth, which rarely continues beyond the age of 20, demands calories and nutrients – notably, protein – to feed expanding tissues. At the start of the 20th century, under-nutrition and childhood infections got in the way. But as diet and health improved, children and adolescents have, on average, increased in height by about an inch and a half every 20 years, a pattern known as the secular trend in height. Yet according to the Centers for Disease Control and Prevention, average height – 5′9″ for men, 5′4″ for women – hasn’t really changed since 1960.
\par
Genetically speaking, there are advantages to avoiding substantial height. During childbirth, larger babies have more difficulty passing through the birth canal. Moreover, even though humans have been upright for millions of years, our feet and back continue to struggle with bipedal posture and cannot easily withstand repeated strain imposed by oversize limbs. “There are some real constraints that are set by the genetic architecture of the individual organism,” says anthropologist William Leonard of Northwestern University.
\par
Genetically speaking, there are advantages to avoiding substantial height. During childbirth, larger babies have more difficulty passing through the birth canal. Moreover, even though humans have been upright for millions of years, our feet and back continue to struggle with bipedal posture and cannot easily withstand repeated strain imposed by oversize limbs. “There are some real constraints that are set by the genetic architecture of the individual organism,” says anthropologist William Leonard of Northwestern University.
\par
Genetic maximums can change, but don’t expect this to happen soon. Claire C. Gordon, senior anthropologist at the Army Research Center in Natick, Mass., ensures that 90 percent of the uniforms and workstations fit recruits without alteration. She says that, unlike those for basketball, the length of military uniforms has not changed for some time. And if you need to predict human height in the near future to design a piece of equipment, Gordon says that by and large, “you could use today’s data and feel fairly confident.”
\\\\
31.	Wilt Chamberlain is cited as an example to\\
$[A]$ illustrate the change of height of NBA players.\\
$[B]$ show the popularity of NBA players in the U.S..\\
$[C]$ compare different generations of NBA players.\\
$[D]$ assess the achievements of famous NBA players.\\
\\
32.	Which of the following plays a key role in body growth according to the text?\\
$[A]$ Genetic modification.\\
$[B]$ Natural environment.\\
$[C]$ Living standards.\\
$[D]$ Daily exercise.\\
\\
33.	On which of the following statements would the author most probably agree?\\
$[A]$ Non-Americans add to the average height of the nation.\\
$[B]$ Human height is conditioned by the upright posture.\\
$[C]$ Americans are the tallest on average in the world.\\
$[D]$ Larger babies tend to become taller in adulthood.\\
\\
34.	We learn from the last paragraph that in the near future\\
$[A]$ the garment industry will reconsider the uniform size.\\
$[B]$ the design of military uniforms will remain unchanged.\\
$[C]$ genetic testing will be employed in selecting sportsmen.\\
$[D]$ the existing data of human height will still be applicable.\\
\\
35.	The text intends to tell us that\\
$[A]$ the change of human height follows a cyclic pattern.\\
$[B]$ human height is becoming even more predictable.\\
$[C]$ Americans have reached their genetic growth limit.\\
$[D]$ the genetic pattern of Americans has altered.\\

\section{Text 4}
In 1784, five years before he became president of the United States, George Washington, 52, was nearly toothless. So he hired a dentist to transplant nine teeth into his jaw – having extracted them from the mouths of his slaves.
\par
That’s a far different image from the cherry-tree-chopping George most people remember from their history books. But recently, many historians have begun to focus on the roles slavery played in the lives of the founding generation. They have been spurred in part by DNA evidence made available in 1998, which almost certainly proved Thomas Jefferson had fathered at least one child with his slave Sally Hemings. And only over the past 30 years have scholars examined history from the bottom up. Works of several historians reveal the moral compromises made by the nation’s early leaders and the fragile nature of the country’s infancy. More significantly, they argue that many of the Founding Fathers knew slavery was wrong – and yet most did little to fight it.
\par
More than anything, the historians say, the founders were hampered by the culture of their time. While Washington and Jefferson privately expressed distaste for slavery, they also understood that it was part of the political and economic bedrock of the country they helped to create.
\par
For one thing, the South could not afford to part with its slaves. Owning slaves was “like having a large bank account,” says Wiencek, author of An Imperfect God: George Washington, His Slaves, and the Creation of America. The southern states would not have signed the Constitution without protections for the “peculiar institution,” including a clause that counted a slave as three fifths of a man for purposes of congressional representation.
\par
And the statesmen’s political lives depended on slavery. The three-fifths formula handed Jefferson his narrow victory in the presidential election of 1800 by inflating the votes of the southern states in the Electoral College. Once in office, Jefferson extended slavery with the Louisiana Purchase in 1803; the new land was carved into 13 states, including three slave states.
\par
Still, Jefferson freed Hemings’s children – though not Hemings herself or his approximately 150 other slaves. Washington, who had begun to believe that all men were created equal after observing the bravery of the black soldiers during the Revolutionary War, overcame the strong opposition of his relatives to grant his slaves their freedom in his will. Only a decade earlier, such an act would have required legislative approval in Virginia.\\
\\
36.	George Washington’s dental surgery is mentioned to\\
$[A]$ show the primitive medical practice in the past.\\
$[B]$ demonstrate the cruelty of slavery in his days.\\
$[C]$ stress the role of slaves in the U.S. history.\\
$[D]$ reveal some unknown aspect of his life.\\
\\
37.	We may infer from the second paragraph that\\
$[A]$ DNA technology has been widely applied to history research.\\
$[B]$ in its early days the U.S. was confronted with delicate situations.\\
$[C]$ historians deliberately made up some stories of Jefferson’s life.\\
$[D]$ political compromises are easily found throughout the U.S. history.\\
\\
38.	What do we learn about Thomas Jefferson?\\
$[A]$ His political view changed his attitude towards slavery.\\
$[B]$ His status as a father made him free the child slaves.\\
$[C]$ His attitude towards slavery was complex.\\
$[D]$ His affair with a slave stained his prestige.\\
\\
39.	Which of the following is true according to the text?\\
$[A]$ Some Founding Fathers benefit politically from slavery.\\
$[B]$ Slaves in the old days did not have the right to vote.\\
$[C]$ Slave owners usually had large savings accounts.\\
$[D]$ Slavery was regarded as a peculiar institution.\\
\\
40.	Washington’s decision to free slaves originated from his\\
$[A]$ moral considerations.\\
$[B]$ military experience.\\
$[C]$ financial conditions.\\
$[D]$ political stand.\\



\end{document}
