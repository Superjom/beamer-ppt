
\documentclass[a4paper]{article} 
\usepackage{geometry}
\geometry{left=2.5cm,right=2.5cm,top=2.5cm,bottom=2.5cm}
\usepackage{color}
\usepackage{titlesec}


\pagestyle{headings}

\titleformat*{\section}{\centering\large\bf}
\titleformat*{\subsection}{\centering\large\bf}

\author{Chunwei Yan}
        
\title{英语真题阅读}
        
\usepackage{xeCJK}
\setCJKmainfont{WenQuanYi Micro Hei Mono}
\begin{document}
\maketitle
\tableofcontents
\setcounter{tocdepth}{4}

\section{Part 1}
\subsection{1993年}
\subsubsection{Text1}

\par
Is language, like food, a basic human need without which a child at a critical period of life can be starved and damaged? Judging from the drastic experiment of Frederick II in the thirteenth century, it may be. Hoping to discover what language a child would speak if he heard no mother tongue, he told the nurses to keep silent.

\par
All the infants died before the first year. But clearly there was more than lack of language here. What was missing was good mothering. Without good mothering, in the first year of life especially, the capacity to survive is seriously affected.

\par
Today no such severe lack exists as that ordered by Frederick. Nevertheless, some children are still backward in speaking. Most often the reason for this is that the mother is insensitive to the signals of the infant, whose brain is programmed to learn language rapidly. If these sensitive periods are neglected, the ideal time for acquiring skills passes and they might never be learned so easily again. A bird learns to sing and to fly rapidly at the right time, but the process is slow and hard once the critical stage has passed.

\par
Experts suggest that speech stages are reached in a fixed sequence and at a constant age, but there are cases where speech has started late in a child who eventually turns out to be of high IQ. At twelve weeks a baby smiles and makes vowel-like sounds; at twelve months he can speak simple words and understand simple commands; at eighteen months he has a vocabulary of three to fifty words. At three he knows about 1,000 words which he can put into sentences, and at four his language differs from that of his parents in style rather than grammar.

\par
Recent evidence suggests that an infant is born with the capacity to speak. What is special about man’s brain, compared with that of the monkey, is the complex system which enables a child to connect the sight and feel of, say, a toy-bear with the sound pattern “toy-bear.” And even more incredible is the young brain’s ability to pick out an order in language from the mixture of sound around him, to analyze, to combine and recombine the parts of a language in new ways.

\par
But speech has to be induced, and this depends on interaction between the mother and the child, where the mother recognizes the signals in the child’s babbling (咿呀学语), grasping and smiling, and responds to them. Insensitivity of the mother to these signals dulls the interaction because the child gets discouraged and sends out only the obvious signals. Sensitivity to the child’s non-verbal signals is essential to the growth and development of language.
\\31.	The purpose of Frederick II’s experiment was \underline{\hbox to 30mm{}}.\\$[A]$ to prove that children are born with the ability to speak\\$[B]$ to discover what language a child would speak without hearing any human speech\\$[C]$ to find out what role careful nursing would play in teaching a child to speak\\$[D]$ to prove that a child could be damaged without learning a language\\\\32.	The reason some children are backward in speaking is most probably that \underline{\hbox to 30mm{}}.\\$[A]$ they are incapable of learning language rapidly\\$[B]$ they are exposed to too much language at once\\$[C]$ their mothers respond inadequately to their attempts to speak\\$[D]$ their mothers are not intelligent enough to help them\\\\33.	What is exceptionally remarkable about a child is that \underline{\hbox to 30mm{}}.\\$[A]$ he is born with the capacity to speak\\$[B]$ he has a brain more complex than an animal’s\\$[C]$ he can produce his own sentences\\$[D]$ he owes his speech ability to good nursing\\\\34.	Which of the following can NOT be inferred from the passage?\\$[A]$ The faculty of speech is inborn in man.\\$[B]$ Encouragement is anything but essential to a child in language learning.\\$[C]$ The child’s brain is highly selective.\\$[D]$ Most children learn their language in definite stages.\\\\35.	If a child starts to speak later than others, he will in future \underline{\hbox to 30mm{}}.\\$[A]$ have a high IQ\\$[B]$ be less intelligent\\$[C]$ be insensitive to verbal signals\\$[D]$ not necessarily be backward\\\subsubsection{Text2}

\par
In general, our society is becoming one of giant enterprises directed by a bureaucratic (官僚主义的) management in which man becomes a small, well-oiled cog in the machinery. The oiling is done with higher wages, well-ventilated factories and piped music, and by psychologists and “human-relations” experts; yet all this oiling does not alter the fact that man has become powerless, that he does not wholeheartedly participate in his work and that he is bored with it. In fact, the blue- and the white-collar workers have become economic puppets who dance to the tune of automated machines and bureaucratic management.

\par
The worker and employee are anxious, not only because they might find themselves out of a job; they are anxious also because they are unable to acquire any real satisfaction or interest in life. They live and die without ever having confronted the fundamental realities of human existence as emotionally and intellectually independent and productive human beings.

\par
Those higher up on the social ladder are no less anxious. Their lives are no less empty than those of their subordinates. They are even more insecure in some respects. They are in a highly competitive race. To be promoted or to fall behind is not a matter of salary but even more a matter of self-respect. When they apply for their first job, they are tested for intelligence as well as for the tight mixture of submissiveness and independence. From that moment on they are tested again and again -- by the psychologists, for whom testing is a big business, and by their superiors, who judge their behavior, sociability, capacity to get along, etc. This constant need to prove that one is as good as or better than one’s fellow-competitor creates constant anxiety and stress, the very causes of unhappiness and illness.

\par
Am I suggesting that we should return to the preindustrial mode of production or to nineteenth-century “free enterprise” capitalism? Certainly not. Problems are never solved by returning to a stage which one has already outgrown. I suggest transforming our social system from a bureaucratically managed industrialism in which maximal production and consumption are ends in themselves into a humanist industrialism in which man and full development of his potentialities -- those of love and of reason -- are the aims of all social arrangements. Production and consumption should serve only as means to this end, and should be prevented from ruling man.
\\36.	By “a well-oiled cog in the machinery” the author intends to render the idea that man is \underline{\hbox to 30mm{}}.\\$[A]$ a necessary part of the society though each individual’s function is negligible\\$[B]$ working in complete harmony with the rest of the society\\$[C]$ an unimportant part in comparison with the rest of the society, though functioning smoothly\\$[D]$ a humble component of the society, especially when working smoothly\\\\37.	The real cause of the anxiety of the workers and employees is that \underline{\hbox to 30mm{}}.\\$[A]$ they are likely to lose their jobs\\$[B]$ they have no genuine satisfaction or interest in life\\$[C]$ they are faced with the fundamental realities of human existence\\$[D]$ they are deprived of their individuality and independence\\\\38.	From the passage we can infer that real happiness of life belongs to those \underline{\hbox to 30mm{}}.\\$[A]$ who are at the bottom of the society\\$[B]$ who are higher up in their social status\\$[C]$ who prove better than their fellow-competitors\\$[D]$ who could keep far away from this competitive world\\\\39.	To solve the present social problems the author suggests that we should \underline{\hbox to 30mm{}}.\\$[A]$ resort to the production mode of our ancestors\\$[B]$ offer higher wages to the workers and employees\\$[C]$ enable man to fully develop his potentialities\\$[D]$ take the fundamental realities for granted\\\\40.	The author’s attitude towards industrialism might best be summarized as one of \underline{\hbox to 30mm{}}.\\$[A]$ approval\\$[B]$ dissatisfaction\\$[C]$ suspicion\\$[D]$ tolerance\\\subsubsection{Text3}

\par
When an invention is made, the inventor has three possible courses of action open to him: he can give the invention to the world by publishing it, keep the idea secret, or patent it.

\par
A granted patent is the result of a bargain struck between an inventor and the state, by which the inventor gets a limited period of monopoly (垄断) and publishes full details of his invention to the public after that period terminates.

\par
Only in the most exceptional circumstances is the lifespan of a patent extended to alter this normal process of events.

\par
The longest extension ever granted was to Georges Valensi; his 1939 patent for color TV receiver circuitry was extended until 1971 because for most of the patent’s normal life there was no colour TV to receive and thus no hope of reward for the invention.

\par
Because a patent remains permanently public after it has terminated, the shelves of the library attached to the patent office contain details of literally millions of ideas that are free for anyone to use and, if older than half a century, sometimes even re-patent. Indeed, patent experts often advise anyone wishing to avoid the high cost of conducting a search through live patents that the one sure way of avoiding violation of any other inventor’s right is to plagiarize a dead patent. Likewise, because publication of an idea in any other form permanently invalidates further patents on that idea, it is traditionally safe to take ideas from other areas of print. Much modern technological advance is based on these presumptions of legal security.

\par
Anyone closely involved in patents and inventions soon learns that most “new” ideas are, in fact, as old as the hills. It is their reduction to commercial practice, either through necessity or dedication, or through the availability of new technology, that makes news and money. The basic patent for the theory of magnetic recording dates back to 1886. Many of the original ideas behind television originate from the late 19th and early 20th century. Even the Volkswagen rear engine car was anticipated by a 1904 patent for a cart with the horse at the rear.
\\41.	The passage is mainly about \underline{\hbox to 30mm{}}.\\$[A]$ an approach to patents\\$[B]$ the application for patents\\$[C]$ the use of patents\\$[D]$ the access to patents\\\\42.	Which of the following is TRUE according to the passage?\\$[A]$ When a patent becomes out of effect, it can be re-patented or extended if necessary.\\$[B]$ It is necessary for an inventor to apply for a patent before he makes his invention public.\\$[C]$ A patent holder must publicize the details of his invention when its legal period is over.\\$[D]$ One can get all the details of a patented invention from a library attached to the patent office.\\\\43.	George Valensi’s patent lasted until 1971 because \underline{\hbox to 30mm{}}.\\$[A]$ nobody would offer any reward for his patent prior to that time\\$[B]$ his patent could not be put to use for an unusually long time\\$[C]$ there were not enough TV stations to provide colour programmes\\$[D]$ the colour TV receiver was not available until that time\\\\44.	The word “plagiarize” (Line 8, Para. 5) most probably means “\underline{\hbox to 30mm{}}.”\\$[A]$ steal and use\\$[B]$ give reward to\\$[C]$ make public\\$[D]$ take and change\\\\45.	From the passage we learn that \underline{\hbox to 30mm{}}.\\$[A]$ an invention will not benefit the inventor unless it is reduced to commercial practice\\$[B]$ products are actually inventions which were made a long time ago\\$[C]$ it is much cheaper to buy an old patent than a new one\\$[D]$ patent experts often recommend patents to others by conducting a search through dead patents\\\subsection{1994年}
\subsubsection{Text5}

\par
Discoveries in science and technology are thought by “untaught minds” to come in blinding flashes or as the result of dramatic accidents. Sir Alexander Fleming did not, as legend would have it, look at the mold (霉) on a piece of cheese and get the idea for penicillin there and then. He experimented with antibacterial substances for nine years before he made his discovery. Inventions and innovations almost always come out of laborious trial and error. Innovation is like soccer; even the best players miss the goal and have their shots blocked much more frequently than they score.

\par
The point is that the players who score most are the ones who take most shots at the goal -- and so it goes with innovation in any field of activity. The prime difference between innovators and others is one of approach. Everybody gets ideas, but innovators work consciously on theirs, and they follow them through until they prove practicable or otherwise. What ordinary people see as fanciful abstractions, professional innovators see as solid possibilities.

\par
“Creative thinking may mean simply the realization that there’s no particular virtue in doing things the way they have always been done,” wrote Rudolph Flesch, a language authority. This accounts for our reaction to seemingly simple innovations like plastic garbage bags and suitcases on wheels that make life more convenient: “How come nobody thought of that before?”

\par
The creative approach begins with the proposition that nothing is as it appears. Innovators will not accept that there is only one way to do anything. Faced with getting from A to B, the average person will automatically set out on the best-known and apparently simplest route. The innovator will search for alternate courses, which may prove easier in the long run and are bound to be more interesting and challenging even if they lead to dead ends.

\par
Highly creative individuals really do march to a different drummer.
\\67.	What does the author probably mean by “untaught mind” in the first paragraph?\\$[A]$ A person ignorant of the hard work involved in experimentation.\\$[B]$ A citizen of a society that restricts personal creativity.\\$[C]$ A person who has had no education.\\$[D]$ An individual who often comes up with new ideas by accident.\\\\68.	According to the author, what distinguishes innovators from non-innovators?\\$[A]$ The variety of ideas they have.\\$[B]$ The intelligence they possess.\\$[C]$ The way they deal with problems.\\$[D]$ The way they present their findings.\\\\69.	The author quotes Rudolph Flesch in Paragraph 3 because \underline{\hbox to 30mm{}}.\\$[A]$ Rudolph Flesch is the best-known expert in the study of human creativity\\$[B]$ the quotation strengthens the assertion that creative individuals look for new ways of doing things\\$[C]$ the reader is familiar with Rudolph Flesch’s point of view\\$[D]$ the quotation adds a new idea to the information previously presented\\\\70.	The phrase “march to a different drummer” (the last line of the passage) suggests that highly creative individuals are \underline{\hbox to 30mm{}}.\\$[A]$ diligent in pursuing their goals\\$[B]$ reluctant to follow common ways of doing things\\$[C]$ devoted to the progress of society\\$[D]$ concerned about the advance of society\\\subsection{1995年}
\subsubsection{Text2}

\par
There are two basic ways to see growth: one as a product, the other as a process. People have generally viewed personal growth as an external result or product that can easily be identified and measured. The worker who gets a promotion, the student whose grades improve, the foreigner who learns a new language -- all these are examples of people who have measurable results to show for their efforts.

\par
By contrast, the process of personal growth is much more difficult to determine, since by definition it is a journey and not the specific signposts or landmarks along the way. The process is not the road itself, but rather the attitudes and feelings people have, their caution or courage, as they encounter new experiences and unexpected obstacles. In this process, the journey never really ends; there are always new ways to experience the world, new ideas to try, new challenges to accept.

\par
In order to grow, to travel new roads, people need to have a willingness to take risks, to confront the unknown, and to accept the possibility that they may “fail” at first. How we see ourselves as we try a new way of being is essential to our ability to grow. Do we perceive ourselves as quick and curious? If so, then we tend to take more chances and to be more open to unfamiliar experiences. Do we think we’re shy and indecisive? Then our sense of timidity can cause us to hesitate, to move slowly, and not to take a step until we know the ground is safe. Do we think we’re slow to adapt to change or that we’re not smart enough to cope with a new challenge? Then we are likely to take a more passive role or not try at all.

\par
These feelings of insecurity and self-doubt are both unavoidable and necessary if we are to change and grow. If we do not confront and overcome these internal fears and doubts, if we protect ourselves too much, then we cease to grow. We become trapped inside a shell of our own making.
\\55.	A person is generally believed to achieve personal growth when \underline{\hbox to 30mm{}}.\\$[A]$ he has given up his smoking habit\\$[B]$ he has made great efforts in his work\\$[C]$ he is keen on leaning anything new\\$[D]$ he has tried to determine where he is on his journey\\\\56.	In the author’s eyes, one who views personal growth as a process would \underline{\hbox to 30mm{}}.\\$[A]$ succeed in climbing up the social ladder\\$[B]$ judge his ability to grow from his own achievements\\$[C]$ face difficulties and take up challenges\\$[D]$ aim high and reach his goal each time\\\\57.	When the author says “a new way of being” (Line 2~3, Para. 3) he is referring to \underline{\hbox to 30mm{}}.\\$[A]$ a new approach to experiencing the world\\$[B]$ a new way of taking risks\\$[C]$ a new method of perceiving ourselves\\$[D]$ a new system of adaptation to change\\\\58.	For personal growth, the author advocates all of the following EXCEPT \underline{\hbox to 30mm{}}.\\$[A]$ curiosity about more chances\\$[B]$ promptness in self-adaptation\\$[C]$ open-mindedness to new experiences\\$[D]$ avoidance of internal fears and doubts\\\subsection{1996年}
\subsubsection{Text1}

\par
Tight-lipped elders used to say, “It’s not what you want in this world, but what you get.”

\par
Psychology teaches that you do get what you want if you know what you want and want the right things.

\par
You can make a mental blueprint of a desire as you would make a blueprint of a house, and each of us is continually making these blueprints in the general routine of everyday living. If we intend to have friends to dinner, we plan the menu, make a shopping list, decide which food to cook first, and such planning is an essential for any type of meal to be served.

\par
Likewise, if you want to find a job, take a sheet of paper, and write a brief account of yourself. In making a blueprint for a job, begin with yourself, for when you know exactly what you have to offer, you can intelligently plan where to sell your services.

\par
This account of yourself is actually a sketch of your working life and should include education, experience and references. Such an account is valuable. It can be referred to in filling out standard application blanks and is extremely helpful in personal interviews. While talking to you, your could-be employer is deciding whether your education, your experience, and other qualifications, will pay him to employ you and your “wares” and abilities must be displayed in an orderly and reasonably connected manner.

\par
When you have carefully prepared a blueprint of your abilities and desires, you have something tangible to sell. Then you are ready to hunt for a job. Get all the possible information about your could-be job. Make inquiries as to the details regarding the job and the firm. Keep your eyes and ears open, and use your own judgment. Spend a certain amount of time each day seeking the employment you wish for, and keep in mind: Securing a job is your job now.
\\51.	What do the elders mean when they say, “It’s not what you want in this world, but what you get.”?\\$[A]$ You’ll certainly get what you want.\\$[B]$ It’s no use dreaming.\\$[C]$ You should be dissatisfied with what you have.\\$[D]$ It’s essential to set a goal for yourself.\\\\52.	A blueprint made before inviting a friend to dinner is used in this passage as \underline{\hbox to 30mm{}}.\\$[A]$ an illustration of how to write an application for a job\\$[B]$ an indication of how to secure a good job\\$[C]$ a guideline for job description\\$[D]$ a principle for job evaluation\\\\53.	According to the passage, one must write an account of himself before starting to find a job because \underline{\hbox to 30mm{}}.\\$[A]$ that is the first step to please the employer\\$[B]$ that is the requirement of the employer\\$[C]$ it enables him to know when to sell his services\\$[D]$ it forces him to become clearly aware of himself\\\\54.	When you have carefully prepared a blueprint of your abilities and desires, you have something \underline{\hbox to 30mm{}}.\\$[A]$ definite to offer\\$[B]$ imaginary to provide\\$[C]$ practical to supply\\$[D]$ desirable to present\\\subsubsection{Text2}

\par
With the start of BBC World Service Television, millions of viewers in Asia and America can now watch the Corporation’s news coverage, as well as listen to it.

\par
And of course in Britain listeners and viewers can tune in to two BBC television channels, five BBC national radio services and dozens of local radio stations. They are brought sport, comedy, drama, music, news and current affairs, education, religion, parliamentary coverage, children’s programmes and films for an annual license fee of £83 per household.

\par
It is a remarkable record, stretching back over 70 years -- yet the BBC’s future is now in doubt. The Corporation will survive as a publicly-funded broadcasting organization, at least for the time being, but its role, its size and its programmes are now the subject of a nation-wide debate in Britain.

\par
The debate was launched by the Government, which invited anyone with an opinion of the BBC -- including ordinary listeners and viewers -- to say what was good or bad about the Corporation, and even whether they thought it was worth keeping. The reason for its inquiry is that the BBC’s royal charter runs out in 1996 and it must decide whether to keep the organization as it is, or to make changes.

\par
Defenders of the Corporation -- of whom there are many -- are fond of quoting the American slogan “If it ain’t broke, don’t fix it.” The BBC “ain’t broke,” they say, by which they mean it is not broken (as distinct from the word ‘broke’, meaning having no money), so why bother to change it?

\par
Yet the BBC will have to change, because the broadcasting world around it is changing. The commercial TV channels – ITV and Channel 4 -- were required by the Thatcher Government’s Broadcasting Act to become more commercial, competing with each other for advertisers, and cutting costs and jobs. But it is the arrival of new satellite channels -- funded partly by advertising and partly by viewers’ subscriptions -- which will bring about the biggest changes in the long term.
\\55.	The world famous BBC now faces \underline{\hbox to 30mm{}}.\\$[A]$ the problem of new coverage\\$[B]$ an uncertain prospect\\$[C]$ inquiries by the general public\\$[D]$ shrinkage of audience\\\\56.	In the passage, which of the following about the BBC is NOT mentioned as the key issue?\\$[A]$ Extension of its TV service to Far East.\\$[B]$ Programmes as the subject of a nation-wide debate.\\$[C]$ Potentials for further international cooperations.\\$[D]$ Its existence as a broadcasting organization.\\\\57.	The BBC’s “royal charter” (Line 4, Paragraph 4) stands for \underline{\hbox to 30mm{}}.\\$[A]$ the financial support from the royal family\\$[B]$ the privileges granted by the Queen\\$[C]$ a contract with the Queen\\$[D]$ a unique relationship with the royal family\\\\58.	The foremost reason why the BBC has to readjust itself is no other than \underline{\hbox to 30mm{}}.\\$[A]$ the emergence of commercial TV channels\\$[B]$ the enforcement of Broadcasting Act by the government\\$[C]$ the urgent necessity to reduce costs and jobs\\$[D]$ the challenge of new satellite channels\\\subsection{1998年}
\subsubsection{Text2}

\par
Well, no gain without pain, they say. But what about pain without gain? Everywhere you go in America, you hear tales of corporate revival. What is harder to establish is whether the productivity revolution that businessmen assume they are presiding over is for real.

\par
The official statistics are mildly discouraging. They show that, if you lump manufacturing and services together, productivity has grown on average by 1.2\% since 1987. That is somewhat faster than the average during the previous decade. And since 1991, productivity has increased by about 2\% a year, which is more than twice the 1978-87 average. The trouble is that part of the recent acceleration is due to the usual rebound that occurs at this point in a business cycle, and so is not conclusive evidence of a revival in the underlying trend. There is, as Robert Rubin, the treasury secretary, says, a “disjunction” between the mass of business anecdote that points to a leap in productivity and the picture reflected by the statistics.

\par
Some of this can be easily explained. New ways of organizing the workplace -- all that re-engineering and downsizing -- are only one contribution to the overall productivity of an economy, which is driven by many other factors such as joint investment in equipment and machinery, new technology, and investment in education and training. Moreover, most of the changes that companies make are intended to keep them profitable, and this need not always mean increasing productivity: switching to new markets or improving quality can matter just as much.

\par
Two other explanations are more speculative. First, some of the business restructuring of recent years may have been ineptly done. Second, even if it was well done, it may have spread much less widely than people suppose.

\par
Leonard Schlesinger, a Harvard academic and former chief executive of Au Bong Pain, a rapidly growing chain of bakery cafes, says that much “re-engineering” has been crude. In many cases, he believes, the loss of revenue has been greater than the reductions in cost. His colleague, Michael Beer, says that far too many companies have applied re-engineering in a mechanistic fashion, chopping out costs without giving sufficient thought to long-term profitability. BBDO’s Al Rosenshine is blunter. He dismisses a lot of the work of re-engineering consultants as mere rubbish -- “the worst sort of ambulance chasing.”
\\55.	According to the author, the American economic situation is \underline{\hbox to 30mm{}}.\\$[A]$ not as good as it seems\\$[B]$ at its turning point\\$[C]$ much better than it seems\\$[D]$ near to complete recovery\\\\56.	The official statistics on productivity growth \underline{\hbox to 30mm{}}.\\$[A]$ exclude the usual rebound in a business cycle\\$[B]$ fall short of businessmen’s anticipation\\$[C]$ meet the expectation of business people\\$[D]$ fail to reflect the true state of economy\\\\57.	The author raises the question “what about pain without gain?” because \underline{\hbox to 30mm{}}.\\$[A]$ he questions the truth of “no gain without pain”\\$[B]$ he does not think the productivity revolution works\\$[C]$ he wonders if the official statistics are misleading\\$[D]$ he has conclusive evidence for the revival of businesses\\\\58.	Which of the following statements is NOT mentioned in the passage?\\$[A]$ Radical reforms are essential for the increase of productivity.\\$[B]$ New ways of organizing workplaces may help to increase productivity.\\$[C]$ The reduction of costs is not a sure way to gain long-term profitability.\\$[D]$ The consultants are a bunch of good-for-nothings.\\\subsubsection{Text3}

\par
Science has long had an uneasy relationship with other aspects of culture. Think of Gallileo’s 17th-century trial for his rebelling belief before the Catholic Church or poet William Blake’s harsh remarks against the mechanistic worldview of Isaac Newton. The schism between science and the humanities has, if anything, deepened in this century.

\par
Until recently, the scientific community was so powerful that it could afford to ignore its critics -- but no longer. As funding for science has declined, scientists have attacked “anti-science” in several books, notably Higher Superstition, by Paul R. Gross, a biologist at the University of Virginia, and Norman Levitt, a mathematician at Rutgers University; and The Demon-Haunted World, by Carl Sagan of Cornell University.

\par
Defenders of science have also voiced their concerns at meetings such as “The Flight from Science and Reason,” held in New York City in 1995, and “Science in the Age of (Mis) information,” which assembled last June near Buffalo.

\par
Anti-science clearly means different things to different people. Gross and Levitt find fault primarily with sociologists, philosophers and other academics who have questioned science’s objectivity. Sagan is more concerned with those who believe in ghosts, creationism and other phenomena that contradict the scientific worldview.

\par
A survey of news stories in 1996 reveals that the anti-science tag has been attached to many other groups as well, from authorities who advocated the elimination of the last remaining stocks of smallpox virus to Republicans who advocated decreased funding for basic research.

\par
Few would dispute that the term applies to the Unabomber, whose manifesto, published in 1995, scorns science and longs for return to a pre-technological utopia. But surely that does not mean environmentalists concerned about uncontrolled industrial growth are anti-science, as an essay in US News \& World Report last May seemed to suggest.

\par
The environmentalists, inevitably, respond to such critics. The true enemies of science, argues Paul Ehrlich of Stanford University, a pioneer of environmental studies, are those who question the evidence supporting global warming, the depletion of the ozone layer and other consequences of industrial growth.

\par
Indeed, some observers fear that the anti-science epithet is in danger of becoming meaningless. “The term ‘anti-science’ can lump together too many, quite different things,” notes Harvard University philosopher Gerald Holton in his 1993 work Science and Anti-Science. “They have in common only one thing that they tend to annoy or threaten those who regard themselves as more enlightened.”
\\59.	The word “schism” (Line 4, Paragraph 1) in the context probably means \underline{\hbox to 30mm{}}.\\$[A]$ confrontation\\$[B]$ dissatisfaction\\$[C]$ separation\\$[D]$ contempt\\\\60.	Paragraphs 2 and 3 are written to \underline{\hbox to 30mm{}}.\\$[A]$ discuss the cause of the decline of science’s power\\$[B]$ show the author’s sympathy with scientists\\$[C]$ explain the way in which science develops\\$[D]$ exemplify the division of science and the humanities\\\\61.	Which of the following is true according to the passage?\\$[A]$ Environmentalists were blamed for anti-science in an essay.\\$[B]$ Politicians are not subject to the labeling of anti-science.\\$[C]$ The “more enlightened” tend to tag others as anti-science.\\$[D]$ Tagging environmentalists as “anti-science” is justifiable.\\\\62.	The author’s attitude toward the issue of “science vs. anti-science” is \underline{\hbox to 30mm{}}.\\$[A]$ impartial\\$[B]$ subjective\\$[C]$ biased\\$[D]$ puzzling\\\subsection{2000年}
\subsubsection{Text1}

\par
A history of long and effortless success can be a dreadful handicap, but, if properly handled, it may become a driving force. When the United States entered just such a glowing period after the end of the Second World War, it had a market eight times larger than any competitor, giving its industries unparalleled economies of scale. Its scientists were the world’s best, its workers the most skilled. America and Americans were prosperous beyond the dreams of the Europeans and Asians whose economies the war had destroyed.

\par
It was inevitable that this primacy should have narrowed as other countries grew richer. Just as inevitably, the retreat from predominance proved painful. By the mid-1980s Americans had found themselves at a loss over their fading industrial competitiveness. Some huge American industries, such as consumer electronics, had shrunk or vanished in the face of foreign competition. By 1987 there was only one American television maker left, Zenith. (Now there is none: Zenith was bought by South Korea’s LG Electronics in July.) Foreign-made cars and textiles were sweeping into the domestic market. America’s machine-tool industry was on the ropes. For a while it looked as though the making of semiconductors, which America had invented and which sat at the heart of the new computer age, was going to be the next casualty.

\par
All of this caused a crisis of confidence. Americans stopped taking prosperity for granted. They began to believe that their way of doing business was failing, and that their incomes would therefore shortly begin to fall as well. The mid-1980s brought one inquiry after another into the causes of America’s industrial decline. Their sometimes sensational findings were filled with warnings about the growing competition from overseas.

\par
How things have changed! In 1995 the United States can look back on five years of solid growth while Japan has been struggling. Few Americans attribute this solely to such obvious causes as a devalued dollar or the turning of the business cycle. Self-doubt has yielded to blind pride. “American industry has changed its structure, has gone on a diet, has learnt to be more quick-witted,” according to Richard Cavanagh, executive dean of Harvard’s Kennedy School of Government. “It makes me proud to be an American just to see how our businesses are improving their productivity,” says Stephen Moore of the Cato Institute, a think-tank in Washington, DC. And William Sahlman of the Harvard Business School believes that people will look back on this period as “a golden age of business management in the United States.”
\\51.	The U.S. achieved its predominance after World War II because \underline{\hbox to 30mm{}}.\\$[A]$ it had made painstaking efforts towards this goal\\$[B]$ its domestic market was eight times larger than before\\$[C]$ the war had destroyed the economies of most potential competitors\\$[D]$ the unparalleled size of its workforce had given an impetus to its economy\\\\52.	The loss of U.S. predominance in the world economy in the 1980s is manifested in the fact that the American \underline{\hbox to 30mm{}}.\\$[A]$ TV industry had withdrawn to its domestic market\\$[B]$ semiconductor industry had been taken over by foreign enterprises\\$[C]$ machine-tool industry had collapsed after suicidal actions\\$[D]$ auto industry had lost part of its domestic market\\\\53.	What can be inferred from the passage?\\$[A]$ It is human nature to shift between self-doubt and blind pride.\\$[B]$ Intense competition may contribute to economic progress.\\$[C]$ The revival of the economy depends on international cooperation.\\$[D]$ A long history of success may pave the way for further development.\\\\54.	The author seems to believe the revival of the U.S. economy in the 1990s can be attributed to the \underline{\hbox to 30mm{}}.\\$[A]$ turning of the business cycle\\$[B]$ restructuring of industry\\$[C]$ improved business management\\$[D]$ success in education\\\subsubsection{Text4}

\par
Aimlessness has hardly been typical of the postwar Japan whose productivity and social harmony are the envy of the United States and Europe. But increasingly the Japanese are seeing a decline of the traditional work-moral values. Ten years ago young people were hardworking and saw their jobs as their primary reason for being, but now Japan has largely fulfilled its economic needs, and young people don’t know where they should go next.

\par
The coming of age of the postwar baby boom and an entry of women into the male-dominated job market have limited the opportunities of teenagers who are already questioning the heavy personal sacrifices involved in climbing Japan’s rigid social ladder to good schools and jobs. In a recent survey, it was found that only 24.5 percent of Japanese students were fully satisfied with school life, compared with 67.2 percent of students in the United States. In addition, far more Japanese workers expressed dissatisfaction with their jobs than did their counterparts in the 10 other countries surveyed.

\par
While often praised by foreigners for its emphasis on the basics, Japanese education tends to stress test taking and mechanical learning over creativity and self-expression. “Those things that do not show up in the test scores -- personality, ability, courage or humanity -- are completely ignored,” says Toshiki Kaifu, chairman of the ruling Liberal Democratic Party’s education committee. “Frustration against this kind of thing leads kids to drop out and run wild.” Last year Japan experienced 2,125 incidents of school violence, including 929 assaults on teachers. Amid the outcry, many conservative leaders are seeking a return to the prewar emphasis on moral education. Last year Mitsuo Setoyama, who was then education minister, raised eyebrows when he argued that liberal reforms introduced by the American occupation authorities after World War II had weakened the “Japanese morality of respect for parents.”

\par
But that may have more to do with Japanese life-styles. “In Japan,” says educator Yoko Muro, “it’s never a question of whether you enjoy your job and your life, but only how much you can endure.” With economic growth has come centralization; fully 76 percent of Japan’s 119 million citizens live in cities where community and the extended family have been abandoned in favor of isolated, two-generation households. Urban Japanese have long endured lengthy commutes (travels to and from work) and crowded living conditions, but as the old group and family values weaken, the discomfort is beginning to tell. In the past decade, the Japanese divorce rate, while still well below that of the United States, has increased by more than 50 percent, and suicides have increased by nearly one-quarter.
\\63.	In the Westerner’s eyes, the postwar Japan was \underline{\hbox to 30mm{}}.\\$[A]$ under aimless development\\$[B]$ a positive example\\$[C]$ a rival to the West\\$[D]$ on the decline\\\\64.	According to the author, what may chiefly be responsible for the moral decline of Japanese society?\\$[A]$ Women’s participation in social activities is limited.\\$[B]$ More workers are dissatisfied with their jobs.\\$[C]$ Excessive emphasis has been placed on the basics.\\$[D]$ The life-style has been influenced by Western values.\\\\65.	Which of the following is true according to the author?\\$[A]$ Japanese education is praised for helping the young climb the social ladder.\\$[B]$ Japanese education is characterized by mechanical learning as well as creativity.\\$[C]$ More stress should be placed on the cultivation of creativity.\\$[D]$ Dropping out leads to frustration against test taking.\\\\66.	The change in Japanese life-style is revealed in the fact that \underline{\hbox to 30mm{}}.\\$[A]$ the young are less tolerant of discomforts in life\\$[B]$ the divorce rate in Japan exceeds that in the U.S.\\$[C]$ the Japanese endure more than ever before\\$[D]$ the Japanese appreciate their present life\\\subsection{2003年}
\subsubsection{Text1}

\par
Wild Bill Donovan would have loved the Internet. The American spymaster who built the Office of Strategic Services in the World War II and later laid the roots for the CIA was fascinated with information. Donovan believed in using whatever tools came to hand in the “great game” of espionage -- spying as a “profession.” These days the Net, which has already re-made such everyday pastimes as buying books and sending mail, is reshaping Donovan’s vocation as well.

\par
The latest revolution isn’t simply a matter of gentlemen reading other gentlemen’s e-mail. That kind of electronic spying has been going on for decades. In the past three or four years, the World Wide Web has given birth to a whole industry of point-and-click spying. The spooks call it “open-source intelligence,” and as the Net grows, it is becoming increasingly influential. In 1995 the CIA held a contest to see who could compile the most data about Burundi. The winner, by a large margin, was a tiny Virginia company called Open Source Solutions, whose clear advantage was its mastery of the electronic world.

\par
Among the firms making the biggest splash in this new world is Straitford, Inc., a private intelligence-analysis firm based in Austin, Texas. Straitford makes money by selling the results of spying (covering nations from Chile to Russia) to corporations like energy-services firm McDermott International. Many of its predictions are available online at www.straitford.com.

\par
Straitford president George Friedman says he sees the online world as a kind of mutually reinforcing tool for both information collection and distribution, a spymaster’s dream. Last week his firm was busy vacuuming up data bits from the far corners of the world and predicting a crisis in Ukraine. “As soon as that report runs, we’ll suddenly get 500 new Internet sign-ups from Ukraine,” says Friedman, a former political science professor. “And we’ll hear back from some of them.” Open-source spying does have its risks, of course, since it can be difficult to tell good information from bad. That’s where Straitford earns its keep.

\par
Friedman relies on a lean staff of 20 in Austin. Several of his staff members have military-intelligence backgrounds. He sees the firm’s outsider status as the key to its success. Straitford’s briefs don’t sound like the usual Washington back-and-forthing, whereby agencies avoid dramatic declarations on the chance they might be wrong. Straitford, says Friedman, takes pride in its independent voice.
\\41.	The emergence of the Net has \underline{\hbox to 30mm{}}.\\$[A]$ received support from fans like Donovan\\$[B]$ remolded the intelligence services\\$[C]$ restored many common pastimes\\$[D]$ revived spying as a profession\\\\42.	Donovan’s story is mentioned in the text to \underline{\hbox to 30mm{}}.\\$[A]$ introduce the topic of online spying\\$[B]$ show how he fought for the U.S.\\$[C]$ give an episode of the information war\\$[D]$ honor his unique services to the CIA\\\\43.	The phrase “making the biggest splash” (Line 1, Paragraph 3) most probably means \underline{\hbox to 30mm{}}.\\$[A]$ causing the biggest trouble\\$[B]$ exerting the greatest effort\\$[C]$ achieving the greatest success\\$[D]$ enjoying the widest popularity\\\\44.	It can be learned from Paragraph 4 that \underline{\hbox to 30mm{}}.\\$[A]$ Straitford’s prediction about Ukraine has proved true\\$[B]$ Straitford guarantees the truthfulness of its information\\$[C]$ Straitford’s business is characterized by unpredictability\\$[D]$ Straitford is able to provide fairly reliable information\\\\45.	Straitford is most proud of its \underline{\hbox to 30mm{}}.\\$[A]$ official status\\$[B]$ nonconformist image\\$[C]$ efficient staff\\$[D]$ military background\\\subsection{2004年}
\subsubsection{Text3}

\par
When it comes to the slowing economy, Ellen Spero isn’t biting her nails just yet. But the 47-year-old manicurist isn’t cutting, filling or polishing as many nails as she’d like to, either. Most of her clients spend \$12 to \$50 weekly, but last month two longtime customers suddenly stopped showing up. Spero blames the softening economy. “I’m a good economic indicator,” she says. “I provide a service that people can do without when they’re concerned about saving some dollars.” So Spero is downscaling, shopping at middle-brow Dillard’s department store near her suburban Cleveland home, instead of Neiman Marcus. “I don’t know if other clients are going to abandon me, too.” she says.

\par
Even before Alan Greenspan’s admission that America’s red-hot economy is cooling, lots of working folks had already seen signs of the slowdown themselves. From car dealerships to Gap outlets, sales have been lagging for months as shoppers temper their spending. For retailers, who last year took in 24 percent of their revenue between Thanksgiving and Christmas, the cautious approach is coming at a crucial time. Already, experts say, holiday sales are off 7 percent from last year’s pace. But don’t sound any alarms just yet. Consumers seem only mildly concerned, not panicked, and many say they remain optimistic about the economy’s long-term prospects, even as they do some modest belt-tightening.

\par
Consumers say they’re not in despair because, despite the dreadful headlines, their own fortunes still feel pretty good. Home prices are holding steady in most regions. In Manhattan, “there’s a new gold rush happening in the \$4 million to \$10 million range, predominantly fed by Wall Street bonuses,” says broker Barbara Corcoran. In San Francisco, prices are still rising even as frenzied overbidding quiets. “Instead of 20 to 30 offers, now maybe you only get two or three,” says John Tealdi, a Bay Area real-estate broker. And most folks still feel pretty comfortable about their ability to find and keep a job.

\par
Many folks see silver linings to this slowdown. Potential home buyers would cheer for lower interest rates. Employers wouldn’t mind a little fewer bubbles in the job market. Many consumers seem to have been influenced by stock-market swings, which investors now view as a necessary ingredient to a sustained boom. Diners might see an upside, too. Getting a table at Manhattan’s hot new Alain Ducasse restaurant used to be impossible. Not anymore. For that, Greenspan \& Co. may still be worth toasting.
\\51.	By “Ellen Spero isn’t biting her nails just yet” (Lines 1-2, Paragraph 1), the author means \underline{\hbox to 30mm{}}.\\$[A]$ Spero can hardly maintain her business\\$[B]$ Spero is too much engaged in her work\\$[C]$ Spero has grown out of her bad habit\\$[D]$ Spero is not in a desperate situation\\\\52.	How do the public feel about the current economic situation?\\$[A]$ Optimistic.\\$[B]$ Confused.\\$[C]$ Carefree.\\$[D]$ Panicked.\\\\53.	When mentioning “the \$4 million to \$10 million range” (Lines 3-4, Paragraph 3) the author is talking about \underline{\hbox to 30mm{}}.\\$[A]$ gold market\\$[B]$ real estate\\$[C]$ stock exchange\\$[D]$ venture investment\\\\54.	Why can many people see “silver linings” to the economic slowdown?\\$[A]$ They would benefit in certain ways.\\$[B]$ The stock market shows signs of recovery.\\$[C]$ Such a slowdown usually precedes a boom.\\$[D]$ The purchasing power would be enhanced.\\\\55.	To which of the following is the author likely to agree?\\$[A]$ A new boom, on the horizon.\\$[B]$ Tighten the belt, the single remedy.\\$[C]$ Caution all right, panic not.\\$[D]$ The more ventures, the more chances.\\\subsection{2005年}
\subsubsection{Text4}

\par
Americans no longer expect public figures, whether in speech or in writing, to command the English language with skill and gift. Nor do they aspire to such command themselves. In his latest book, Doing Our Own Thing: The Degradation of Language and Music and Why We Should, Like, Care, John McWhorter, a linguist and controversialist of mixed liberal and conservative views, sees the triumph of 1960s counter-culture as responsible for the decline of formal English.

\par
Blaming the permissive 1960s is nothing new, but this is not yet another criticism against the decline in education. Mr. McWhorter’s academic speciality is language history and change, and he sees the gradual disappearance of “whom,” for example, to be natural and no more regrettable than the loss of the case-endings of Old English.

\par
But the cult of the authentic and the personal, “doing our own thing,” has spelt the death of formal speech, writing, poetry and music. While even the modestly educated sought an elevated tone when they put pen to paper before the 1960s, even the most well regarded writing since then has sought to capture spoken English on the page. Equally, in poetry, the highly personal, performative genre is the only form that could claim real liveliness. In both oral and written English, talking is triumphing over speaking, spontaneity over craft.

\par
Illustrated with an entertaining array of examples from both high and low culture, the trend that Mr. McWhorter documents is unmistakable. But it is less clear, to take the question of his subtitle, why we should, like, care. As a linguist, he acknowledges that all varieties of human language, including non-standard ones like Black English, can be powerfully expressive -- there exists no language or dialect in the world that cannot convey complex ideas. He is not arguing, as many do, that we can no longer think straight because we do not talk proper.

\par
Russians have a deep love for their own language and carry large chunks of memorized poetry in their heads, while Italian politicians tend to elaborate speech that would seem old-fashioned to most English-speakers. Mr. McWhorter acknowledges that formal language is not strictly necessary, and proposes no radical education reforms -- he is really grieving over the loss of something beautiful more than useful. We now take our English “on paper plates instead of china.” A shame, perhaps, but probably an inevitable one.
\\36.	According to McWhorter, the decline of formal English \underline{\hbox to 30mm{}}.\\$[A]$ is inevitable in radical education reforms\\$[B]$ is but all too natural in language development\\$[C]$ has caused the controversy over the counter-culture\\$[D]$ brought about changes in public attitudes in the 1960s\\\\37.	The word “talking” (Line 6, Paragraph 3) denotes \underline{\hbox to 30mm{}}.\\$[A]$ modesty\\$[B]$ personality\\$[C]$ liveliness\\$[D]$ informality\\\\38.	To which of the following statements would McWhorter most likely agree?\\$[A]$ Logical thinking is not necessarily related to the way we talk.\\$[B]$ Black English can be more expressive than standard English.\\$[C]$ Non-standard varieties of human language are just as entertaining.\\$[D]$ Of all the varieties, standard English can best convey complex ideas.\\\\39.	The description of Russians’ love of memorizing poetry shows the author’s \underline{\hbox to 30mm{}}.\\$[A]$ interest in their language\\$[B]$ appreciation of their efforts\\$[C]$ admiration for their memory\\$[D]$ contempt for their old-fashionedness\\\\40.	According to the last paragraph, “paper plates” is to “china” as \underline{\hbox to 30mm{}}.\\$[A]$ “temporary” is to “permanent”\\$[B]$ “radical” is to “conservative”\\$[C]$ “functional” is to “artistic”\\$[D]$ “humble” is to “noble”\\\section{Part 2}
\subsection{1995年}
\subsubsection{Text5}

\par
That experiences influence subsequent behaviour is evidence of an obvious but nevertheless remarkable activity called remembering. Learning could not occur without the function popularly named memory. Constant practice has such an effect on memory as to lead to skillful performance on the piano, to recitation of a poem, and even to reading and understanding these words. So-called intelligent behaviour demands memory, remembering being a primary requirement for reasoning. The ability to solve any problem or even to recognize that a problem exists depends on memory. Typically, the decision to cross a street is based on remembering many earlier experiences.

\par
Practice (or review) tends to build and maintain memory for a task or for any learned material. Over a period of no practice what has been learned tends to be forgotten; and the adaptive consequences may not seem obvious. Yet, dramatic instances of sudden forgetting can be seen to be adaptive. In this sense, the ability to forget can be interpreted to have survived through a process of natural selection in animals. Indeed, when one’s memory of an emotionally painful experience lead to serious anxiety, forgetting may produce relief. Nevertheless, an evolutionary interpretation might make it difficult to understand how the commonly gradual process of forgetting survived natural selection.

\par
In thinking about the evolution of memory together with all its possible aspects, it is helpful to consider what would happen if memories failed to fade. Forgetting clearly aids orientation in time, since old memories weaken and the new tend to stand out, providing clues for inferring duration. Without forgetting, adaptive ability would suffer, for example, learned behaviour that might have been correct a decade ago may no longer be. Cases are recorded of people who (by ordinary standards) forgot so little that their everyday activities were full of confusion. Thus forgetting seems to serve that survival of the individual and the species.

\par
Another line of thought assumes a memory storage system of limited capacity that provides adaptive flexibility specifically through forgetting. In this view, continual adjustments are made between learning or memory storage (input) and forgetting (output). Indeed, there is evidence that the rate at which individuals forget is directly related to how much they have learned. Such data offer gross support of contemporary models of memory that assume an input-output balance.
\\67.	From the evolutionary point of view, \underline{\hbox to 30mm{}}.\\$[A]$ forgetting for lack of practice tends to be obviously inadaptive\\$[B]$ if a person gets very forgetful all of a sudden he must be very adaptive\\$[C]$ the gradual process of forgetting is an indication of an individual’s adaptability\\$[D]$ sudden forgetting may bring about adaptive consequences\\\\68.	According to the passage, if a person never forgot, \underline{\hbox to 30mm{}}.\\$[A]$ he would survive best\\$[B]$ he would have a lot of trouble\\$[C]$ his ability to learn would be enhanced\\$[D]$ the evolution of memory would stop\\\\69.	From the last paragraph we know that \underline{\hbox to 30mm{}}.\\$[A]$ forgetfulness is a response to learning\\$[B]$ the memory storage system is an exactly balanced input-output system\\$[C]$ memory is a compensation for forgetting\\$[D]$ the capacity of a memory storage system is limited because forgetting occurs\\\\70.	In this article, the author tries to interpret the function of \underline{\hbox to 30mm{}}.\\$[A]$ remembering\\$[B]$ forgetting\\$[C]$ adapting\\$[D]$ experiencing\\\subsection{1996年}
\subsubsection{Text4}

\par
What accounts for the great outburst of major inventions in early America -- breakthroughs such as the telegraph, the steamboat and the weaving machine?

\par
Among the many shaping factors, I would single out the country’s excellent elementary schools: a labor force that welcomed the new technology; the practice of giving premiums to inventors; and above all the American genius for nonverbal, “spatial” thinking about things technological.

\par
Why mention the elementary schools? Because thanks to these schools our early mechanics, especially in the New England and Middle Atlantic states, were generally literate and at home in arithmetic and in some aspects of geometry and trigonometry.

\par
Acute foreign observers related American adaptiveness and inventiveness to this educational advantage. As a member of a British commission visiting here in 1853 reported, “With a mind prepared by thorough school discipline, the American boy develops rapidly into the skilled workman.”

\par
A further stimulus to invention came from the “premium” system, which preceded our patent system and for years ran parallel with it. This approach, originated abroad, offered inventors medals, cash prizes and other incentives.

\par
In the United States, multitudes of premiums for new devices were awarded at country fairs and at the industrial fairs in major cities. Americans flocked to these fairs to admire the new machines and thus to renew their faith in the beneficence of technological advance.

\par
Given this optimistic approach to technological innovation, the American worker took readily to that special kind of nonverbal thinking required in mechanical technology. As Eugene Ferguson has pointed out, “A technologist thinks about objects that cannot be reduced to unambiguous verbal descriptions: they are dealt with in his mind by a visual, nonverbal process... The designer and the inventor... are able to assemble and manipulate in their minds devices that as yet do not exist.”

\par
This nonverbal “spatial” thinking can be just as creative as painting and writing. Robert Fulton once wrote, “The mechanic should sit down among levers, screws, wedges, wheels, etc., like a poet among the letters of the alphabet, considering them as an exhibition of his thoughts, in which a new arrangement transmits a new idea.”

\par
When all these shaping forces -- schools, open attitudes, the premium system, a genius for spatial thinking -- interacted with one another on the rich U.S. mainland, they produced that American characteristic, emulation. Today that word implies mere imitation. But in earlier times it meant a friendly but competitive striving for fame and excellence.
\\63.	According to the author, the great outburst of major inventions in early America was in a large part due to \underline{\hbox to 30mm{}}.\\$[A]$ elementary schools\\$[B]$ enthusiastic workers\\$[C]$ the attractive premium system\\$[D]$ a special way of thinking\\\\64.	It is implied that adaptiveness and inventiveness of the early American mechanics \underline{\hbox to 30mm{}}.\\$[A]$ benefited a lot from their mathematical knowledge\\$[B]$ shed light on disciplined school management\\$[C]$ was brought about by privileged home training\\$[D]$ owed a lot to the technological development\\\\65.	A technologist can be compared to an artist because \underline{\hbox to 30mm{}}.\\$[A]$ they are both winners of awards\\$[B]$ they are both experts in spatial thinking\\$[C]$ they both abandon verbal description\\$[D]$ they both use various instruments\\\\66.	The best title for this passage might be \underline{\hbox to 30mm{}}.\\$[A]$ Inventive Mind\\$[B]$ Effective Schooling\\$[B]$ Ways of Thinking\\$[D]$ Outpouring of Inventions\\\subsubsection{Text5}

\par
Rumor has it that more than 20 books on creationism/evolution are in the publisher’s pipelines. A few have already appeared. The goal of all will be to try to explain to a confused and often unenlightened citizenry that there are not two equally valid scientific theories for the origin and evolution of universe and life. Cosmology, geology, and biology have provided a consistent, unified, and constantly improving account of what happened. “Scientific” creationism, which is being pushed by some for “equal time” in the classrooms whenever the scientific accounts of evolution are given, is based on religion, not science. Virtually all scientists and the majority of non-fundamentalist religious leaders have come to regard “scientific” creationism as bad science and bad religion.

\par
The first four chapters of Kitcher’s book give a very brief introduction to evolution. At appropriate places, he introduces the criticisms of the creationists and provides answers. In the last three chapters, he takes off his gloves and gives the creationists a good beating. He describes their programmes and tactics, and, for those unfamiliar with the ways of creationists, the extent of their deception and distortion may come as an unpleasant surprise. When their basic motivation is religious, one might have expected more Christian behavior.

\par
Kitcher is a philosopher, and this may account, in part, for the clarity and effectiveness of his arguments. The non-specialist will be able to obtain at least a notion of the sorts of data and argument that support evolutionary theory. The final chapter on the creationists will be extremely clear to all. On the dust jacket of this fine book, Stephen Jay Gould says: “This book stands for reason itself.” And so it does -- and all would be well were reason the only judge in the creationism/evolution debate.
\\67.	“Creationism” in the passage refers to \underline{\hbox to 30mm{}}.\\$[A]$ evolution in its true sense as to the origin of the universe\\$[B]$ a notion of the creation of religion\\$[C]$ the scientific explanation of the earth formation\\$[D]$ the deceptive theory about the origin of the universe\\\\68.	Kitcher’s book is intended to \underline{\hbox to 30mm{}}.\\$[A]$ recommend the views of the evolutionists\\$[B]$ expose the true features of creationists\\$[C]$ curse bitterly at this opponents\\$[D]$ launch a surprise attack on creationists\\\\69.	From the passage we can infer that \underline{\hbox to 30mm{}}.\\$[A]$ reasoning has played a decisive role in the debate\\$[B]$ creationists do not base their argument on reasoning\\$[C]$ evolutionary theory is too difficult for non-specialists\\$[D]$ creationism is supported by scientific findings\\\\70.	This passage appears to be a digest of \underline{\hbox to 30mm{}}.\\$[A]$ a book review\\$[B]$ a scientific paper\\$[C]$ a magazine feature\\$[D]$ a newspaper editorial\\\subsection{1997年}
\subsubsection{Text1}

\par
It was 3:45 in the morning when the vote was finally taken. After six months of arguing and final 16 hours of hot parliamentary debates, Australia’s Northern Territory became the first legal authority in the world to allow doctors to take the lives of incurably ill patients who wish to die. The measure passed by the convincing vote of 15 to 10. Almost immediately word flashed on the Internet and was picked up, half a world away, by John Hofsess, executive director of the Right to Die Society of Canada. He sent it on via the group’s on-line service, Death NET. Says Hofsess: “We posted bulletins all day long, because of course this isn’t just something that happened in Australia. It’s world history.”

\par
The full import may take a while to sink in. The NT Rights of the Terminally Ill law has left physicians and citizens alike trying to deal with its moral and practical implications. Some have breathed sighs of relief, others, including churches, right-to-life groups and the Australian Medical Association, bitterly attacked the bill and the haste of its passage. But the tide is unlikely to turn back. In Australia -- where an aging population, life-extending technology and changing community attitudes have all played their part -- other states are going to consider making a similar law to deal with euthanasia. In the US and Canada, where the right-to-die movement is gathering strength, observers are waiting for the dominoes to start falling.

\par
Under the new Northern Territory law, an adult patient can request death -- probably by a deadly injection or pill -- to put an end to suffering. The patient must be diagnosed as terminally ill by two doctors. After a “cooling off” period of seven days, the patient can sign a certificate of request. After 48 hours the wish for death can be met. For Lloyd Nickson, a 54-year-old Darwin resident suffering from lung cancer, the NT Rights of Terminally Ill law means he can get on with living without the haunting fear of his suffering: a terrifying death from his breathing condition. “I’m not afraid of dying from a spiritual point of view, but what I was afraid of was how I’d go, because I’ve watched people die in the hospital fighting for oxygen and clawing at their masks,” he says.
\\51.	From the second paragraph we learn that \underline{\hbox to 30mm{}}.\\$[A]$ the objection to euthanasia is slow to come in other countries\\$[B]$ physicians and citizens share the same view on euthanasia\\$[C]$ changing technology is chiefly responsible for the hasty passage of the law\\$[D]$ it takes time to realize the significance of the law’s passage\\\\52.	When the author says that observers are waiting for the dominoes to start falling, he means \underline{\hbox to 30mm{}}.\\$[A]$ observers are taking a wait-and-see attitude towards the future of euthanasia\\$[B]$ similar bills are likely to be passed in the US, Canada and other countries\\$[C]$ observers are waiting to see the result of the game of dominoes\\$[D]$ the effect-taking process of the passed bill may finally come to a stop\\\\53.	When Lloyd Nickson dies, he will \underline{\hbox to 30mm{}}.\\$[A]$ face his death with calm characteristic of euthanasia\\$[B]$ experience the suffering of a lung cancer patient\\$[C]$ have an intense fear of terrible suffering\\$[D]$ undergo a cooling off period of seven days\\\\54.	The author’s attitude towards euthanasia seems to be that of \underline{\hbox to 30mm{}}.\\$[A]$ opposition\\$[B]$ suspicion\\$[C]$ approval\\$[D]$ indifference\\\subsubsection{Text2}

\par
A report consistently brought back by visitors to the US is how friendly, courteous, and helpful most Americans were to them. To be fair, this observation is also frequently made of Canada and Canadians, and should best be considered North American. There are, of course, exceptions. Small-minded officials, rude waiters, and ill-mannered taxi drivers are hardly unknown in the US. Yet it is an observation made so frequently that it deserves comment.

\par
For a long period of time and in many parts of the country, a traveler was a welcome break in an otherwise dull existence. Dullness and loneliness were common problems of the families who generally lived distant from one another. Strangers and travelers were welcome sources of diversion, and brought news of the outside world.

\par
The harsh realities of the frontier also shaped this tradition of hospitality. Someone traveling alone, if hungry, injured, or ill, often had nowhere to turn except to the nearest cabin or settlement. It was not a matter of choice for the traveler or merely a charitable impulse on the part of the settlers. It reflected the harshness of daily life: if you didn’t take in the stranger and take care of him, there was no one else who would. And someday, remember, you might be in the same situation.

\par
Today there are many charitable organizations which specialize in helping the weary traveler. Yet, the old tradition of hospitality to strangers is still very strong in the US, especially in the smaller cities and towns away from the busy tourist trails. “I was just traveling through, got talking with this American, and pretty soon he invited me home for dinner -- amazing.” Such observations reported by visitors to the US are not uncommon, but are not always understood properly. The casual friendliness of many Americans should be interpreted neither as superficial nor as artificial, but as the result of a historically developed cultural tradition.

\par
As is true of any developed society, in America a complex set of cultural signals, assumptions, and conventions underlies all social interrelationships. And, of course, speaking a language does not necessarily mean that someone understands social and cultural patterns. Visitors who fail to “translate” cultural meanings properly often draw wrong conclusions. For example, when an American uses the word “friend,” the cultural implications of the word may be quite different from those it has in the visitor’s language and culture. It takes more than a brief encounter on a bus to distinguish between courteous convention and individual interest. Yet, being friendly is a virtue that many Americans value highly and expect from both neighbors and strangers.
\\55.	In the eyes of visitors from the outside world, \underline{\hbox to 30mm{}}.\\$[A]$ rude taxi drivers are rarely seen in the US\\$[B]$ small-minded officials deserve a serious comment\\$[C]$ Canadians are not so friendly as their neighbors\\$[D]$ most Americans are ready to offer help\\\\56.	It could be inferred from the last paragraph that \underline{\hbox to 30mm{}}.\\$[A]$ culture exercises an influence over social interrelationship\\$[B]$ courteous convention and individual interest are interrelated\\$[C]$ various virtues manifest themselves exclusively among friends\\$[D]$ social interrelationships equal the complex set of cultural conventions\\\\57.	Families in frontier settlements used to entertain strangers \underline{\hbox to 30mm{}}.\\$[A]$ to improve their hard life\\$[B]$ in view of their long-distance travel\\$[C]$ to add some flavor to their own daily life\\$[D]$ out of a charitable impulse\\\\58.	The tradition of hospitality to strangers \underline{\hbox to 30mm{}}.\\$[A]$ tends to be superficial and artificial\\$[B]$ is generally well kept up in the United States\\$[C]$ is always understood properly\\$[D]$ has something to do with the busy tourist trails\\\subsubsection{Text5}

\par
Much of the language used to describe monetary policy, such as “steering the economy to a soft landing” or “a touch on the brakes,” makes it sound like a precise science. Nothing could be further from the truth. The link between interest rates and inflation is uncertain. And there are long, variable lags before policy changes have any effect on the economy. Hence the analogy that likens the conduct of monetary policy to driving a car with a blackened windscreen, a cracked rear-view mirror and a faulty steering wheel.

\par
Given all these disadvantages, central bankers seem to have had much to boast about of late. Average inflation in the big seven industrial economies fell to a mere 2.3\% last year, close to its lowest level in 30 years, before rising slightly to 2.5\% this July. This is a long way below the double-digit rates which many countries experienced in the 1970s and early 1980s.

\par
It is also less than most forecasters had predicted. In late 1994 the panel of economists which The Economist polls each month said that America’s inflation rate would average 3.5\% in 1995. In fact, it fell to 2.6\% in August, and is expected to average only about 3\% for the year as a whole. In Britain and Japan inflation is running half a percentage point below the rate predicted at the end of last year. This is no flash in the pan; over the past couple of years, inflation has been consistently lower than expected in Britain and America.

\par
Economists have been particularly surprised by favorable inflation figures in Britain and the United States, since conventional measures suggest that both economies, and especially America’s, have little productive slack. America’s capacity utilization, for example, hit historically high levels earlier this year, and its jobless rate (5.6\% in August) has fallen below most estimates of the natural rate of unemployment -- the rate below which inflation has taken off in the past.

\par
Why has inflation proved so mild? The most thrilling explanation is, unfortunately, a little defective. Some economists argue that powerful structural changes in the world have upended the old economic models that were based upon the historical link between growth and inflation.
\\67.	From the passage we learn that \underline{\hbox to 30mm{}}.\\$[A]$ there is a definite relationship between inflation and interest rates\\$[B]$ economy will always follow certain models\\$[C]$ the economic situation is better than expected\\$[D]$ economists had foreseen the present economic situation\\\\68.	According to the passage, which of the following is TRUE?\\$[A]$ Making monetary policies is comparable to driving a car\\$[B]$ An extremely low jobless rate will lead to inflation\\$[C]$ A high unemployment rate will result from inflation\\$[D]$ Interest rates have an immediate effect on the economy\\\\69.	The sentence “This is no flash in the pan” (Line 5, Paragraph 3) means that \underline{\hbox to 30mm{}}.\\$[A]$ the low inflation rate will last for some time\\$[B]$ the inflation rate will soon rise\\$[C]$ the inflation will disappear quickly\\$[D]$ there is no inflation at present\\\\70.	The passage shows that the author is \underline{\hbox to 30mm{}} the present situation.\\$[A]$ critical of\\$[B]$ puzzled by\\$[C]$ disappointed at\\$[D]$ amazed at\\\subsection{1998年}
\subsubsection{Text1}

\par
Few creations of big technology capture the imagination like giant dams. Perhaps it is humankind’s long suffering at the mercy of flood and drought that makes the idea of forcing the waters to do our bidding so fascinating. But to be fascinated is also, sometimes, to be blind. Several giant dam projects threaten to do more harm than good.

\par
The lesson from dams is that big is not always beautiful. It doesn’t help that building a big, powerful dam has become a symbol of achievement for nations and people striving to assert themselves. Egypt’s leadership in the Arab world was cemented by the Aswan High Dam. Turkey’s bid for First World status includes the giant Ataturk Dam.

\par
But big dams tend not to work as intended. The Aswan Dam, for example, stopped the Nile flooding but deprived Egypt of the fertile silt that floods left -- all in return for a giant reservoir of disease which is now so full of silt that it barely generates electricity.

\par
And yet, the myth of controlling the waters persists. This week, in the heart of civilized Europe, Slovaks and Hungarians stopped just short of sending in the troops in their contention over a dam on the Danube. The huge complex will probably have all the usual problems of big dams. But Slovakia is bidding for independence from the Czechs, and now needs a dam to prove itself.

\par
Meanwhile, in India, the World Bank has given the go-ahead to the even more wrong-headed Narmada Dam. And the bank has done this even though its advisors say the dam will cause hardship for the powerless and environmental destruction. The benefits are for the powerful, but they are far from guaranteed.

\par
Proper, scientific study of the impacts of dams and of the cost and benefits of controlling water can help to resolve these conflicts. Hydroelectric power and flood control and irrigation are possible without building monster dams. But when you are dealing with myths, it is hard to be either proper, or scientific. It is time that the world learned the lessons of Aswan. You don’t need a dam to be saved.
\\51.	The third sentence of Paragraph 1 implies that \underline{\hbox to 30mm{}}.\\$[A]$ people would be happy if they shut their eyes to reality\\$[B]$ the blind could be happier than the sighted\\$[C]$ over-excited people tend to neglect vital things\\$[D]$ fascination makes people lose their eyesight\\\\52.	In Paragraph 5, “the powerless” probably refers to \underline{\hbox to 30mm{}}.\\$[A]$ areas short of electricity\\$[B]$ dams without power stations\\$[C]$ poor countries around India\\$[D]$ common people in the Narmada Dam area\\\\53.	What is the myth concerning giant dams?\\$[A]$ They bring in more fertile soil.\\$[B]$ They help defend the country.\\$[C]$ They strengthen international ties.\\$[D]$ They have universal control of the waters.\\\\54.	What the author tries to suggest may best be interpreted as \underline{\hbox to 30mm{}}.\\$[A]$ “It’s no use crying over spilt milk”\\$[B]$ “More haste, less speed”\\$[C]$ “Look before you leap”\\$[D]$ “He who laughs last laughs best”\\\subsubsection{Text4}

\par
Emerging from the 1980 census is the picture of a nation developing more and more regional competition, as population growth in the Northeast and Midwest reaches a near standstill.

\par
This development -- and its strong implications for US politics and economy in years ahead -- has enthroned the South as America’s most densely populated region for the first time in the history of the nation’s head counting.

\par
Altogether, the US population rose in the 1970s by 23.2 million people -- numerically the third-largest growth ever recorded in a single decade. Even so, that gain adds up to only 11.4 percent, lowest in American annual records except for the Depression years.

\par
Americans have been migrating south and west in larger numbers since World War II, and the pattern still prevails.

\par
Three sun-belt states -- Florida, Texas and California -- together had nearly 10 million more people in 1980 than a decade earlier. Among large cities, San Diego moved from 14th to 8th and San Antonio from 15th to 10th -- with Cleveland and Washington. D. C., dropping out of the top 10.

\par
Not all that shift can be attributed to the movement out of the snow belt, census officials say. Nonstop waves of immigrants played a role, too -- and so did bigger crops of babies as yesterday’s “baby boom” generation reached its child-bearing years.

\par
Moreover, demographers see the continuing shift south and west as joined by a related but newer phenomenon: More and more, Americans apparently are looking not just for places with more jobs but with fewer people, too. Some instances—

\par
■Regionally, the Rocky Mountain states reported the most rapid growth rate -- 37.1 percent since 1970 in a vast area with only 5 percent of the US population.

\par
■Among states, Nevada and Arizona grew fastest of all: 63.5 and 53.1 percent respectively. Except for Florida and Texas, the top 10 in rate of growth is composed of Western states with 7.5 million people -- about 9 per square mile.

\par
The flight from overcrowdedness affects the migration from snow belt to more bearable climates.

\par
Nowhere do 1980 census statistics dramatize more the American search for spacious living than in the Far West. There, California added 3.7 million to its population in the 1970s, more than any other state.

\par
In that decade, however, large numbers also migrated from California, mostly to other parts of the West. Often they chose -- and still are choosing -- somewhat colder climates such as Oregon, Idaho and Alaska in order to escape smog, crime and other plagues of urbanization in the Golden State.

\par
As a result, California’s growth rate dropped during the 1970s, to 18.5 percent -- little more than two thirds the 1960s’ growth figure and considerably below that of other Western states.
\\63.	Discerned from the perplexing picture of population growth the 1980 census provided, America in 1970s \underline{\hbox to 30mm{}}.\\$[A]$ enjoyed the lowest net growth of population in history\\$[B]$ witnessed a southwestern shift of population\\$[C]$ underwent an unparalleled period of population growth\\$[D]$ brought to a standstill its pattern of migration since World War II\\\\64.	The census distinguished itself from previous studies on population movement in that \underline{\hbox to 30mm{}}.\\$[A]$ it stresses the climatic influence on population distribution\\$[B]$ it highlights the contribution of continuous waves of immigrants\\$[C]$ it reveals the Americans’ new pursuit of spacious living\\$[D]$ it elaborates the delayed effects of yesterday’s “baby boom”\\\\65.	We can see from the available statistics that \underline{\hbox to 30mm{}}.\\$[A]$ California was once the most thinly populated area in the whole US\\$[B]$ the top 10 states in growth rate of population were all located in the West\\$[C]$ cities with better climates benefited unanimously from migration\\$[D]$ Arizona ranked second of all states in its growth rate of population\\\\66.	The word “demographers” (Line 1, Paragraph 8) most probably means \underline{\hbox to 30mm{}}.\\$[A]$ people in favor of the trend of democracy\\$[B]$ advocates of migration between states\\$[C]$ scientists engaged in the study of population\\$[D]$ conservatives clinging to old patterns of life\\\subsubsection{Text5}

\par
Scattered around the globe are more than 100 small regions of isolated volcanic activity known to geologists as hot spots. Unlike most of the world’s volcanoes, they are not always found at the boundaries of the great drifting plates that make up the earth’s surface; on the contrary, many of them lie deep in the interior of a plate. Most of the hot spots move only slowly, and in some cases the movement of the plates past them has left trails of dead volcanoes. The hot spots and their volcanic trails are milestones that mark the passage of the plates.

\par
That the plates are moving is now beyond dispute. Africa and South America, for example, are moving away from each other as new material is injected into the sea floor between them. The complementary coastlines and certain geological features that seem to span the ocean are reminders of where the two continents were once joined. The relative motion of the plates carrying these continents has been constructed in detail, but the motion of one plate with respect to another cannot readily be translated into motion with respect to the earth’s interior. It is not possible to determine whether both continents are moving in opposite directions or whether one continent is stationary and the other is drifting away from it. Hot spots, anchored in the deeper layers of the earth, provide the measuring instruments needed to resolve the question. From an analysis of the hot-spot population it appears that the African plate is stationary and that it has not moved during the past 30 million years.

\par
The significance of hot spots is not confined to their role as a frame of reference. It now appears that they also have an important influence on the geophysical processes that propel the plates across the globe. When a continental plate come to rest over a hot spot, the material rising from deeper layers creates a broad dome. As the dome grows, it develops deep fissures (cracks); in at least a few cases the continent may break entirely along some of these fissures, so that the hot spot initiates the formation of a new ocean. Thus just as earlier theories have explained the mobility of the continents, so hot spots may explain their mutability (inconstancy).
\\67.	The author believes that \underline{\hbox to 30mm{}}.\\$[A]$ the motion of the plates corresponds to that of the earth’s interior\\$[B]$ the geological theory about drifting plates has been proved to be true\\$[C]$ the hot spots and the plates move slowly in opposite directions\\$[D]$ the movement of hot spots proves the continents are moving apart\\\\68.	That Africa and South America were once joined can be deduced from the fact that \underline{\hbox to 30mm{}}.\\$[A]$ the two continents are still moving in opposite directions\\$[B]$ they have been found to share certain geological features\\$[C]$ the African plate has been stable for 30 million years\\$[D]$ over 100 hot spots are scattered all around the globe\\\\69.	The hot spot theory may prove useful in explaining \underline{\hbox to 30mm{}}.\\$[A]$ the structure of the African plates\\$[B]$ the revival of dead volcanoes\\$[C]$ the mobility of the continents\\$[D]$ the formation of new oceans\\\\70.	The passage is mainly about \underline{\hbox to 30mm{}}.\\$[A]$ the features of volcanic activities\\$[B]$ the importance of the theory about drifting plates\\$[C]$ the significance of hot spots in geophysical studies\\$[D]$ the process of the formation of volcanoes\\\subsection{1999年}
\subsubsection{Text2}

\par
In the first year or so of Web business, most of the action has revolved around efforts to tap the consumer market. More recently, as the Web proved to be more than a fashion, companies have started to buy and sell products and services with one another. Such business-to-business sales make sense because businesspeople typically know what product they’re looking for.

\par
Nonetheless, many companies still hesitate to use the Web because of doubts about its reliability. “Businesses need to feel they can trust the pathway between them and the supplier,” says senior analyst Blane Erwin of Forrester Research. Some companies are limiting the risk by conducting online transactions only with established business partners who are given access to the company’s private intranet.

\par
Another major shift in the model for Internet commerce concerns the technology available for marketing. Until recently, Internet marketing activities have focused on strategies to “pull” customers into sites. In the past year, however, software companies have developed tools that allow companies to “push” information directly out to consumers, transmitting marketing messages directly to targeted customers. Most notably, the Pointcast Network uses a screen saver to deliver a continually updated stream of news and advertisements to subscribers’ computer monitors. Subscribers can customize the information they want to receive and proceed directly to a company’s Web site. Companies such as Virtual Vineyards are already starting to use similar technologies to push messages to customers about special sales, product offerings, or other events. But push technology has earned the contempt of many Web users. Online culture thinks highly of the notion that the information flowing onto the screen comes there by specific

\par
request. Once commercial promotion begins to fill the screen uninvited, the distinction between the Web and television fades. That’s a prospect that horrifies Net purists.

\par
But it is hardly inevitable that companies on the Web will need to resort to push strategies to make money. The examples of Virtual Vineyards, Amazon.com, and other pioneers show that a Web site selling the right kind of products with the right mix of interactivity, hospitality, and security will attract online customers. And the cost of computing power continues to free fall, which is a good sign for any enterprise setting up shop in silicon. People looking back 5 or 10 years from now may well wonder why so few companies took the online plunge.
\\55.	We learn from the beginning of the passage that Web business \underline{\hbox to 30mm{}}.\\$[A]$ has been striving to expand its market\\$[B]$ intended to follow a fanciful fashion\\$[C]$ tried but in vain to control the market\\$[D]$ has been booming for one year or so\\\\56.	Speaking of the online technology available for marketing, the author implies that \underline{\hbox to 30mm{}}.\\$[A]$ the technology is popular with many Web users\\$[B]$ businesses have faith in the reliability of online transactions\\$[C]$ there is a radical change in strategy\\$[D]$ it is accessible limitedly to established partners\\\\57.	In the view of Net purists, \underline{\hbox to 30mm{}}.\\$[A]$ there should be no marketing messages in online culture\\$[B]$ money making should be given priority to on the Web\\$[C]$ the Web should be able to function as the television set\\$[D]$ there should be no online commercial information without requests\\\\58.	We learn from the last paragraph that \underline{\hbox to 30mm{}}.\\$[A]$ pushing information on the Web is essential to Internet commerce\\$[B]$ interactivity, hospitality and security are important to online customers\\$[C]$ leading companies began to take the online plunge decades ago\\$[D]$ setting up shops in silicon is independent of the cost of computing power\\\subsubsection{Text3}

\par
An invisible border divides those arguing for computers in the classroom on the behalf of students’ career prospects and those arguing for computers in the classroom for broader reasons of radical educational reform. Very few writers on the subject have explored this distinction -- indeed, contradiction -- which goes to the heart of what is wrong with the campaign to put computers in the classroom.

\par
An education that aims at getting a student a certain kind of job is a technical education, justified for reasons radically different from why education is universally required by law. It is not simply to raise everyone’s job prospects that all children are legally required to attend school into their teens. Rather, we have a certain conception of the American citizen, a character who is incomplete if he cannot competently assess how his livelihood and happiness are affected by things outside of himself. But this was not always the case; before it was legally required for all children to attend school until a certain age, it was widely accepted that some were just not equipped by nature to pursue this kind of education. With optimism characteristic of all industrialized countries, we came to accept that everyone is fit to be educated. Computer-education advocates forsake this optimistic notion for a pessimism that betrays their otherwise cheery outlook. Banking on the confusion between educational and

\par
vocational reasons for bringing computers into schools, computered advocates often emphasize the job prospects of graduates over their educational achievement.

\par
There are some good arguments for a technical education given the right kind of student. Many European schools introduce the concept of professional training early on in order to make sure children are properly equipped for the professions they want to join. It is, however, presumptuous to insist that there will only be so many jobs for so many scientists, so many businessmen, so many accountants. Besides, this is unlikely to produce the needed number of every kind of professional in a country as large as ours and where the economy is spread over so many states and involves so many international corporations.

\par
But, for a small group of students, professional training might be the way to go since well-developed skills, all other factors being equal, can be the difference between having a job and not. Of course, the basics of using any computer these days are very simple. It does not take a lifelong acquaintance to pick up various software programs. If one wanted to become a computer engineer, that is, of course, an entirely different story. Basic computer skills take -- at the very longest -- a couple of months to learn. In any case, basic computer skills are only complementary to the host of real skills that are necessary to becoming any kind of professional. It should be observed, of course, that no school, vocational or not, is helped by a confusion over its purpose.
\\59.	The author thinks the present rush to put computers in the classroom is \underline{\hbox to 30mm{}}.\\$[A]$ far-reaching\\$[B]$ dubiously oriented\\$[C]$ self-contradictory\\$[D]$ radically reformatory\\\\60.	The belief that education is indispensable to all children \underline{\hbox to 30mm{}}.\\$[A]$ is indicative of a pessimism in disguise\\$[B]$ came into being along with the arrival of computers\\$[C]$ is deeply rooted in the minds of computered advocates\\$[D]$ originated from the optimistic attitude of industrialized countries\\\\61.	It could be inferred from the passage that in the author’s country the European model of professional training is \underline{\hbox to 30mm{}}.\\$[A]$ dependent upon the starting age of candidates\\$[B]$ worth trying in various social sections\\$[C]$ of little practical value\\$[D]$ attractive to every kind of professional\\\\62.	According to the author, basic computer skills should be \underline{\hbox to 30mm{}}.\\$[A]$ included as an auxiliary course in school\\$[B]$ highlighted in acquisition of professional qualifications\\$[C]$ mastered through a life-long course\\$[D]$ equally emphasized by any school, vocational or otherwise\\\subsubsection{Text5}

\par
Science, in practice, depends far less on the experiments it prepares than on the preparedness of the minds of the men who watch the experiments. Sir Isaac Newton supposedly discovered gravity through the fall of an apple. Apples had been falling in many places for centuries and thousands of people had seen them fall. But Newton for years had been curious about the cause of the orbital motion of the moon and planets. What kept them in place? Why didn’t they fall out of the sky? The fact that the apple fell down toward the earth and not up into the tree answered the question he had been asking himself about those larger fruits of the heavens, the moon and the planets.

\par
How many men would have considered the possibility of an apple falling up into the tree? Newton did because he was not trying to predict anything. He was just wondering. His mind was ready for the unpredictable. Unpredictability is part of the essential nature of research. If you don’t have unpredictable things, you don’t have research. Scientists tend to forget this when writing their cut and dried reports for the technical journals, but history is filled with examples of it.

\par
In talking to some scientists, particularly younger ones, you might gather the impression that they find the “scientific method” a substitute for imaginative thought. I’ve attended research conferences where a scientist has been asked what he thinks about the advisability of continuing a certain experiment. The scientist has frowned, looked at the graphs, and said “the data are still inconclusive.” “We know that,” the men from the budget office have said, “but what do you think? Is it worthwhile going on? What do you think we might expect?” The scientist has been shocked at having even been asked to speculate.

\par
What this amounts to, of course, is that the scientist has become the victim of his own writings. He has put forward unquestioned claims so consistently that he not only believes them himself, but has convinced industrial and business management that they are true. If experiments are planned and carried out according to plan as faithfully as the reports in the science journals indicate, then it is perfectly logical for management to expect research to produce results measurable in dollars and cents. It is entirely reasonable for auditors to believe that scientists who know exactly where they are going and how they will get there should not be distracted by the necessity of keeping one eye on the cash register while the other eye is on the microscope. Nor, if regularity and conformity to a standard pattern are as desirable to the scientist as the writing of his papers would appear to reflect, is management to be blamed for discriminating against the “odd balls” among researchers in favor of more conventional

\par
thinkers who “work well with the team.”
\\67.	The author wants to prove with the example of Isaac Newton that \underline{\hbox to 30mm{}}.\\$[A]$ inquiring minds are more important than scientific experiments\\$[B]$ science advances when fruitful researches are conducted\\$[C]$ scientists seldom forget the essential nature of research\\$[D]$ unpredictability weighs less than prediction in scientific research\\\\68.	The author asserts that scientists \underline{\hbox to 30mm{}}.\\$[A]$ shouldn’t replace “scientific method” with imaginative thought\\$[B]$ shouldn’t neglect to speculate on unpredictable things\\$[C]$ should write more concise reports for technical journals\\$[D]$ should be confident about their research findings\\\\69.	It seems that some young scientists \underline{\hbox to 30mm{}}.\\$[A]$ have a keen interest in prediction\\$[B]$ often speculate on the future\\$[C]$ think highly of creative thinking\\$[D]$ stick to “scientific method”\\\\70.	The author implies that the results of scientific research \underline{\hbox to 30mm{}}.\\$[A]$ may not be as profitable as they are expected\\$[B]$ can be measured in dollars and cents\\$[C]$ rely on conformity to a standard pattern\\$[D]$ are mostly underestimated by management\\\subsection{2001年}
\subsubsection{Text1}

\par
Specialization can be seen as a response to the problem of an increasing accumulation of scientific knowledge. By splitting up the subject matter into smaller units, one man could continue to handle the information and use it as the basis for further research. But specialization was only one of a series of related developments in science affecting the process of communication. Another was the growing professionalisation of scientific activity.

\par
No clear-cut distinction can be drawn between professionals and amateurs in science: exceptions can be found to any rule. Nevertheless, the word “amateur” does carry a connotation that the person concerned is not fully integrated into the scientific community and, in particular, may not fully share its values. The growth of specialization in the nineteenth century, with its consequent requirement of a longer, more complex training, implied greater problems for amateur participation in science. The trend was naturally most obvious in those areas of science based especially on a mathematical or laboratory training, and can be illustrated in terms of the development of geology in the United Kingdom.

\par
A comparison of British geological publications over the last century and a half reveals not simply an increasing emphasis on the primacy of research, but also a changing definition of what constitutes an acceptable research paper. Thus, in the nineteenth century, local geological studies represented worthwhile research in their own right; but, in the twentieth century, local studies have increasingly become acceptable to professionals only if they incorporate, and reflect on, the wider geological picture. Amateurs, on the other hand, have continued to pursue local studies in the old way. The overall result has been to make entrance to professional geological journals harder for amateurs, a result that has been reinforced by the widespread introduction of refereeing, first by national journals in the nineteenth century and then by several local geological journals in the twentieth century. As a logical consequence of this development, separate journals have now appeared aimed mainly towards either

\par
professional or amateur readership. A rather similar process of differentiation has led to professional geologists coming together nationally within one or two specific societies, whereas the amateurs have tended either to remain in local societies or to come together nationally in a different way.

\par
Although the process of professionalisation and specialization was already well under way in British geology during the nineteenth century, its full consequences were thus delayed until the twentieth century. In science generally, however, the nineteenth century must be reckoned as the crucial period for this change in the structure of science.
\\51.	The growth of specialization in the 19th century might be more clearly seen in sciences such as \underline{\hbox to 30mm{}}.\\$[A]$ sociology and chemistry\\$[B]$ physics and psychology\\$[C]$ sociology and psychology\\$[D]$ physics and chemistry\\\\52.	We can infer from the passage that \underline{\hbox to 30mm{}}.\\$[A]$ there is little distinction between specialization and professionalisation\\$[B]$ amateurs can compete with professionals in some areas of science\\$[C]$ professionals tend to welcome amateurs into the scientific community\\$[D]$ amateurs have national academic societies but no local ones\\\\53.	The author writes of the development of geology to demonstrate \underline{\hbox to 30mm{}}.\\$[A]$ the process of specialization and professionalisation\\$[B]$ the hardship of amateurs in scientific study\\$[C]$ the change of policies in scientific publications\\$[D]$ the discrimination of professionals against amateurs\\\\54.	The direct reason for specialization is \underline{\hbox to 30mm{}}.\\$[A]$ the development in communication\\$[B]$ the growth of professionalisation\\$[C]$ the expansion of scientific knowledge\\$[D]$ the splitting up of academic societies\\\subsubsection{Text4}

\par
The world is going through the biggest wave of mergers and acquisitions ever witnessed. The process sweeps from hyperactive America to Europe and reaches the emerging countries with unsurpassed might. Many in these countries are looking at this process and worrying: “Won’t the wave of business concentration turn into an uncontrollable anti-competitive force?”

\par
There’s no question that the big are getting bigger and more powerful. Multinational corporations accounted for less than 20\% of international trade in 1982. Today the figure is more than 25\% and growing rapidly. International affiliates account for a fast-growing segment of production in economies that open up and welcome foreign investment. In Argentina, for instance, after the reforms of the early 1990s, multinationals went from 43\% to almost 70\% of the industrial production of the 200 largest firms. This phenomenon has created serious concerns over the role of smaller economic firms, of national businessmen and over the ultimate stability of the world economy.

\par
I believe that the most important forces behind the massive M\&A wave are the same that underlie the globalization process: falling transportation and communication costs, lower trade and investment barriers and enlarged markets that require enlarged operations capable of meeting customer’s demands. All these are beneficial, not detrimental, to consumers. As productivity grows, the world’s wealth increases.

\par
Examples of benefits or costs of the current concentration wave are scanty. Yet it is hard to imagine that the merger of a few oil firms today could re-create the same threats to competition that were feared nearly a century ago in the U.S., when the Standard Oil Trust was broken up. The mergers of telecom companies, such as WorldCom, hardly seem to bring higher prices for consumers or a reduction in the pace of technical progress. On the contrary, the price of communications is coming down fast. In cars, too, concentration is increasing -- witness Daimler and Chrysler, Renault and Nissan -- but it does not appear that consumers are being hurt.

\par
Yet the fact remains that the merger movement must be watched. A few weeks ago, Alan Greenspan warned against the megamergers in the banking industry. Who is going to supervise, regulate and operate as lender of last resort with the gigantic banks that are being created? Won’t multinationals shift production from one place to another when a nation gets too strict about infringements to fair competition? And should one country take upon itself the role of “defending competition” on issues that affect many other nations, as in the U.S. vs. Microsoft case?
\\63.	What is the typical trend of businesses today?\\$[A]$ to take in more foreign funds\\$[B]$ to invest more abroad\\$[C]$ to combine and become bigger\\$[D]$ to trade with more countries\\\\64.	According to the author, one of the driving forces behind M\&A wave is \underline{\hbox to 30mm{}}.\\$[A]$ the greater customer demands\\$[B]$ a surplus supply for the market\\$[C]$ a growing productivity\\$[D]$ the increase of the world’s wealth\\\\65.	From Paragraph 4 we can infer that \underline{\hbox to 30mm{}}.\\$[A]$ the increasing concentration is certain to hurt consumers\\$[B]$ WorldCom serves as a good example of both benefits and costs\\$[C]$ the costs of the globalization process are enormous\\$[D]$ the Standard Oil Trust might have threatened competition\\\\66.	Toward the new business wave, the writer’s attitude can be said to be \underline{\hbox to 30mm{}}.\\$[A]$ optimistic\\$[B]$ objective\\$[C]$ pessimistic\\$[D]$ biased\\\subsection{2002年}
\subsubsection{Text1}

\par
If you intend using humor in your talk to make people smile, you must know how to identify shared experiences and problems. Your humor must be relevant to the audience and should help to show them that you are one of them or that you understand their situation and are in sympathy with their point of view. Depending on whom you are addressing, the problems will be different. If you are talking to a group of managers, you may refer to the disorganized methods of their secretaries; alternatively if you are addressing secretaries, you may want to comment on their disorganized bosses.

\par
Here is an example, which I heard at a nurses’ convention, of a story which works well because the audience all shared the same view of doctors. A man arrives in heaven and is being shown around by St. Peter. He sees wonderful accommodations, beautiful gardens, sunny weather, and so on. Everyone is very peaceful, polite and friendly until, waiting in a line for lunch, the new arrival is suddenly pushed aside by a man in a white coat, who rushes to the head of the line, grabs his food and stomps over to a table by himself. “Who is that?” the new arrival asked St. Peter. “Oh, that’s God,” came the reply, “but sometimes he thinks he’s a doctor.”

\par
If you are part of the group, which you are addressing, you will be in a position to know the experiences and problems which are common to all of you and it’ll be appropriate for you to make a passing remark about the inedible canteen food or the chairman’s notorious bad taste in ties. With other audiences you mustn’t attempt to cut in with humor as they will resent an outsider making disparaging remarks about their canteen or their chairman. You will be on safer ground if you stick to scapegoats like the Post Office or the telephone system.

\par
If you feel awkward being humorous, you must practice so that it becomes more natural. Include a few casual and apparently off-the-cuff remarks which you can deliver in a relaxed and unforced manner. Often it’s the delivery which causes the audience to smile, so speak slowly and remember that a raised eyebrow or an unbelieving look may help to show that you are making a light-hearted remark.

\par
Look for the humor. It often comes from the unexpected. A twist on a familiar quote “If at first you don’t succeed, give up” or a play on words or on a situation. Search for exaggeration and understatements. Look at your talk and pick out a few words or sentences which you can turn about and inject with humor.
\\41.	To make your humor work, you should \underline{\hbox to 30mm{}}.\\$[A]$ take advantage of different kinds of audience\\$[B]$ make fun of the disorganized people\\$[C]$ address different problems to different people\\$[D]$ show sympathy for your listeners\\\\42.	The joke about doctors implies that, in the eyes of nurses, they are \underline{\hbox to 30mm{}}.\\$[A]$ impolite to new arrivals\\$[B]$ very conscious of their godlike role\\$[C]$ entitled to some privileges\\$[D]$ very busy even during lunch hours\\\\43.	It can be inferred from the text that public services \underline{\hbox to 30mm{}}.\\$[A]$ have benefited many people\\$[B]$ are the focus of public attention\\$[C]$ are an inappropriate subject for humor\\$[D]$ have often been the laughing stock\\\\44.	To achieve the desired result, humorous stories should be delivered \underline{\hbox to 30mm{}}.\\$[A]$ in well-worded language\\$[B]$ as awkwardly as possible\\$[C]$ in exaggerated statements\\$[D]$ as casually as possible\\\\45.	The best title for the text may be \underline{\hbox to 30mm{}}.\\$[A]$ Use Humor Effectively\\$[B]$ Various Kinds of Humor\\$[C]$ Add Humor to Speech\\$[D]$ Different Humor Strategies\\\subsection{2003年}
\subsubsection{Text4}

\par
It is said that in England death is pressing, in Canada inevitable and in California optional. Small wonder. Americans’ life expectancy has nearly doubled over the past century. Failing hips can be replaced, clinical depression controlled, cataracts removed in a 30-minute surgical procedure. Such advances offer the aging population a quality of life that was unimaginable when I entered medicine 50 years ago. But not even a great health-care system can cure death -- and our failure to confront that reality now threatens this greatness of ours.

\par
Death is normal; we are genetically programmed to disintegrate and perish, even under ideal conditions. We all understand that at some level, yet as medical consumers we treat death as a problem to be solved. Shielded by third-party payers from the cost of our care, we demand everything that can possibly be done for us, even if it’s useless. The most obvious example is late-stage cancer care. Physicians -- frustrated by their inability to cure the disease and fearing loss of hope in the patient -- too often offer aggressive treatment far beyond what is scientifically justified.

\par
In 1950, the U.S. spent \$12.7 billion on health care. In 2002, the cost will be \$1,540 billion. Anyone can see this trend is unsustainable. Yet few seem willing to try to reverse it. Some scholars conclude that a government with finite resources should simply stop paying for medical care that sustains life beyond a certain age -- say 83 or so. Former Colorado governor Richard Lamm has been quoted as saying that the old and infirm “have a duty to die and get out of the way,” so that younger, healthier people can realize their potential.

\par
I would not go that far. Energetic people now routinely work through their 60s and beyond, and remain dazzlingly productive. At 78, Viacom chairman Sumner Redstone jokingly claims to be 53. Supreme Court Justice Sandra Day O’Connor is in her 70s, and former surgeon general C. Everett Koop chairs an Internet start-up in his 80s. These leaders are living proof that prevention works and that we can manage the health problems that come naturally with age. As a mere 68-year-old, I wish to age as productively as they have.

\par
Yet there are limits to what a society can spend in this pursuit. As a physician, I know the most costly and dramatic measures may be ineffective and painful. I also know that people in Japan and Sweden, countries that spend far less on medical care, have achieved longer, healthier lives than we have. As a nation, we may be overfunding the quest for unlikely cures while underfunding research on humbler therapies that could improve people’s lives.
\\56.	What is implied in the first sentence?\\$[A]$ Americans are better prepared for death than other people.\\$[B]$ Americans enjoy a higher life quality than ever before.\\$[C]$ Americans are over-confident of their medical technology.\\$[D]$ Americans take a vain pride in their long life expectancy.\\\\57.	The author uses the example of cancer patients to show that \underline{\hbox to 30mm{}}.\\$[A]$ medical resources are often wasted\\$[B]$ doctors are helpless against fatal diseases\\$[C]$ some treatments are too aggressive\\$[D]$ medical costs are becoming unaffordable\\\\58.	The author’s attitude toward Richard Lamm’s remark is one of \underline{\hbox to 30mm{}}.\\$[A]$ strong disapproval\\$[B]$ reserved consent\\$[C]$ slight contempt\\$[D]$ enthusiastic support\\\\59.	In contrast to the U.S., Japan and Sweden are funding their medical care \underline{\hbox to 30mm{}}.\\$[A]$ more flexibly\\$[B]$ more extravagantly\\$[C]$ more cautiously\\$[D]$ more reasonably\\\\60.	The text intends to express the idea that \underline{\hbox to 30mm{}}.\\$[A]$ medicine will further prolong people’s lives\\$[B]$ life beyond a certain limit is not worth living\\$[C]$ death should be accepted as a fact of life\\$[D]$ excessive demands increase the cost of health care\\\subsection{2004年}
\subsubsection{Text2}

\par
Over the past century, all kinds of unfairness and discrimination have been condemned or made illegal. But one insidious form continues to thrive: alphabetism. This, for those as yet unaware of such a disadvantage, refers to discrimination against those whose surnames begin with a letter in the lower half of the alphabet.

\par
It has long been known that a taxi firm called AAAA cars has a big advantage over Zodiac cars when customers thumb through their phone directories. Less well known is the advantage that Adam Abbott has in life over Zoë Zysman. English names are fairly evenly spread between the halves of the alphabet. Yet a suspiciously large number of top people have surnames beginning with letters between A and K.

\par
Thus the American president and vice-president have surnames starting with B and C respectively; and 26 of George Bush’s predecessors (including his father) had surnames in the first half of the alphabet against just 16 in the second half. Even more striking, six of the seven heads of government of the G7 rich countries are alphabetically advantaged (Berlusconi, Blair, Bush, Chirac, Chrétien and Koizumi). The world’s three top central bankers (Greenspan, Duisenberg and Hayami) are all close to the top of the alphabet, even if one of them really uses Japanese characters. As are the world’s five richest men (Gates, Buffett, Allen, Ellison and Albrecht).

\par
Can this merely be coincidence? One theory, dreamt up in all the spare time enjoyed by the alphabetically disadvantaged, is that the rot sets in early. At the start of the first year in infant school, teachers seat pupils alphabetically from the front, to make it easier to remember their names. So short-sighted Zysman junior gets stuck in the back row, and is rarely asked the improving questions posed by those insensitive teachers. At the time the alphabetically disadvantaged may think they have had a lucky escape. Yet the result may be worse qualifications, because they get less individual attention, as well as less confidence in speaking publicly.

\par
The humiliation continues. At university graduation ceremonies, the ABCs proudly get their awards first; by the time they reach the Zysmans most people are literally having a ZZZ. Shortlists for job interviews, election ballot papers, lists of conference speakers and attendees: all tend to be drawn up alphabetically, and their recipients lose interest as they plough through them.
\\46.	What does the author intend to illustrate with AAA A cars and Zodiac cars?\\$[A]$ A kind of overlooked inequality.\\$[B]$ A type of conspicuous bias.\\$[C]$ A type of personal prejudice.\\$[D]$ A kind of brand discrimination.\\\\47.	What can we infer from the first three paragraphs?\\$[A]$ In both East and West, names are essential to success.\\$[B]$ The alphabet is to blame for the failure of Zoë Zysman.\\$[C]$ Customers often pay a lot of attention to companies’ names.\\$[D]$ Some form of discrimination is too subtle to recognize.\\\\48.	The 4th paragraph suggests that \underline{\hbox to 30mm{}}.\\$[A]$ questions are often put to the more intelligent students\\$[B]$ alphabetically disadvantaged students often escape from class\\$[C]$ teachers should pay attention to all of their students\\$[D]$ students should be seated according to their eyesight\\\\49.	What does the author mean by “most people are literally having a ZZZ” (Lines 2-3, Paragraph 5)?\\$[A]$ They are getting impatient.\\$[B]$ They are noisily dozing off.\\$[C]$ They are feeling humiliated.\\$[D]$ They are busy with word puzzles.\\\\50.	Which of the following is true according to the text?\\$[A]$ People with surnames beginning with N to Z are often ill-treated.\\$[B]$ VIPs in the Western world gain a great deal from alphabetism.\\$[C]$ The campaign to eliminate alphabetism still has a long way to go.\\$[D]$ Putting things alphabetically may lead to unintentional bias.\\\subsubsection{Text4}

\par
Americans today don’t place a very high value on intellect. Our heroes are athletes, entertainers, and entrepreneurs, not scholars. Even our schools are where we send our children to get a practical education -- not to pursue knowledge for the sake of knowledge. Symptoms of pervasive anti-intellectualism in our schools aren’t difficult to find.

\par
“Schools have always been in a society where practical is more important than intellectual,” says education writer Diane Ravitch. “Schools could be a counterbalance.” Ravitch’s latest book, Left Back: A Century of Failed School Reforms, traces the roots of anti-intellectualism in our schools, concluding they are anything but a counterbalance to the American distaste for intellectual pursuits.

\par
But they could and should be. Encouraging kids to reject the life of the mind leaves them vulnerable to exploitation and control. Without the ability to think critically, to defend their ideas and understand the ideas of others, they cannot fully participate in our democracy. Continuing along this path, says writer Earl Shorris, “We will become a second-rate country. We will have a less civil society.”

\par
“Intellect is resented as a form of power or privilege,” writes historian and professor Richard Hofstadter in Anti-Intellectualism in American Life, a Pulitzer-Prize winning book on the roots of anti-intellectualism in US politics, religion, and education. From the beginning of our history, says Hofstadter, our democratic and populist urges have driven us to reject anything that smells of elitism. Practicality, common sense, and native intelligence have been considered more noble qualities than anything you could learn from a book.

\par
Ralph Waldo Emerson and other Transcendentalist philosophers thought schooling and rigorous book learning put unnatural restraints on children: “We are shut up in schools and college recitation rooms for 10 or 15 years and come out at last with a bellyful of words and do not know a thing.” Mark Twain’s Huckleberry Finn exemplified American anti-intellectualism. Its hero avoids being civilized -- going to school and learning to read -- so he can preserve his innate goodness.

\par
Intellect, according to Hofstadter, is different from native intelligence, a quality we reluctantly admire. Intellect is the critical, creative, and contemplative side of the mind. Intelligence seeks to grasp, manipulate, re-order, and adjust, while intellect examines, ponders, wonders, theorizes, criticizes and imagines.

\par
School remains a place where intellect is mistrusted. Hofstadter says our country’s educational system is in the grips of people who “joyfully and militantly proclaim their hostility to intellect and their eagerness to identify with children who show the least intellectual promise.”
\\56.	What do American parents expect their children to acquire in school?\\$[A]$ The habit of thinking independently.\\$[B]$ Profound knowledge of the world.\\$[C]$ Practical abilities for future career.\\$[D]$ The confidence in intellectual pursuits.\\\\57.	We can learn from the text that Americans have a history of \underline{\hbox to 30mm{}}.\\$[A]$ undervaluing intellect\\$[B]$ favoring intellectualism\\$[C]$ supporting school reform\\$[D]$ suppressing native intelligence\\\\58.	The views of Ravitch and Emerson on schooling are \underline{\hbox to 30mm{}}.\\$[A]$ identical\\$[B]$ similar\\$[C]$ complementary\\$[D]$ opposite\\\\59.	Emerson, according to the text, is probably \underline{\hbox to 30mm{}}.\\$[A]$ a pioneer of education reform\\$[B]$ an opponent of intellectualism\\$[C]$ a scholar in favor of intellect\\$[D]$ an advocate of regular schooling\\\\60.	What does the author think of intellect?\\$[A]$ It is second to intelligence.\\$[B]$ It evolves from common sense.\\$[C]$ It is to be pursued.\\$[D]$ It underlies power.\\\subsection{2006年}
\subsubsection{Text1}

\par
In spite of “endless talk of difference,” American society is an amazing machine for homogenizing people. There is “the democratizing uniformity of dress and discourse, and the casualness and absence of deference” characteristic of popular culture. People are absorbed into “a culture of consumption” launched by the 19th-century department stores that offered “vast arrays of goods in an elegant atmosphere. Instead of intimate shops catering to a knowledgeable elite,” these were stores “anyone could enter, regardless of class or background. This turned shopping into a public and democratic act.” The mass media, advertising and sports are other forces for homogenization.

\par
Immigrants are quickly fitting into this common culture, which may not be altogether elevating but is hardly poisonous. Writing for the National Immigration Forum, Gregory Rodriguez reports that today’s immigration is neither at unprecedented levels nor resistant to assimilation. In 1998 immigrants were 9.8 percent of population; in 1900, 13.6 percent. In the 10 years prior to 1990, 3.1 immigrants arrived for every 1,000 residents; in the 10 years prior to 1890, 9.2 for every 1,000. Now, consider three indices of assimilation -- language, home ownership and intermarriage.

\par
The 1990 Census revealed that “a majority of immigrants from each of the fifteen most common countries of origin spoke English ‘well’ or ‘very well’ after ten years of residence.” The children of immigrants tend to be bilingual and proficient in English. “By the third generation, the original language is lost in the majority of immigrant families.” Hence the description of America as a “graveyard” for languages. By 1996 foreign-born immigrants who had arrived before 1970 had a home ownership rate of 75.6 percent, higher than the 69.8 percent rate among native-born Americans.

\par
Foreign-born Asians and Hispanics “have higher rates of intermarriage than do U.S.-born whites and blacks.” By the third generation, one third of Hispanic women are married to non-Hispanics, and 41 percent of Asian-American women are married to non-Asians.

\par
Rodriguez notes that children in remote villages around the world are fans of superstars like Arnold Schwarzenegger and Garth Brooks, yet “some Americans fear that immigrants living within the United States remain somehow immune to the nation’s assimilative power.”

\par
Are there divisive issues and pockets of seething anger in America? Indeed. It is big enough to have a bit of everything. But particularly when viewed against America’s turbulent past, today’s social indices hardly suggest a dark and deteriorating social environment.
\\21.	The word “homogenizing” (Line 2, Paragraph 1) most probably means \underline{\hbox to 30mm{}}.\\$[A]$ identifying\\$[B]$ associating\\$[C]$ assimilating\\$[D]$ monopolizing\\\\22.	According to the author, the department stores of the 19th century \underline{\hbox to 30mm{}}.\\$[A]$ played a role in the spread of popular culture\\$[B]$ became intimate shops for common consumers\\$[C]$ satisfied the needs of a knowledgeable elite\\$[D]$ owed its emergence to the culture of consumption\\\\23.	The text suggests that immigrants now in the U.S. \underline{\hbox to 30mm{}}.\\$[A]$ are resistant to homogenization\\$[B]$ exert a great influence on American culture\\$[C]$ are hardly a threat to the common culture\\$[D]$ constitute the majority of the population\\\\24.	Why are Arnold Schwarzenegger and Garth Brooks mentioned in Paragraph 5?\\$[A]$ To prove their popularity around the world.\\$[B]$ To reveal the public’s fear of immigrants.\\$[C]$ To give examples of successful immigrants.\\$[D]$ To show the powerful influence of American culture.\\\\25.	In the author’s opinion, the absorption of immigrants into American society is \underline{\hbox to 30mm{}}.\\$[A]$ rewarding\\$[B]$ successful\\$[C]$ fruitless\\$[D]$ harmful\\\subsubsection{Text3}

\par
When prehistoric man arrived in new parts of the world, something strange happened to the large animals. They suddenly became extinct. Smaller species survived. The large, slow-growing animals were easy game, and were quickly hunted to extinction. Now something similar could be happening in the oceans.

\par
That the seas are being overfished has been known for years. What researchers such as Ransom Myers and Boris Worm have shown is just how fast things are changing. They have looked at half a century of data from fisheries around the world. Their methods do not attempt to estimate the actual biomass (the amount of living biological matter) of fish species in particular parts of the ocean, but rather changes in that biomass over time. According to their latest paper published in Nature, the biomass of large predators (animals that kill and eat other animals) in a new fishery is reduced on average by 80\% within 15 years of the start of exploitation. In some long-fished areas, it has halved again since then.

\par
Dr. Worm acknowledges that these figures are conservative. One reason for this is that fishing technology has improved. Today’s vessels can find their prey using satellites and sonar, which were not available 50 years ago. That means a higher proportion of what is in the sea is being caught, so the real difference between present and past is likely to be worse than the one recorded by changes in catch sizes. In the early days, too, longlines would have been more saturated with fish. Some individuals would therefore not have been caught, since no baited hooks would have been available to trap them, leading to an underestimate of fish stocks in the past. Furthermore, in the early days of longline fishing, a lot of fish were lost to sharks after they had been hooked. That is no longer a problem, because there are fewer sharks around now.

\par
Dr. Myers and Dr. Worm argue that their work gives a correct baseline, which future management efforts must take into account. They believe the data support an idea current among marine biologists, that of the “shifting baseline.” The notion is that people have failed to detect the massive changes which have happened in the ocean because they have been looking back only a relatively short time into the past. That matters because theory suggests that the maximum sustainable yield that can be cropped from a fishery comes when the biomass of a target species is about 50\% of its original levels. Most fisheries are well below that, which is a bad way to do business.
\\31.	The extinction of large prehistoric animals is noted to suggest that \underline{\hbox to 30mm{}}.\\$[A]$ large animal were vulnerable to the changing environment\\$[B]$ small species survived as large animals disappeared\\$[C]$ large sea animals may face the same threat today\\$[D]$ slow-growing fish outlive fast-growing ones\\\\32.	We can infer from Dr. Myers and Dr. Worm’s paper that \underline{\hbox to 30mm{}}.\\$[A]$ the stock of large predators in some old fisheries has reduced by 90\%\\$[B]$ there are only half as many fisheries as there were 15 years ago\\$[C]$ the catch sizes in new fisheries are only 20\% of the original amount\\$[D]$ the number of larger predators dropped faster in new fisheries than in the old\\\\33.	By saying "these figures are conservative" (Line 1, paragraph 3), Dr. Worm means that \underline{\hbox to 30mm{}}.\\$[A]$ fishing technology has improved rapidly\\$[B]$ the catch-sizes are actually smaller than recorded\\$[C]$ the marine biomass has suffered a greater loss\\$[D]$ the data collected so far are out of date\\\\34.	Dr. Myers and other researchers hold that \underline{\hbox to 30mm{}}.\\$[A]$ people should look for a baseline that can work for a longer time\\$[B]$ fisheries should keep their yields below 50\% of the biomass\\$[C]$ the ocean biomass should be restored to its original level\\$[D]$ people should adjust the fishing baseline to the changing situation\\\\35.	The author seems to be mainly concerned with most fisheries’ \underline{\hbox to 30mm{}}.\\$[A]$ management efficiency\\$[B]$ biomass level\\$[C]$ catch-size limits\\$[D]$ technological application\\\subsection{2007年}
\subsubsection{Text1}

\par
If you were to examine the birth certificates of every soccer player in 2006’s World Cup tournament, you would most likely find a noteworthy quirk: elite soccer players are more likely to have been born in the earlier months of the year than in the later months. If you then examined the European national youth teams that feed the World Cup and professional ranks, you would find this strange phenomenon to be even more pronounced.

\par
What might account for this strange phenomenon? Here are a few guesses: a) certain astrological signs confer superior soccer skills; b) winter-born babies tend to have higher oxygen capacity, which increases soccer stamina; c) soccer-mad parents are more likely to conceive children in springtime, at the annual peak of soccer mania; d) none of the above.

\par
Anders Ericsson, a 58-year-old psychology professor at Florida State University, says he believes strongly in “none of the above.” Ericsson grew up in Sweden, and studied nuclear engineering until he realized he would have more opportunity to conduct his own research if he switched to psychology. His first experiment, nearly 30 years ago, involved memory: training a person to hear and then repeat a random series of numbers. “With the first subject, after about 20 hours of training, his digit span had risen from 7 to 20,” Ericsson recalls. “He kept improving, and after about 200 hours of training he had risen to over 80 numbers.”

\par
This success, coupled with later research showing that memory itself is not genetically determined, led Ericsson to conclude that the act of memorizing is more of a cognitive exercise than an intuitive one. In other words, whatever inborn differences two people may exhibit in their abilities to memorize, those differences are swamped by how well each person “encodes” the information. And the best way to learn how to encode information meaningfully, Ericsson determined, was a process known as deliberate practice. Deliberate practice entails more than simply repeating a task. Rather, it involves setting specific goals, obtaining immediate feedback and concentrating as much on technique as on outcome.

\par
Ericsson and his colleagues have thus taken to studying expert performers in a wide range of pursuits, including soccer. They gather all the data they can, not just performance statistics and biographical details but also the results of their own laboratory experiments with high achievers. Their work makes a rather startling assertion: the trait we commonly call talent is highly overrated. Or, put another way, expert performers – whether in memory or surgery, ballet or computer programming – are nearly always made, not born.
\\21.	The birthday phenomenon found among soccer players is mentioned to\\$[A]$ stress the importance of professional training.\\$[B]$ spotlight the soccer superstars in the World Cup.\\$[C]$ introduce the topic of what makes expert performance.\\$[D]$ explain why some soccer teams play better than others.\\\\22.	The word “mania” (Line 4, Paragraph 2) most probably means\\$[A]$ fun.\\$[B]$ craze.\\$[C]$ hysteria.\\$[D]$ excitement.\\\\23.	According to Ericsson, good memory\\$[A]$ depends on meaningful processing of information.\\$[B]$ results from intuitive rather than cognitive exercises.\\$[C]$ is determined by genetic rather than psychological factors.\\$[D]$ requires immediate feedback and a high degree of concentration.\\\\24.	Ericsson and his colleagues believe that\\$[A]$ talent is a dominating factor for professional success.\\$[B]$ biographical data provide the key to excellent performance.\\$[C]$ the role of talent tends to be overlooked.\\$[D]$ high achievers owe their success mostly to nurture.\\\\25.	Which of the following proverbs is closest to the message the text tries to convey?\\$[A]$ “Faith will move mountains.”\\$[B]$ “One reaps what one sows.”\\$[C]$ “Practice makes perfect.”\\$[D]$ “Like father, like son.”\\\subsubsection{Text3}

\par
During the past generation, the American middle-class family that once could count on hard work and fair play to keep itself financially secure had been transformed by economic risk and new realities. Now a pink slip, a bad diagnosis, or a disappearing spouse can reduce a family from solidly middle class to newly poor in a few months.

\par
In just one generation, millions of mothers have gone to work, transforming basic family economics. Scholars, policymakers, and critics of all stripes have debated the social implications of these changes, but few have looked at the side effect: family risk has risen as well. Today’s families have budgeted to the limits of their new two-paycheck status. As a result, they have lost the parachute they once had in times of financial setback – a back-up earner (usually Mom) who could go into the workforce if the primary earner got laid off or fell sick. This “added-worker effect” could support the safety net offered by unemployment insurance or disability insurance to help families weather bad times. But today, a disruption to family fortunes can no longer be made up with extra income from an otherwise-stay-at-home partner.

\par
During the same period, families have been asked to absorb much more risk in their retirement income. Steelworkers, airline employees, and now those in the auto industry are joining millions of families who must worry about interest rates, stock market fluctuation, and the harsh reality that they may outlive their retirement money. For much of the past year, President Bush campaigned to move Social Security to a saving-account model, with retirees trading much or all of their guaranteed payments for payments depending on investment returns. For younger families, the picture is not any better. Both the absolute cost of healthcare and the share of it borne by families have risen – and newly fashionable health-savings plans are spreading from legislative halls to Wal-Mart workers, with much higher deductibles and a large new dose of investment risk for families’ future healthcare. Even demographics are working against the middle class family, as the odds of having a weak elderly parent – and all the attendant

\par
need for physical and financial assistance – have jumped eightfold in just one generation.

\par
From the middle-class family perspective, much of this, understandably, looks far less like an opportunity to exercise more financial responsibility, and a good deal more like a frightening acceleration of the wholesale shift of financial risk onto their already overburdened shoulders. The financial fallout has begun, and the political fallout may not be far behind.
\\31.	Today’s double-income families are at greater financial risk in that\\$[A]$ the safety net they used to enjoy has disappeared.\\$[B]$ their chances of being laid off have greatly increased.\\$[C]$ they are more vulnerable to changes in family economics.\\$[D]$ they are deprived of unemployment or disability insurance.\\\\32.	As a result of President Bush’s reform, retired people may have\\$[A]$ a higher sense of security.\\$[B]$ less secured payments.\\$[C]$ less chance to invest.\\$[D]$ a guaranteed future.\\\\33.	According to the author, health-savings plans will\\$[A]$ help reduce the cost of healthcare.\\$[B]$ popularize among the middle class.\\$[C]$ compensate for the reduced pensions.\\$[D]$ increase the families’ investment risk.\\\\34.	It can be inferred from the last paragraph that\\$[A]$ financial risks tend to outweigh political risks.\\$[B]$ the middle class may face greater political challenges.\\$[C]$ financial problems may bring about political problems.\\$[D]$ financial responsibility is an indicator of political status.\\\\35.	Which of the following is the best title for this text?\\$[A]$ The Middle Class on the Alert\\$[B]$ The Middle Class on the Cliff\\$[C]$ The Middle Class in Conflict\\$[D]$ The Middle Class in Ruins\\\subsection{2008年}
\subsubsection{Text1}

\par
While still catching-up to men in some spheres of modern life, women appear to be way ahead in at least one undesirable category. “Women are particularly susceptible to developing depression and anxiety disorders in response to stress compared to men,” according to Dr. Yehuda, chief psychiatrist at New York’s Veteran’s Administration Hospital.

\par
Studies of both animals and humans have shown that sex hormones somehow affect the stress response, causing females under stress to produce more of the trigger chemicals than do males under the same conditions. In several of the studies, when stressed-out female rats had their ovaries (the female reproductive organs) removed, their chemical responses became equal to those of the males.

\par
Adding to a woman’s increased dose of stress chemicals, are her increased “opportunities” for stress. “It’s not necessarily that women don’t cope as well. It’s just that they have so much more to cope with,” says Dr. Yehuda. “Their capacity for tolerating stress may even be greater than men’s,” she observes, “it’s just that they’re dealing with so many more things that they become worn out from it more visibly and sooner.”

\par
Dr. Yehuda notes another difference between the sexes. “I think that the kinds of things that women are exposed to tend to be in more of a chronic or repeated nature. Men go to war and are exposed to combat stress. Men are exposed to more acts of random physical violence. The kinds of interpersonal violence that women are exposed to tend to be in domestic situations, by, unfortunately, parents or other family members, and they tend not to be one-shot deals. The wear-and-tear that comes from these longer relationships can be quite devastating.”

\par
Adeline Alvarez married at 18 and gave birth to a son, but was determined to finish college. “I struggled a lot to get the college degree. I was living in so much frustration that that was my escape, to go to school, and get ahead and do better.” Later, her marriage ended and she became a single mother. “It’s the hardest thing to take care of a teenager, have a job, pay the rent, pay the car payment, and pay the debt. I lived from paycheck to paycheck.”

\par
Not everyone experiences the kinds of severe chronic stresses Alvarez describes. But most women today are coping with a lot of obligations, with few breaks, and feeling the strain. Alvarez’s experience demonstrates the importance of finding ways to diffuse stress before it threatens your health and your ability to function.
\\21.	Which of the following is true according to the first two paragraphs?\\$[A]$ Women are biologically more vulnerable to stress.\\$[B]$ Women are still suffering much stress caused by men.\\$[C]$ Women are more experienced than men in coping with stress.\\$[D]$ Men and women show different inclinations when faced with stress.\\\\22.	Dr. Yehuda’s research suggests that women\\$[A]$ need extra doses of chemicals to handle stress.\\$[B]$ have limited capacity for tolerating stress.\\$[C]$ are more capable of avoiding stress.\\$[D]$ are exposed to more stress.\\\\23.	According to Paragraph 4, the stress women confront tends to be\\$[A]$ domestic and temporary.\\$[B]$ irregular and violent.\\$[C]$ durable and frequent.\\$[D]$ trivial and random.\\\\24.	The sentence “I lived from paycheck to paycheck.” (Line 6, Para. 5) shows that\\$[A]$ Alvarez cared about nothing but making money.\\$[B]$ Alvarez’s salary barely covered her household expenses.\\$[C]$ Alvarez got paychecks from different jobs.\\$[D]$ Alvarez paid practically everything by check.\\\\25.	Which of the following would be the best title for the text?\\$[A]$ Strain of Stress: No Way Out?\\$[B]$ Responses to Stress: Gender Difference\\$[C]$ Stress Analysis: What Chemicals Say\\$[D]$ Gender Inequality: Women Under Stress\\\subsubsection{Text2}

\par
It used to be so straightforward. A team of researchers working together in the laboratory would submit the results of their research to a journal. A journal editor would then remove the authors’ names and affiliations from the paper and send it to their peers for review. Depending on the comments received, the editor would accept the paper for publication or decline it. Copyright rested with the journal publisher, and researchers seeking knowledge of the results would have to subscribe to the journal.

\par
No longer. The Internet – and pressure from funding agencies, who are questioning why commercial publishers are making money from government-funded research by restricting access to it – is making access to scientific results a reality. The Organization for Economic Co-operation and Development (OECD) has just issued a report describing the far-reaching consequences of this. The report, by John Houghton of Victoria University in Australia and Graham Vickery of the OECD, makes heavy reading for publishers who have, so far, made handsome profits. But it goes further than that. It signals a change in what has, until now, been a key element of scientific endeavor.

\par
The value of knowledge and the return on the public investment in research depends, in part, upon wide distribution and ready access. It is big business. In America, the core scientific publishing market is estimated at between \$7 billion and \$11 billion. The International Association of Scientific, Technical and Medical Publishers says that there are more than 2,000 publishers worldwide specializing in these subjects. They publish more than 1.2 million articles each year in some 16,000 journals.

\par
This is now changing. According to the OECD report, some 75\% of scholarly journals are now online. Entirely new business models are emerging; three main ones were identified by the report’s authors. There is the so-called big deal, where institutional subscribers pay for access to a collection of online journal titles through site-licensing agreements. There is open-access publishing, typically supported by asking the author (or his employer) to pay for the paper to be published. Finally, there are open-access archives, where organizations such as universities or international laboratories support institutional repositories. Other models exist that are hybrids of these three, such as delayed open-access, where journals allow only subscribers to read a paper for the first six months, before making it freely available to everyone who wishes to see it. All this could change the traditional form of the peer-review process, at least for the publication of papers.
\\26.	In the first paragraph, the author discusses\\$[A]$ the background information of journal editing.\\$[B]$ the publication routine of laboratory reports.\\$[C]$ the relations of authors with journal publishers.\\$[D]$ the traditional process of journal publication.\\\\27.	Which of the following is true of the OECD report?\\$[A]$ It criticizes government-funded research.\\$[B]$ It introduces an effective means of publication.\\$[C]$ It upsets profit-making journal publishers.\\$[D]$ It benefits scientific research considerably.\\\\28.	According to the text, online publication is significant in that\\$[A]$ it provides an easier access to scientific results.\\$[B]$ it brings huge profits to scientific researchers.\\$[C]$ it emphasizes the crucial role of scientific knowledge.\\$[D]$ it facilitates public investment in scientific research.\\\\29.	With the open-access publishing model, the author of a paper is required to\\$[A]$ cover the cost of its publication.\\$[B]$ subscribe to the journal publishing it.\\$[C]$ allow other online journals to use it freely.\\$[D]$ complete the peer-review before submission.\\\\30.	Which of the following best summarizes the text?\\$[A]$ The Internet is posing a threat to publishers.\\$[B]$ A new mode of publication is emerging.\\$[C]$ Authors welcome the new channel for publication.\\$[D]$ Publication is rendered easier by online service.\\\section{Part 3}
\subsection{1995年}
\subsubsection{Text1}

\par
Money spent on advertising is money spent as well as any I know of. It serves directly to assist a rapid distribution of goods at reasonable prices, thereby establishing a firm home market and so making it possible to provide for export at competitive prices. By drawing attention to new ideas it helps enormously to raise standards of living. By helping to increase demand it ensures an increased need for labour, and is therefore an effective way to fight unemployment. It lowers the costs of many services: without advertisements your daily newspaper would cost four times as much, the price of your television license would need to be doubled and travel by bus or tube would cost 20 per cent more.

\par
And perhaps most important of all, advertising provides a guarantee of reasonable value in the products and services you buy. Apart from the fact that twenty-seven Acts of Parliament govern the terms of advertising, no regular advertiser dare promote a product that fails to live up to the promise of his advertisements. He might fool some people for a little while through misleading advertising. He will not do so for long, for mercifully the public has the good sense not to buy the inferior article more than once. If you see an article consistently advertised, it is the surest proof I know that the article does what is claimed for it, and that it represents good value.

\par
Advertising does more for the material benefit of the community than any other force I can think of.

\par
There is one point I feel I ought to touch on. Recently I heard a well-known television personality declare that he was against advertising because it persuades rather than informs. He was drawing excessively fine distinctions. Of course advertising seeks to persuade.

\par
If its message were confined merely to information -- and that in itself would be difficult if not impossible to achieve, for even a detail such as the choice of the colour of a shirt is subtly persuasive -- advertising would be so boring that no one would pay any attention. But perhaps that is what the well-known television personality wants.
\\51.	By the first sentence of the passage the author means that \underline{\hbox to 30mm{}}.\\$[A]$ he is fairly familiar with the cost of advertising\\$[B]$ everybody knows well that advertising is money consuming\\$[C]$ advertising costs money like everything else\\$[D]$ it is worthwhile to spend money on advertising\\\\52.	In the passage, which of the following is NOT included in the advantages of advertising?\\$[A]$ Securing greater fame.\\$[B]$ Providing more jobs.\\$[C]$ Enhancing living standards.\\$[D]$ Reducing newspaper cost.\\\\53.	The author deems that the well-known TV personality is \underline{\hbox to 30mm{}}.\\$[A]$ very precise in passing his judgment on advertising\\$[B]$ interested in nothing but the buyers’ attention\\$[C]$ correct in telling the difference between persuasion and information\\$[D]$ obviously partial in his views on advertising\\\\54.	In the author’s opinion, \underline{\hbox to 30mm{}}.\\$[A]$ advertising can seldom bring material benefit to man by providing information\\$[B]$ advertising informs people of new ideas rather than wins them over\\$[C]$ there is nothing wrong with advertising in persuading the buyer\\$[D]$ the buyer is not interested in getting information from an advertisement\\\subsubsection{Text3}

\par
In such a changing, complex society formerly simple solutions to informational needs become complicated. Many of life’s problems which were solved by asking family members, friends or colleagues are beyond the capability of the extended family to resolve. Where to turn for expert information and how to determine which expert advice to accept are questions facing many people today.

\par
In addition to this, there is the growing mobility of people since World War II. As families move away from their stable community, their friends of many years, their extended family relationships, the informal flow of information is cut off, and with it the confidence that information will be available when needed and will be trustworthy and reliable. The almost unconscious flow of information about the simplest aspects of living can be cut off. Thus, things once learned subconsciously through the casual communications of the extended family must be consciously learned.

\par
Adding to societal changes today is an enormous stockpile of information. The individual now has more information available than any generation, and the task of finding that one piece of information relevant to his or her specific problem is complicated, time-consuming and sometimes even overwhelming.

\par
Coupled with the growing quantity of information is the development of technologies which enable the storage and delivery of more information with greater speed to more locations than has ever been possible before. Computer technology makes it possible to store vast amounts of data in machine-readable files, and to program computers to locate specific information. Telecommunications developments enable the sending of messages via television, radio, and very shortly, electronic mail to bombard people with multitudes of messages. Satellites have extended the power of communications to report events at the instant of occurrence. Expertise can be shared worldwide through teleconferencing, and problems in dispute can be settled without the participants leaving their homes and/or jobs to travel to a distant conference site. Technology has facilitated the sharing of information and the storage and delivery of information, thus making more information available to more people.

\par
In this world of change and complexity, the need for information is of greatest importance. Those people who have accurate, reliable up-to-date information to solve the day-to-day problems, the critical problems of their business, social and family life, will survive and succeed. “Knowledge is power” may well be the truest saying and access to information may be the most critical requirement of all people.
\\59.	The word “it” (Line 3, Para. 2) most probably refers to \underline{\hbox to 30mm{}}.\\$[A]$ the lack of stable communities\\$[B]$ the breakdown of informal information channels\\$[C]$ the increased mobility of families\\$[D]$ the growing number of people moving from place to place\\\\60.	The main problem people may encounter today arises from the fact that \underline{\hbox to 30mm{}}.\\$[A]$ they have to learn new things consciously\\$[B]$ they lack the confidence of securing reliable and trustworthy information\\$[C]$ they have difficulty obtaining the needed information readily\\$[D]$ they can hardly carry out casual communications with an extended family\\\\61.	From the passage we can infer that \underline{\hbox to 30mm{}}.\\$[A]$ electronic mail will soon play a dominant role in transmitting messages\\$[B]$ it will become more difficult for people to keep secrets in an information era\\$[C]$ people will spend less time holding meetings or conferences\\$[D]$ events will be reported on the spot mainly through satellites\\\\62.	We can learn from the last paragraph that \underline{\hbox to 30mm{}}.\\$[A]$ it is necessary to obtain as much knowledge as possible\\$[B]$ people should make the best use of the information accessible\\$[C]$ we should realize the importance of accumulating information\\$[D]$ it is of vital importance to acquire needed information efficiently\\\subsubsection{Text4}

\par
Personality is to a large extent inherent -- A-type parents usually bring about A-type offspring. But the environment must also have a profound effect, since if competition is important to the parents, it is likely to become a major factor in the lives of their children.

\par
One place where children soak up A-characteristics is school, which is, by its very nature, a highly competitive institution. Too many schools adopt the “win at all costs” moral standard and measure their success by sporting achievements. The current passion for making children compete against their classmates or against the clock produces a two-layer system, in which competitive A-types seem in some way better than their B-type fellows. Being too keen to win can have dangerous consequences: remember that Pheidippides, the first marathon runner, dropped dead seconds after saying: “Rejoice, we conquer!”

\par
By far the worst form of competition in schools is the disproportionate emphasis on examinations. It is a rare school that allows pupils to concentrate on those things they do well. The merits of competition by examination are somewhat questionable, but competition in the certain knowledge of failure is positively harmful.

\par
Obviously, it is neither practical nor desirable that all A youngsters change into B’s. The world needs A types, and schools have an important duty to try to fit a child’s personality to his possible future employment. It is top management.

\par
If the preoccupation of schools with academic work was lessened, more time might be spent teaching children surer values. Perhaps selection for the caring professions, especially medicine, could be made less by good grades in chemistry and more by such considerations as sensitivity and sympathy. It is surely a mistake to choose our doctors exclusively from A-type stock. B’s are important and should be encouraged.
\\63.	According to the passage, A-type individuals are usually \underline{\hbox to 30mm{}}.\\$[A]$ impatient\\$[B]$ considerate\\$[C]$ aggressive\\$[D]$ agreeable\\\\64.	The author is strongly opposed to the practice of examinations at schools because \underline{\hbox to 30mm{}}.\\$[A]$ the pressure is too great on the students\\$[B]$ some students are bound to fail\\$[C]$ failure rates are too high\\$[D]$ the results of exanimations are doubtful\\\\65.	The selection of medical professionals is currently based on \underline{\hbox to 30mm{}}.\\$[A]$ candidates’ sensitivity\\$[B]$ academic achievements\\$[C]$ competitive spirit\\$[D]$ surer values\\\\66.	From the passage we can draw the conclusion that \underline{\hbox to 30mm{}}.\\$[A]$ the personality of a child is well established at birth\\$[B]$ family influence dominates the shaping of one’s characteristics\\$[C]$ the development of one’s personality is due to multiple factors\\$[D]$ B-type characteristics can find no place in competitive society\\\subsection{1996年}
\subsubsection{Text3}

\par
In the last half of the nineteenth century “capital” and “labour” were enlarging and perfecting their rival organizations on modern lines. Many an old firm was replaced by a limited liability company with a bureaucracy of salaried managers. The change met the technical requirements of the new age by engaging a large professional element and prevented the decline in efficiency that so commonly spoiled the fortunes of family firms in the second and third generation after the energetic founders. It was moreover a step away from individual initiative, towards collectivism and municipal and state-owned business. The railway companies, though still private business managed for the benefit of shareholders, were very unlike old family business. At the same time the great municipalities went into business to supply lighting, trams and other services to the taxpayers.

\par
The growth of the limited liability company and municipal business had important consequences. Such large, impersonal manipulation of capital and industry greatly increased the numbers and importance of shareholders as a class, an element in national life representing irresponsible wealth detached from the land and the duties of the landowners; and almost equally detached from the responsible management of business. All through the nineteenth century, America, Africa, India, Australia and parts of Europe were being developed by British capital, and British shareholders were thus enriched by the world’s movement towards industrialization. Towns like Bournemouth and Eastbourne sprang up to house large “comfortable” classes who had retired on their incomes, and who had no relation to the rest of the community except that of drawing dividends and occasionally attending a shareholders’ meeting to dictate their orders to the management. On the other hand “shareholding” meant leisure and freedom which was used by

\par
many of the later Victorians for the highest purpose of a great civilization.

\par
The “shareholders” as such had no knowledge of the lives, thoughts or needs of the workmen employed by the company in which he held shares, and his influence on the relations of capital and labour was not good. The paid manager acting for the company was in more direct relation with the men and their demands, but even he had seldom that familiar personal knowledge of the workmen which the employer had often had under the more patriarchal system of the old family business now passing away. Indeed the mere size of operations and the numbers of workmen involved rendered such personal relations impossible. Fortunately, however, the increasing power and organization of the trade unions, at least in all skilled trades, enabled the workmen to meet on equal terms the managers of the companies who employed them. The cruel discipline of the strike and lockout taught the two parties to respect each other’s strength and understand the value of fair negotiation.
\\59.	It’s true of the old family firms that \underline{\hbox to 30mm{}}.\\$[A]$ they were spoiled by the younger generations\\$[B]$ they failed for lack of individual initiative\\$[C]$ they lacked efficiency compared with modern companies\\$[D]$ they could supply adequate services to the taxpayers\\\\60.	The growth of limited liability companies resulted in \underline{\hbox to 30mm{}}.\\$[A]$ the separation of capital from management\\$[B]$ the ownership of capital by managers\\$[C]$ the emergence of capital and labour as two classes\\$[D]$ the participation of shareholders in municipal business\\\\61.	According to the passage, all of the following are true EXCEPT that \underline{\hbox to 30mm{}}.\\$[A]$ the shareholders were unaware of the needs of the workers\\$[B]$ the old firm owners had a better understanding of their workers\\$[C]$ the limited liability companies were too large to run smoothly\\$[D]$ the trade unions seemed to play a positive role\\\\62.	The author is most critical of \underline{\hbox to 30mm{}}.\\$[A]$ family film owners\\$[B]$ landowners\\$[C]$ managers\\$[D]$ shareholders\\\subsection{1997年}
\subsubsection{Text4}

\par
No company likes to be told it is contributing to the moral decline of a nation. “Is this what you intended to accomplish with your careers?” Senator Robert Dole asked Time Warner executives last week. “You have sold your souls, but must you corrupt our nation and threaten our children as well?” At Time Warner, however, such questions are simply the latest manifestation of the soul-searching that has involved the company ever since the company was born in 1990. It’s a self-examination that has, at various times, involved issues of responsibility, creative freedom and the corporate bottom line.

\par
At the core of this debate is chairman Gerald Levin, 56, who took over for the late Steve Ross in 1992. On the financial front, Levin is under pressure to raise the stock price and reduce the company’s mountainous debt, which will increase to \$17.3 billion after two new cable deals close. He has promised to sell off some of the property and restructure the company, but investors are waiting impatiently.

\par
The flap over rap is not making life any easier for him. Levin has consistently defended the company’s rap music on the grounds of expression. In 1992, when Time Warner was under fire for releasing Ice-T’s violent rap song Cop Killer, Levin described rap as a lawful expression of street culture, which deserves an outlet. “The test of any democratic society,” he wrote in a Wall Street Journal column, “lies not in how well it can control expression but in whether it gives freedom of thought and expression the widest possible latitude, however disputable or irritating the results may sometimes be. We won’t retreat in the face of any threats.”

\par
Levin would not comment on the debate last week, but there were signs that the chairman was backing off his hard-line stand, at least to some extent. During the discussion of rock singing verses at last month’s stockholders’ meeting, Levin asserted that “music is not the cause of society’s ills” and even cited his son, a teacher in the Bronx, New York, who uses rap to communicate with students. But he talked as well about the “balanced struggle” between creative freedom and social responsibility, and he announced that the company would launch a drive to develop standards for distribution and labeling of potentially objectionable music.

\par
The 15-member Time Warner board is generally supportive of Levin and his corporate strategy. But insiders say several of them have shown their concerns in this matter. “Some of us have known for many, many years that the freedoms under the First Amendment are not totally unlimited,” says Luce. “I think it is perhaps the case that some people associated with the company have only recently come to realize this.”
\\63.	Senator Robert Dole criticized Time Warner for \underline{\hbox to 30mm{}}.\\$[A]$ its raising of the corporate stock price\\$[B]$ its self-examination of soul\\$[C]$ its neglect of social responsibility\\$[D]$ its emphasis on creative freedom\\\\64.	According to the passage, which of the following is TRUE?\\$[A]$ Luce is a spokesman of Time Warner.\\$[B]$ Gerald Levin is liable to compromise.\\$[C]$ Time Warner is united as one in the face of the debate.\\$[D]$ Steve Ross is no longer alive.\\\\65.	In face of the recent attacks on the company, the chairman \underline{\hbox to 30mm{}}.\\$[A]$ stuck to a strong stand to defend freedom of expression\\$[B]$ softened his tone and adopted some new policy\\$[C]$ changed his attitude and yielded to objection\\$[D]$ received more support from the 15-member board\\\\66.	The best title for this passage could be \underline{\hbox to 30mm{}}.\\$[A]$ A Company under Fire\\$[B]$ A Debate on Moral Decline\\$[C]$ A Lawful Outlet of Street Culture\\$[D]$ A Form of Creative Freedom\\\subsection{1999年}
\subsubsection{Text4}

\par
When a Scottish research team startled the world by revealing 3 months ago that it had cloned an adult sheep, President Clinton moved swiftly. Declaring that he was opposed to using this unusual animal husbandry technique to clone humans, he ordered that federal funds not be used for such an experiment -- although no one had proposed to do so -- and asked an independent panel of experts chaired by Princeton President Harold Shapiro to report back to the White House in 90 days with recommendations for a national policy on human cloning. That group -- the National Bioethics Advisory Commission (NBAC) -- has been working feverishly to put its wisdom on paper, and at a meeting on 17 May, members agreed on a near-final draft of their recommendations.

\par
NBAC will ask that Clinton’s 90-day ban on federal funds for human cloning be extended indefinitely, and possibly that it be made law. But NBAC members are planning to word the recommendation narrowly to avoid new restrictions on research that involves the cloning of human DNA or cells -- routine in molecular biology. The panel has not yet reached agreement on a crucial question, however, whether to recommend legislation that would make it a crime for private funding to be used for human cloning.

\par
In a draft preface to the recommendations, discussed at the 17 May meeting, Shapiro suggested that the panel had found a broad consensus that it would be “morally unacceptable to attempt to create a human child by adult nuclear cloning.” Shapiro explained during the meeting that the moral doubt stems mainly from fears about the risk to the health of the child. The panel then informally accepted several general conclusions, although some details have not been settled.

\par
NBAC plans to call for a continued ban on federal government funding for any attempt to clone body cell nuclei to create a child. Because current federal law already forbids the use of federal funds to create embryos (the earliest stage of human offspring before birth) for research or to knowingly endanger an embryo’s life, NBAC will remain silent on embryo research.

\par
NBAC members also indicated that they will appeal to privately funded researchers and clinics not to try to clone humans by body cell nuclear transfer. But they were divided on whether to go further by calling for a federal law that would impose a complete ban on human cloning. Shapiro and most members favored an appeal for such legislation, but in a phone interview, he said this issue was still “up in the air.”
\\63.	We can learn from the first paragraph that \underline{\hbox to 30mm{}}.\\$[A]$ federal funds have been used in a project to clone humans\\$[B]$ the White House responded strongly to the news of cloning\\$[C]$ NBAC was authorized to control the misuse of cloning technique\\$[D]$ the White House has got the panel’s recommendations on cloning\\\\64.	The panel agreed on all of the following except that \underline{\hbox to 30mm{}}.\\$[A]$ the ban on federal funds for human cloning should be made a law\\$[B]$ the cloning of human DNA is not to be put under more control\\$[C]$ it is criminal to use private funding for human cloning\\$[D]$ it would be against ethical values to clone a human being\\\\65.	NBAC will leave the issue of embryo research undiscussed because \underline{\hbox to 30mm{}}.\\$[A]$ embryo research is just a current development of cloning\\$[B]$ the health of the child is not the main concern of embryo research\\$[C]$ an embryo’s life will not be endangered in embryo research\\$[D]$ the issue is explicitly stated and settled in the law\\\\66.	It can be inferred from the last paragraph that \underline{\hbox to 30mm{}}.\\$[A]$ some NBAC members hesitate to ban human cloning completely\\$[B]$ a law banning human cloning is to be passed in no time\\$[C]$ privately funded researchers will respond positively to NBAC’s appeal\\$[D]$ the issue of human cloning will soon be settled\\\subsection{2000年}
\subsubsection{Text2}

\par
Being a man has always been dangerous. There are about 105 males born for every 100 females, but this ratio drops to near balance at the age of maturity, and among 70-year-olds there are twice as many women as men. But the great universal of male mortality is being changed. Now, boy babies survive almost as well as girls do. This means that, for the first time, there will be an excess of boys in those crucial years when they are searching for a mate. More important, another chance for natural selection has been removed. Fifty years ago, the chance of a baby (particularly a boy baby) surviving depended on its weight. A kilogram too light or too heavy meant almost certain death. Today it makes almost no difference. Since much of the variation is due to genes, one more agent of evolution has gone.

\par
There is another way to commit evolutionary suicide: stay alive, but have fewer children. Few people are as fertile as in the past. Except in some religious communities, very few women have 15 children. Nowadays the number of births, like the age of death, has become average. Most of us have roughly the same number of offspring. Again, differences between people and the opportunity for natural selection to take advantage of it have diminished. India shows what is happening. The country offers wealth for a few in the great cities and poverty for the remaining tribal peoples. The grand mediocrity of today -- everyone being the same in survival and number of offspring -- means that natural selection has lost 80\% of its power in upper-middle-class India compared to the tribes.

\par
For us, this means that evolution is over; the biological Utopia has arrived. Strangely, it has involved little physical change. No other species fills so many places in nature. But in the pass 100,000 years -- even the pass 100 years -- our lives have been transformed but our bodies have not. We did not evolve, because machines and society did it for us. Darwin had a phrase to describe those ignorant of evolution: they “look at an organic being as a savage looks at a ship, as at something wholly beyond his comprehension.” No doubt we will remember a 20th century way of life beyond comprehension for its ugliness. But however amazed our descendants may be at how far from Utopia we were, they will look just like us.
\\55.	What used to be the danger in being a man according to the first paragraph?\\$[A]$ A lack of mates.\\$[B]$ A fierce competition.\\$[C]$ A lower survival rate.\\$[D]$ A defective gene.\\\\56.	What does the example of India illustrate?\\$[A]$ Wealthy people tend to have fewer children than poor people.\\$[B]$ Natural selection hardly works among the rich and the poor.\\$[C]$ The middle class population is 80\% smaller than that of the tribes.\\$[D]$ India is one of the countries with a very high birth rate.\\\\57.	The author argues that our bodies have stopped evolving because \underline{\hbox to 30mm{}}.\\$[A]$ life has been improved by technological advance\\$[B]$ the number of female babies has been declining\\$[C]$ our species has reached the highest stage of evolution\\$[D]$ the difference between wealth and poverty is disappearing\\\\58.	Which of the following would be the best title for the passage?\\$[A]$ Sex Ratio Changes in Human Evolution\\$[B]$ Ways of Continuing Man’s Evolution\\$[C]$ The Evolutionary Future of Nature\\$[D]$ Human Evolution Going Nowhere\\\subsection{2001年}
\subsubsection{Text3}

\par
Why do so many Americans distrust what they read in their newspapers? The American Society of Newspaper Editors is trying to answer this painful question. The organization is deep into a long self-analysis known as the journalism credibility project.

\par
Sad to say, this project has turned out to be mostly low-level findings about factual errors and spelling and grammar mistakes, combined with lots of head-scratching puzzlement about what in the world those readers really want.

\par
But the sources of distrust go way deeper. Most journalists learn to see the world through a set of standard templates (patterns) into which they plug each day’s events. In other words, there is a conventional story line in the newsroom culture that provides a backbone and a ready-made narrative structure for otherwise confusing news.

\par
There exists a social and cultural disconnect between journalists and their readers, which helps explain why the “standard templates” of the newsroom seem alien to many readers. In a recent survey, questionnaires were sent to reporters in five middle-size cities around the country, plus one large metropolitan area. Then residents in these communities were phoned at random and asked the same questions.

\par
Replies show that compared with other Americans, journalists are more likely to live in upscale neighborhoods, have maids, own Mercedeses, and trade stocks, and they’re less likely to go to church, do volunteer work, or put down roots in a community.

\par
Reporters tend to be part of a broadly defined social and cultural elite, so their work tends to reflect the conventional values of this elite. The astonishing distrust of the news media isn’t rooted in inaccuracy or poor reportorial skills but in the daily clash of world views between reporters and their readers.

\par
This is an explosive situation for any industry, particularly a declining one. Here is a troubled business that keeps hiring employees whose attitudes vastly annoy the customers. Then it sponsors lots of symposiums and a credibility project dedicated to wondering why customers are annoyed and fleeing in large numbers. But it never seems to get around to noticing the cultural and class biases that so many former buyers are complaining about. If it did, it would open up its diversity program, now focused narrowly on race and gender, and look for reporters who differ broadly by outlook, values, education, and class.
\\59.	What is the passage mainly about?\\$[A]$ needs of the readers all over the world\\$[B]$ causes of the public disappointment about newspapers\\$[C]$ origins of the declining newspaper industry\\$[D]$ aims of a journalism credibility project\\\\60.	The results of the journalism credibility project turned out to be \underline{\hbox to 30mm{}}.\\$[A]$ quite trustworthy\\$[B]$ somewhat contradictory\\$[C]$ very illuminating\\$[D]$ rather superficial\\\\61.	The basic problem of journalists as pointed out by the writer lies in their \underline{\hbox to 30mm{}}.\\$[A]$ working attitude\\$[B]$ conventional lifestyle\\$[C]$ world outlook\\$[D]$ educational background\\\\62.	Despite its efforts, the newspaper industry still cannot satisfy the readers owing to its \underline{\hbox to 30mm{}}.\\$[A]$ failure to realize its real problem\\$[B]$ tendency to hire annoying reporters\\$[C]$ likeliness to do inaccurate reporting\\$[D]$ prejudice in matters of race and gender\\\subsection{2002年}
\subsubsection{Text2}

\par
Since the dawn of human ingenuity, people have devised ever more cunning tools to cope with work that is dangerous, boring, burdensome, or just plain nasty. That compulsion has resulted in robotics -- the science of conferring various human capabilities on machines. And if scientists have yet to create the mechanical version of science fiction, they have begun to come close.

\par
As a result, the modern world is increasingly populated by intelligent gizmos whose presence we barely notice but whose universal existence has removed much human labor. Our factories hum to the rhythm of robot assembly arms. Our banking is done at automated teller terminals that thank us with mechanical politeness for the transaction. Our subway trains are controlled by tireless robot-drivers. And thanks to the continual miniaturization of electronics and micro-mechanics, there are already robot systems that can perform some kinds of brain and bone surgery with submillimeter accuracy -- far greater precision than highly skilled physicians can achieve with their hands alone.

\par
But if robots are to reach the next stage of laborsaving utility, they will have to operate with less human supervision and be able to make at least a few decisions for themselves -- goals that pose a real challenge. “While we know how to tell a robot to handle a specific error,” says Dave Lavery, manager of a robotics program at NASA, “we can’t yet give a robot enough ‘common sense’ to reliably interact with a dynamic world.”

\par
Indeed the quest for true artificial intelligence has produced very mixed results. Despite a spell of initial optimism in the 1960s and 1970s when it appeared that transistor circuits and microprocessors might be able to copy the action of the human brain by the year 2010, researchers lately have begun to extend that forecast by decades if not centuries.

\par
What they found, in attempting to model thought, is that the human brain’s roughly one hundred billion nerve cells are much more talented -- and human perception far more complicated -- than previously imagined. They have built robots that can recognize the error of a machine panel by a fraction of a millimeter in a controlled factory environment. But the human mind can glimpse a rapidly changing scene and immediately disregard the 98 percent that is irrelevant, instantaneously focusing on the monkey at the side of a winding forest road or the single suspicious face in a big crowd. The most advanced computer systems on Earth can’t approach that kind of ability, and neuroscientists still don’t know quite how we do it.
\\46.	Human ingenuity was initially demonstrated in \underline{\hbox to 30mm{}}.\\$[A]$ the use of machines to produce science fiction\\$[B]$ the wide use of machines in manufacturing industry\\$[C]$ the invention of tools for difficult and dangerous work\\$[D]$ the elite’s cunning tackling of dangerous and boring work\\\\47.	The word “gizmos” (Line 1, Paragraph 2) most probably means \underline{\hbox to 30mm{}}.\\$[A]$ programs\\$[B]$ experts\\$[C]$ devices\\$[D]$ creatures\\\\48.	According to the text, what is beyond man’s ability now is to design a robot that can \underline{\hbox to 30mm{}}.\\$[A]$ fulfill delicate tasks like performing brain surgery\\$[B]$ interact with human beings verbally\\$[C]$ have a little common sense\\$[D]$ respond independently to a changing world\\\\49.	Besides reducing human labor, robots can also \underline{\hbox to 30mm{}}.\\$[A]$ make a few decisions for themselves\\$[B]$ deal with some errors with human intervention\\$[C]$ improve factory environments\\$[D]$ cultivate human creativity\\\\50.	The author uses the example of a monkey to argue that robots are \underline{\hbox to 30mm{}}.\\$[A]$ expected to copy human brain in internal structure\\$[B]$ able to perceive abnormalities immediately\\$[C]$ far less able than human brain in focusing on relevant information\\$[D]$ best used in a controlled environment\\\subsection{2003年}
\subsubsection{Text3}

\par
In recent years, railroads have been combining with each other, merging into supersystems, causing heightened concerns about monopoly. As recently as 1995, the top four railroads accounted for under 70 percent of the total ton-miles moved by rails. Next year, after a series of mergers is completed, just four railroads will control well over 90 percent of all the freight moved by major rail carriers.

\par
Supporters of the new supersystems argue that these mergers will allow for substantial cost reductions and better coordinated service. Any threat of monopoly, they argue, is removed by fierce competition from trucks. But many shippers complain that for heavy bulk commodities traveling long distances, such as coal, chemicals, and grain, trucking is too costly and the railroads therefore have them by the throat.

\par
The vast consolidation within the rail industry means that most shippers are served by only one rail company. Railroads typically charge such “captive” shippers 20 to 30 percent more than they do when another railroad is competing for the business. Shippers who feel they are being overcharged have the right to appeal to the federal government’s Surface Transportation Board for rate relief, but the process is expensive, time-consuming, and will work only in truly extreme cases.

\par
Railroads justify rate discrimination against captive shippers on the grounds that in the long run it reduces everyone’s cost. If railroads charged all customers the same average rate, they argue, shippers who have the option of switching to trucks or other forms of transportation would do so, leaving remaining customers to shoulder the cost of keeping up the line. It’s a theory to which many economists subscribe, but in practice it often leaves railroads in the position of determining which companies will flourish and which will fail. “Do we really want railroads to be the arbiters of who wins and who loses in the marketplace?” asks Martin Bercovici, a Washington lawyer who frequently represents shippers.

\par
Many captive shippers also worry they will soon be hit with a round of huge rate increases. The railroad industry as a whole, despite its brightening fortunes, still does not earn enough to cover the cost of the capital it must invest to keep up with its surging traffic. Yet railroads continue to borrow billions to acquire one another, with Wall Street cheering them on. Consider the \$10.2 billion bid by Norfolk Southern and CSX to acquire Conrail this year. Conrail’s net railway operating income in 1996 was just \$427 million, less than half of the carrying costs of the transaction. Who’s going to pay for the rest of the bill? Many captive shippers fear that they will, as Norfolk Southern and CSX increase their grip on the market.
\\51.	According to those who support mergers, railway monopoly is unlikely because \underline{\hbox to 30mm{}}.\\$[A]$ cost reduction is based on competition\\$[B]$ services call for cross-trade coordination\\$[C]$ outside competitors will continue to exist\\$[D]$ shippers will have the railway by the throat\\\\52.	What is many captive shippers’ attitude towards the consolidation in the rail industry?\\$[A]$ Indifferent.\\$[B]$ Supportive.\\$[C]$ Indignant.\\$[D]$ Apprehensive.\\\\53.	It can be inferred from Paragraph 3 that \underline{\hbox to 30mm{}}.\\$[A]$ shippers will be charged less without a rival railroad\\$[B]$ there will soon be only one railroad company nationwide\\$[C]$ overcharged shippers are unlikely to appeal for rate relief\\$[D]$ a government board ensures fair play in railway business\\\\54.	The word “arbiters” (Line 7, Paragraph 4) most probably refers to those \underline{\hbox to 30mm{}}.\\$[A]$ who work as coordinators\\$[B]$ who function as judges\\$[C]$ who supervise transactions\\$[D]$ who determine the price\\\\55.	According to the text, the cost increase in the rail industry is mainly caused by \underline{\hbox to 30mm{}}.\\$[A]$ the continuing acquisition\\$[B]$ the growing traffic\\$[C]$ the cheering Wall Street\\$[D]$ the shrinking market\\\subsection{2004年}
\subsubsection{Text1}

\par
Hunting for a job late last year, lawyer Gant Redmon stumbled across CareerBuilder, a job database on the Internet. He searched it with no success but was attracted by the site’s “personal search agent.” It’s an interactive feature that lets visitors key in job criteria such as location, title, and salary, then E-mails them when a matching position is posted in the database. Redmon chose the keywords legal, intellectual property, and Washington, D.C. Three weeks later, he got his first notification of an opening. “I struck gold,” says Redmon, who E-mailed his resume to the employer and won a position as in-house counsel for a company.

\par
With thousands of career-related sites on the Internet, finding promising openings can be time-consuming and inefficient. Search agents reduce the need for repeated visits to the databases. But although a search agent worked for Redmon, career experts see drawbacks. Narrowing your criteria, for example, may work against you: “Every time you answer a question you eliminate a possibility.” says one expert.

\par
For any job search, you should start with a narrow concept—what you think you want to do -- then broaden it. “None of these programs do that,” says another expert. “There’s no career counseling implicit in all of this.” Instead, the best strategy is to use the agent as a kind of tip service to keep abreast of jobs in a particular database; when you get E-mail, consider it a reminder to check the database again. “I would not rely on agents for finding everything that is added to a database that might interest me,” says the author of a job-searching guide.

\par
Some sites design their agents to tempt job hunters to return. When CareerSite’s agent sends out messages to those who have signed up for its service, for example, it includes only three potential jobs -- those it considers the best matches. There may be more matches in the database; job hunters will have to visit the site again to find them -- and they do. “On the day after we send our messages, we see a sharp increase in our traffic,” says Seth Peets, vice president of marketing for CareerSite.

\par
Even those who aren’t hunting for jobs may find search agents worthwhile. Some use them to keep a close watch on the demand for their line of work or gather information on compensation to arm themselves when negotiating for a raise. Although happily employed, Redmon maintains his agent at CareerBuilder. “You always keep your eyes open,” he says. Working with a personal search agent means having another set of eyes looking out for you.
\\41.	How did Redmon find his job?\\$[A]$ By searching openings in a job database.\\$[B]$ By posting a matching position in a database.\\$[C]$ By using a special service of a database.\\$[D]$ By E-mailing his resume to a database.\\\\42.	Which of the following can be a disadvantage of search agents?\\$[A]$ Lack of counseling.\\$[B]$ Limited number of visits.\\$[C]$ Lower efficiency.\\$[D]$ Fewer successful matches.\\\\43.	The expression “tip service” (Line 4, Paragraph 3) most probably means \underline{\hbox to 30mm{}}.\\$[A]$ advisory\\$[B]$ compensation\\$[C]$ interaction\\$[D]$ reminder\\\\44.	Why does CareerSite’s agent offer each job hunter only three job options?\\$[A]$ To focus on better job matches.\\$[B]$ To attract more returning visits.\\$[C]$ To reserve space for more messages.\\$[D]$ To increase the rate of success.\\\\45.	Which of the following is true according to the text?\\$[A]$ Personal search agents are indispensable to job-hunters.\\$[B]$ Some sites keep E-mailing job seekers to trace their demands.\\$[C]$ Personal search agents are also helpful to those already employed.\\$[D]$ Some agents stop sending information to people once they are employed.\\\subsection{2005年}
\subsubsection{Text1}

\par
Everybody loves a fat pay rise. Yet pleasure at your own can vanish if you learn that a colleague has been given a bigger one. Indeed, if he has a reputation for slacking, you might even be outraged. Such behaviour is regarded as “all too human,” with the underlying assumption that other animals would not be capable of this finely developed sense of grievance. But a study by Sarah Brosnan and Frans de Waal of Emory University in Atlanta, Georgia, which has just been published in Nature, suggests that it is all too monkey, as well.

\par
The researchers studied the behaviour of female brown capuchin monkeys. They look cute. They are good-natured, co-operative creatures, and they share their food readily. Above all, like their female human counterparts, they tend to pay much closer attention to the value of “goods and services” than males.

\par
Such characteristics make them perfect candidates for Dr. Brosnan’s and Dr. de Waal’s study. The researchers spent two years teaching their monkeys to exchange tokens for food. Normally, the monkeys were happy enough to exchange pieces of rock for slices of cucumber. However, when two monkeys were placed in separate but adjoining chambers, so that each could observe what the other was getting in return for its rock, their behaviour became markedly different.

\par
In the world of capuchins, grapes are luxury goods (and much preferable to cucumbers). So when one monkey was handed a grape in exchange for her token, the second was reluctant to hand hers over for a mere piece of cucumber. And if one received a grape without having to provide her token in exchange at all, the other either tossed her own token at the researcher or out of the chamber, or refused to accept the slice of cucumber. Indeed, the mere presence of a grape in the other chamber (without an actual monkey to eat it) was enough to induce resentment in a female capuchin.

\par
The researchers suggest that capuchin monkeys, like humans, are guided by social emotions. In the wild, they are a co-operative, group-living species. Such co-operation is likely to be stable only when each animal feels it is not being cheated. Feelings of righteous indignation, it seems, are not the preserve of people alone. Refusing a lesser reward completely makes these feelings abundantly clear to other members of the group. However, whether such a sense of fairness evolved independently in capuchins and humans, or whether it stems from the common ancestor that the species had 35 million years ago, is, as yet, an unanswered question.
\\21.	In the opening paragraph, the author introduces his topic by \underline{\hbox to 30mm{}}.\\$[A]$ posing a contrast\\$[B]$ justifying an assumption\\$[C]$ making a comparison\\$[D]$ explaining a phenomenon\\\\22.	The statement “it is all too monkey” (Last line, Paragraph l) implies that \underline{\hbox to 30mm{}}.\\$[A]$ monkeys are also outraged by slack rivals\\$[B]$ resenting unfairness is also monkeys’ nature\\$[C]$ monkeys, like humans, tend to be jealous of each other\\$[D]$ no animals other than monkeys can develop such emotions\\\\23.	Female capuchin monkeys were chosen for the research most probably because they are \underline{\hbox to 30mm{}}.\\$[A]$ more inclined to weigh what they get\\$[B]$ attentive to researchers’ instructions\\$[C]$ nice in both appearance and temperament\\$[D]$ more generous than their male companions\\\\24.	Dr. Brosnan and Dr. de Waal have eventually found in their study that the monkeys \underline{\hbox to 30mm{}}.\\$[A]$ prefer grapes to cucumbers\\$[B]$ can be taught to exchange things\\$[C]$ will not be co-operative if feeling cheated\\$[D]$ are unhappy when separated from others\\\\25.	What can we infer from the last paragraph?\\$[A]$ Monkeys can be trained to develop social emotions.\\$[B]$ Human indignation evolved from an uncertain source.\\$[C]$ Animals usually show their feelings openly as humans do.\\$[D]$ Cooperation among monkeys remains stable only in the wild.\\\subsubsection{Text2}

\par
Do you remember all those years when scientists argued that smoking would kill us but the doubters insisted that we didn’t know for sure? That the evidence was inconclusive, the science uncertain? That the antismoking lobby was out to destroy our way of life and the government should stay out of the way? Lots of Americans bought that nonsense, and over three decades, some 10 million smokers went to early graves.

\par
There are upsetting parallels today, as scientists in one wave after another try to awaken us to the growing threat of global warming. The latest was a panel from the National Academy of Sciences, enlisted by the White House, to tell us that the Earth’s atmosphere is definitely warming and that the problem is largely man-made. The clear message is that we should get moving to protect ourselves. The president of the National Academy, Bruce Alberts, added this key point in the preface to the panel’s report: “Science never has all the answers. But science does provide us with the best available guide to the future, and it is critical that our nation and the world base important policies on the best judgments that science can provide concerning the future consequences of present actions.”

\par
Just as on smoking, voices now come from many quarters insisting that the science about global warming is incomplete, that it’s OK to keep pouring fumes into the air until we know for sure. This is a dangerous game: by the time 100 percent of the evidence is in, it may be too late. With the risks obvious and growing, a prudent people would take out an insurance policy now.

\par
Fortunately, the White House is starting to pay attention. But it’s obvious that a majority of the president’s advisers still don’t take global warming seriously. Instead of a plan of action, they continue to press for more research -- a classic case of “paralysis by analysis.”

\par
To serve as responsible stewards of the planet, we must press forward on deeper atmospheric and oceanic research. But research alone is inadequate. If the Administration won’t take the legislative initiative, Congress should help to begin fashioning conservation measures. A bill by Democratic Senator Robert Byrd of West Virginia, which would offer financial incentives for private industry, is a promising start. Many see that the country is getting ready to build lots of new power plants to meet our energy needs. If we are ever going to protect the atmosphere, it is crucial that those new plants be environmentally sound.
\\26.	An argument made by supporters of smoking was that \underline{\hbox to 30mm{}}.\\$[A]$ there was no scientific evidence of the correlation between smoking and death\\$[B]$ the number of early deaths of smokers in the past decades was insignificant\\$[C]$ people had the freedom to choose their own way of life\\$[D]$ antismoking people were usually talking nonsense\\\\27.	According to Bruce Alberts, science can serve as \underline{\hbox to 30mm{}}.\\$[A]$ a protector\\$[B]$ a judge\\$[C]$ a critic\\$[D]$ a guide\\\\28.	What does the author mean by “paralysis by analysis” (Last line, Paragraph 4)?\\$[A]$ Endless studies kill action.\\$[B]$ Careful investigation reveals truth.\\$[C]$ Prudent planning hinders progress.\\$[D]$ Extensive research helps decision-making.\\\\29.	According to the author, what should the Administration do about global warming?\\$[A]$ Offer aid to build cleaner power plants.\\$[B]$ Raise public awareness of conservation.\\$[C]$ Press for further scientific research.\\$[D]$ Take some legislative measures.\\\\30.	The author associates the issue of global warming with that of smoking because \underline{\hbox to 30mm{}}.\\$[A]$ they both suffered from the government’s negligence\\$[B]$ a lesson from the latter is applicable to the former\\$[C]$ the outcome of the latter aggravates the former\\$[D]$ both of them have turned from bad to worse\\\subsubsection{Text3}

\par
Of all the components of a good night’s sleep, dreams seem to be least within our control. In dreams, a window opens into a world where logic is suspended and dead people speak. A century ago, Freud formulated his revolutionary theory that dreams were the disguised shadows of our unconscious desires and fears; by the late 1970s, neurologists had switched to thinking of them as just “mental noise” -- the random byproducts of the neural-repair work that goes on during sleep. Now researchers suspect that dreams are part of the mind’s emotional thermostat, regulating moods while the brain is “off-line.” And one leading authority says that these intensely powerful mental events can be not only harnessed but actually brought under conscious control, to help us sleep and feel better, “It’s your dream,” says Rosalind Cartwright, chair of psychology at Chicago’s Medical Center. “If you don’t like it, change it.”

\par
Evidence from brain imaging supports this view. The brain is as active during REM (rapid eye movement) sleep -- when most vivid dreams occur -- as it is when fully awake, says Dr, Eric Nofzinger at the University of Pittsburgh. But not all parts of the brain are equally involved; the limbic system (the “emotional brain”) is especially active, while the prefrontal cortex (the center of intellect and reasoning) is relatively quiet. “We wake up from dreams happy or depressed, and those feelings can stay with us all day.” says Stanford sleep researcher Dr. William Dement.

\par
The link between dreams and emotions shows up among the patients in Cartwright’s clinic. Most people seem to have more bad dreams early in the night, progressing toward happier ones before awakening, suggesting that they are working through negative feelings generated during the day. Because our conscious mind is occupied with daily life we don’t always think about the emotional significance of the day’s events -- until, it appears, we begin to dream.

\par
And this process need not be left to the unconscious. Cartwright believes one can exercise conscious control over recurring bad dreams. As soon as you awaken, identify what is upsetting about the dream. Visualize how you would like it to end instead; the next time it occurs, try to wake up just enough to control its course. With much practice people can learn to, literally, do it in their sleep.

\par
At the end of the day, there’s probably little reason to pay attention to our dreams at all unless they keep us from sleeping or “we wake up in a panic,” Cartwright says. Terrorism, economic uncertainties and general feelings of insecurity have increased people’s anxiety. Those suffering from persistent nightmares should seek help from a therapist. For the rest of us, the brain has its ways of working through bad feelings. Sleep -- or rather dream -- on it and you’ll feel better in the morning.
\\31.	Researchers have come to believe that dreams \underline{\hbox to 30mm{}}.\\$[A]$ can be modified in their courses\\$[B]$ are susceptible to emotional changes\\$[C]$ reflect our innermost desires and fears\\$[D]$ are a random outcome of neural repairs\\\\32.	By referring to the limbic system, the author intends to show \underline{\hbox to 30mm{}}.\\$[A]$ its function in our dreams\\$[B]$ the mechanism of REM sleep\\$[C]$ the relation of dreams to emotions\\$[D]$ its difference from the prefrontal cortex\\\\33.	The negative feelings generated during the day tend to \underline{\hbox to 30mm{}}.\\$[A]$ aggravate in our unconscious mind\\$[B]$ develop into happy dreams\\$[C]$ persist till the time we fall asleep\\$[D]$ show up in dreams early at night\\\\34.	Cartwright seems to suggest that \underline{\hbox to 30mm{}}.\\$[A]$ waking up in time is essential to the ridding of bad dreams\\$[B]$ visualizing bad dreams helps bring them under control\\$[C]$ dreams should be left to their natural progression\\$[D]$ dreaming may not entirely belong to the unconscious\\\\35.	What advice might Cartwright give to those who sometimes have bad dreams?\\$[A]$ Lead your life as usual.\\$[B]$ Seek professional help.\\$[C]$ Exercise conscious control.\\$[D]$ Avoid anxiety in the daytime.\\\subsection{2006年}
\subsubsection{Text2}

\par
Stratford-on-Avon, as we all know, has only one industry -- William Shakespeare -- but there are two distinctly separate and increasingly hostile branches. There is the Royal Shakespeare Company (RSC), which presents superb productions of the plays at the Shakespeare Memorial Theatre on the Avon. And there are the townsfolk who largely live off the tourists who come, not to see the plays, but to look at Anne Hathaway’s Cottage, Shakespeare’s birthplace and the other sights.

\par
The worthy residents of Stratford doubt that the theatre adds a penny to their revenue. They frankly dislike the RSC’s actors, them with their long hair and beards and sandals and noisiness. It’s all deliciously ironic when you consider that Shakespeare, who earns their living, was himself an actor (with a beard) and did his share of noise-making.

\par
The tourist streams are not entirely separate. The sightseers who come by bus -- and often take in Warwick Castle and Blenheim Palace on the side -- don’t usually see the plays, and some of them are even surprised to find a theatre in Stratford. However, the playgoers do manage a little sight-seeing along with their playgoing. It is the playgoers, the RSC contends, who bring in much of the town’s revenue because they spend the night (some of them four or five nights) pouring cash into the hotels and restaurants. The sightseers can take in everything and get out of town by nightfall.

\par
The townsfolk don’t see it this way and local council does not contribute directly to the subsidy of the Royal Shakespeare Company. Stratford cries poor traditionally. Nevertheless every hotel in town seems to be adding a new wing or cocktail lounge. Hilton is building its own hotel there, which you may be sure will be decorated with Hamlet Hamburger Bars, the Lear Lounge, the Banquo Banqueting Room, and so forth, and will be very expensive.

\par
Anyway, the townsfolk can’t understand why the Royal Shakespeare Company needs a subsidy. (The theatre has broken attendance records for three years in a row. Last year its 1,431 seats were 94 percent occupied all year long and this year they’ll do better.) The reason, of course, is that costs have rocketed and ticket prices have stayed low.

\par
It would be a shame to raise prices too much because it would drive away the young people who are Stratford’s most attractive clientele. They come entirely for the plays, not the sights. They all seem to look alike (though they come from all over) -- lean, pointed, dedicated faces, wearing jeans and sandals, eating their buns and bedding down for the night on the flagstones outside the theatre to buy the 20 seats and 80 standing-room tickets held for the sleepers and sold to them when the box office opens at 10:30 a.m.
\\26.	From the first two paragraphs, we learn that \underline{\hbox to 30mm{}}.\\$[A]$ the townsfolk deny the RSC’s contribution to the town’s revenue\\$[B]$ the actors of the RSC imitate Shakespeare on and off stage\\$[C]$ the two branches of the RSC are not on good terms\\$[D]$ the townsfolk earn little from tourism\\\\27.	It can be inferred from Paragraph 3 that \underline{\hbox to 30mm{}}.\\$[A]$ the sightseers cannot visit the Castle and the Palace separately\\$[B]$ the playgoers spend more money than the sightseers\\$[C]$ the sightseers do more shopping than the playgoers\\$[D]$ the playgoers go to no other places in town than the theater\\\\28.	By saying “Stratford cries poor traditionally” (Line 2-3, Paragraph 4), the author implies that \underline{\hbox to 30mm{}}.\\$[A]$ Stratford cannot afford the expansion projects\\$[B]$ Stratford has long been in financial difficulties\\$[C]$ the town is not really short of money\\$[D]$ the townsfolk used to be poorly paid\\\\29.	According to the townsfolk, the RSC deserves no subsidy because \underline{\hbox to 30mm{}}.\\$[A]$ ticket prices can be raised to cover the spending\\$[B]$ the company is financially ill-managed\\$[C]$ the behavior of the actors is not socially acceptable\\$[D]$ the theatre attendance is on the rise\\\\30.	From the text we can conclude that the author \underline{\hbox to 30mm{}}.\\$[A]$ is supportive of both sides\\$[B]$ favors the townsfolk’s view\\$[C]$ takes a detached attitude\\$[D]$ is sympathetic to the RSC\\\subsubsection{Text4}

\par
Many things make people think artists are weird. But the weirdest may be this: artists’ only job is to explore emotions, and yet they choose to focus on the ones that feel bad.

\par
This wasn’t always so. The earliest forms of art, like painting and music, are those best suited for expressing joy. But somewhere from the 19th century onward, more artists began seeing happiness as meaningless, phony or, worst of all, boring, as we went from Wordsworth’s daffodils to Baudelaire’s flowers of evil.

\par
You could argue that art became more skeptical of happiness because modern times have seen so much misery. But it’s not as if earlier times didn’t know perpetual war, disaster and the massacre of innocents. The reason, in fact, may be just the opposite: there is too much damn happiness in the world today.

\par
After all, what is the one modern form of expression almost completely dedicated to depicting happiness? Advertising. The rise of anti-happy art almost exactly tracks the emergence of mass media, and with it, a commercial culture in which happiness is not just an ideal but an ideology.

\par
People in earlier eras were surrounded by reminders of misery. They worked until exhausted, lived with few protections and died young. In the West, before mass communication and literacy, the most powerful mass medium was the church, which reminded worshippers that their souls were in danger and that they would someday be meat for worms. Given all this, they did not exactly need their art to be a bummer too.

\par
Today the messages the average Westerner is surrounded with are not religious but commercial, and forever happy. Fast-food eaters, news anchors, text messengers, all smiling, smiling, smiling. Our magazines feature beaming celebrities and happy families in perfect homes. And since these messages have an agenda -- to lure us to open our wallets -- they make the very idea of happiness seem unreliable. “Celebrate!” commanded the ads for the arthritis drug Celebrex, before we found out it could increase the risk of heart attacks.

\par
But what we forget -- what our economy depends on us forgetting -- is that happiness is more than pleasure without pain. The things that bring the greatest joy carry the greatest potential for loss and disappointment. Today, surrounded by promises of easy happiness, we need art to tell us, as religion once did, Memento mori: remember that you will die, that everything ends, and that happiness comes not in denying this but in living with it. It’s a message even more bitter than a clove cigarette, yet, somehow, a breath of fresh air.
\\36.	By citing the examples of poets Wordsworth and Baudelaire, the author intends to show that \underline{\hbox to 30mm{}}.\\$[A]$ poetry is not as expressive of joy as painting or music\\$[B]$ art grows out of both positive and negative feelings\\$[C]$ poets today are less skeptical of happiness\\$[D]$ artists have changed their focus of interest\\\\37.	The word “bummer” (Line 5, paragraph 5) most probably means something \underline{\hbox to 30mm{}}.\\$[A]$ religious\\$[B]$ unpleasant\\$[C]$ entertaining\\$[D]$ commercial\\\\38.	In the author’s opinion, advertising \underline{\hbox to 30mm{}}.\\$[A]$ emerges in the wake of the anti-happy art\\$[B]$ is a cause of disappointment for the general public\\$[C]$ replaces the church as a major source of information\\$[D]$ creates an illusion of happiness rather than happiness itself\\\\39.	We can learn from the last paragraph that the author believes \underline{\hbox to 30mm{}}.\\$[A]$ happiness more often than not ends in sadness\\$[B]$ the anti-happy art is distasteful but refreshing\\$[C]$ misery should be enjoyed rather than denied\\$[D]$ the anti-happy art flourishes when economy booms\\\\40.	Which of the following is true of the text?\\$[A]$ Religion once functioned as a reminder of misery.\\$[B]$ Art provides a balance between expectation and reality.\\$[C]$ People feel disappointed at the realities of modern society.\\$[D]$ Mass media are inclined to cover disasters and deaths.\\\subsection{2007年}
\subsubsection{Text2}

\par
For the past several years, the Sunday newspaper supplement Parade has featured a column called “Ask Marilyn.” People are invited to query Marilyn vos Savant, who at age 10 had tested at a mental level of someone about 23 years old; that gave her an IQ of 228 – the highest score ever recorded. IQ tests ask you to complete verbal and visual analogies, to envision paper after it has been folded and cut, and to deduce numerical sequences, among other similar tasks. So it is a bit confusing when vos Savant fields such queries from the average Joe (whose IQ is 100) as, What’s the difference between love and fondness? Or what is the nature of luck and coincidence? It’s not obvious how the capacity to visualize objects and to figure out numerical patterns suits one to answer questions that have eluded some of the best poets and philosophers.

\par
Clearly, intelligence encompasses more than a score on a test. Just what does it mean to be smart? How much of intelligence can be specified, and how much can we learn about it from neurology, genetics, computer science and other fields?

\par
The defining term of intelligence in humans still seems to be the IQ score, even though IQ tests are not given as often as they used to be. The test comes primarily in two forms: the Stanford-Binet Intelligence Scale and the Wechsler Intelligence Scales (both come in adult and children’s version). Generally costing several hundred dollars, they are usually given only by psychologists, although variations of them populate bookstores and the World Wide Web. Superhigh scores like vos Savant’s are no longer possible, because scoring is now based on a statistical population distribution among age peers, rather than simply dividing the mental age by the chronological age and multiplying by 100. Other standardized tests, such as the Scholastic Assessment Test (SAT) and the Graduate Record Exam (GRE), capture the main aspects of IQ tests.

\par
Such standardized tests may not assess all the important elements necessary to succeed in school and in life, argues Robert J. Sternberg. In his article “How Intelligent Is Intelligence Testing?”, Sternberg notes that traditional test best assess analytical and verbal skills but fail to measure creativity and practical knowledge, components also critical to problem solving and life success. Moreover, IQ tests do not necessarily predict so well once populations or situations change. Research has found that IQ predicted leadership skills when the tests were given under low-stress conditions, but under high-stress conditions, IQ was negatively correlated with leadership – that is, it predicted the opposite. Anyone who has toiled through SAT will testify that test-taking skill also matters, whether it’s knowing when to guess or what questions to skip.
\\26.	Which of the following may be required in an intelligence test?\\$[A]$ Answering philosophical questions.\\$[B]$ Folding or cutting paper into different shapes.\\$[C]$ Telling the differences between certain concepts.\\$[D]$ Choosing words or graphs similar to the given ones.\\\\27.	What can be inferred about intelligence testing from Paragraph 3?\\$[A]$ People no longer use IQ scores as an indicator of intelligence.\\$[B]$ More versions of IQ tests are now available on the Internet.\\$[C]$ The test contents and formats for adults and children may be different.\\$[D]$ Scientists have defined the important elements of human intelligence.\\\\28.	People nowadays can no longer achieve IQ scores as high as vos Savant’s because\\$[A]$ the scores are obtained through different computational procedures.\\$[B]$ creativity rather than analytical skills is emphasized now.\\$[C]$ vos Savant’s case is an extreme one that will not repeat.\\$[D]$ the defining characteristic of IQ tests has changed.\\\\29.	We can conclude from the last paragraph that\\$[A]$ test scores may not be reliable indicators of one’s ability.\\$[B]$ IQ scores and SAT results are highly correlated.\\$[C]$ testing involves a lot of guesswork.\\$[D]$ traditional test are out of date.\\\\30.	What is the author’s attitude towards IQ tests?\\$[A]$ Supportive.\\$[B]$ Skeptical.\\$[C]$ Impartial.\\$[D]$ Biased.\\\subsubsection{Text4}

\par
It never rains but it pours. Just as bosses and boards have finally sorted out their worst accounting and compliance troubles, and improved their feeble corporation governance, a new problem threatens to earn them – especially in America – the sort of nasty headlines that inevitably lead to heads rolling in the executive suite: data insecurity. Left, until now, to odd, low-level IT staff to put right, and seen as a concern only of data-rich industries such as banking, telecoms and air travel, information protection is now high on the boss’s agenda in businesses of every variety.

\par
Several massive leakages of customer and employee data this year – from organizations as diverse as Time Warner, the American defense contractor Science Applications International Corp and even the University of California, Berkeley – have left managers hurriedly peering into their intricate IT systems and business processes in search of potential vulnerabilities.

\par
“Data is becoming an asset which needs to be guarded as much as any other asset,” says Haim Mendelson of Stanford University’s business school. “The ability to guard customer data is the key to market value, which the board is responsible for on behalf of shareholders.” Indeed, just as there is the concept of Generally Accepted Accounting Principles (GAAP), perhaps it is time for GASP, Generally Accepted Security Practices, suggested Eli Noam of New York’s Columbia Business School. “Setting the proper investment level for security, redundancy, and recovery is a management issue, not a technical one,” he says.

\par
The mystery is that this should come as a surprise to any boss. Surely it should be obvious to the dimmest executive that trust, that most valuable of economic assets, is easily destroyed and hugely expensive to restore – and that few things are more likely to destroy trust than a company letting sensitive personal data get into the wrong hands.

\par
The current state of affairs may have been encouraged – though not justified – by the lack of legal penalty (in America, but not Europe) for data leakage. Until California recently passed a law, American firms did not have to tell anyone, even the victim, when data went astray. That may change fast: lots of proposed data-security legislation is now doing the rounds in Washington, D.C. Meanwhile, the theft of information about some 40 million credit-card accounts in America, disclosed on June 17th, overshadowed a hugely important decision a day earlier by America’s Federal Trade Commission (FTC) that puts corporate America on notice that regulators will act if firms fail to provide adequate data security.
\\36.	The statement “It never rains but it pours” is used to introduce\\$[A]$ the fierce business competition.\\$[B]$ the feeble boss-board relations.\\$[C]$ the threat from news reports.\\$[D]$ the severity of data leakage.\\\\37.	According to Paragraph 2, some organizations check their systems to find out\\$[A]$ whether there is any weak point.\\$[B]$ what sort of data has been stolen.\\$[C]$ who is responsible for the leakage.\\$[D]$ how the potential spies can be located.\\\\38.	In bringing up the concept of GASP the author is making the point that\\$[A]$ shareholders’ interests should be properly attended to.\\$[B]$ information protection should be given due attention.\\$[C]$ businesses should enhance their level of accounting security.\\$[D]$ the market value of customer data should be emphasized.\\\\39.	According to Paragraph 4, what puzzles the author is that some bosses fail to\\$[A]$ see the link between trust and data protection.\\$[B]$ perceive the sensitivity of personal data.\\$[C]$ realize the high cost of data restoration.\\$[D]$ appreciate the economic value of trust.\\\\40.	It can be inferred from Paragraph 5 that\\$[A]$ data leakage is more severe in Europe.\\$[B]$ FTC’s decision is essential to data security.\\$[C]$ California takes the lead in security legislation.\\$[D]$ legal penalty is a major solution to data leakage.\\\subsection{2008年}
\subsubsection{Text4}

\par
In 1784, five years before he became president of the United States, George Washington, 52, was nearly toothless. So he hired a dentist to transplant nine teeth into his jaw – having extracted them from the mouths of his slaves.

\par
That’s a far different image from the cherry-tree-chopping George most people remember from their history books. But recently, many historians have begun to focus on the roles slavery played in the lives of the founding generation. They have been spurred in part by DNA evidence made available in 1998, which almost certainly proved Thomas Jefferson had fathered at least one child with his slave Sally Hemings. And only over the past 30 years have scholars examined history from the bottom up. Works of several historians reveal the moral compromises made by the nation’s early leaders and the fragile nature of the country’s infancy. More significantly, they argue that many of the Founding Fathers knew slavery was wrong – and yet most did little to fight it.

\par
More than anything, the historians say, the founders were hampered by the culture of their time. While Washington and Jefferson privately expressed distaste for slavery, they also understood that it was part of the political and economic bedrock of the country they helped to create.

\par
For one thing, the South could not afford to part with its slaves. Owning slaves was “like having a large bank account,” says Wiencek, author of An Imperfect God: George Washington, His Slaves, and the Creation of America. The southern states would not have signed the Constitution without protections for the “peculiar institution,” including a clause that counted a slave as three fifths of a man for purposes of congressional representation.

\par
And the statesmen’s political lives depended on slavery. The three-fifths formula handed Jefferson his narrow victory in the presidential election of 1800 by inflating the votes of the southern states in the Electoral College. Once in office, Jefferson extended slavery with the Louisiana Purchase in 1803; the new land was carved into 13 states, including three slave states.

\par
Still, Jefferson freed Hemings’s children – though not Hemings herself or his approximately 150 other slaves. Washington, who had begun to believe that all men were created equal after observing the bravery of the black soldiers during the Revolutionary War, overcame the strong opposition of his relatives to grant his slaves their freedom in his will. Only a decade earlier, such an act would have required legislative approval in Virginia.
\\36.	George Washington’s dental surgery is mentioned to\\$[A]$ show the primitive medical practice in the past.\\$[B]$ demonstrate the cruelty of slavery in his days.\\$[C]$ stress the role of slaves in the U.S. history.\\$[D]$ reveal some unknown aspect of his life.\\\\37.	We may infer from the second paragraph that\\$[A]$ DNA technology has been widely applied to history research.\\$[B]$ in its early days the U.S. was confronted with delicate situations.\\$[C]$ historians deliberately made up some stories of Jefferson’s life.\\$[D]$ political compromises are easily found throughout the U.S. history.\\\\38.	What do we learn about Thomas Jefferson?\\$[A]$ His political view changed his attitude towards slavery.\\$[B]$ His status as a father made him free the child slaves.\\$[C]$ His attitude towards slavery was complex.\\$[D]$ His affair with a slave stained his prestige.\\\\39.	Which of the following is true according to the text?\\$[A]$ Some Founding Fathers benefit politically from slavery.\\$[B]$ Slaves in the old days did not have the right to vote.\\$[C]$ Slave owners usually had large savings accounts.\\$[D]$ Slavery was regarded as a peculiar institution.\\\\40.	Washington’s decision to free slaves originated from his\\$[A]$ moral considerations.\\$[B]$ military experience.\\$[C]$ financial conditions.\\$[D]$ political stand.\\
\end{document}
        
